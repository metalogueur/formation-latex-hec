% Apparence du texte

\section{Text Appearance}

\subsection{Fonts}

% Polices de caractères
\begin{frame}[c]{Fonts}
	\begin{itemize}
		\item By default, all \LaTeX\  documents use the same font, \textrm{Computer Modern}.
		\item You should choose high-quality and complete fonts (diacritics, great choice of symbols).
		\item Very few fonts are adapted to maths: Palatino, Times, Lucida (\$) are safe choices.
		\item In the \textbf{hecthese} document class, mathptmx and mathpazo packages are preloaded so you can choose between Times and Palatino fonts.
	\end{itemize}
\end{frame}

% Changement d'attribut de la police
\begin{frame}{Font Attributes}
	\begin{tabularx}{\textwidth}{XXX}
		\arrayrulecolor{grisPrimaire!40}\hline\hline
		\multicolumn{3}{l}{\textbf{families}}	\\
		\hline
		\textrm{roman}						&	\cmd{rmfamily}		&	\cmd{textrm\{<text>\}}\\
		\texttt{fixed width}				&	\cmd{ttfamily}		&	\cmd{texttt\{<text>\}}\\
		sans serif							&	\cmd{sffamily}		&	\cmd{textsf\{<text>\}}\\
		\hline
		\multicolumn{3}{l}{\textbf{shapes}}	\\
		\hline
		upright								&	\cmd{upshape}		&	\cmd{textup\{<text>\}}\\
		\emph{italic}						&	\cmd{itshape}		&	\cmd{textit\{<text>\}}\\
		\textsl{slanted}					&	\cmd{slshape}		&	\cmd{textsl\{<text>\}}\\
		\textrm{\textsc{small caps}}		&	\cmd{scshape}		&	\cmd{textsc\{<text>\}}\\
		\hline
		\multicolumn{3}{l}{\textbf{series}}	\\
		\hline
		\textmd{medium}						&	\cmd{mdseries}		&	\cmd{textmd\{<text>\}}\\
		\textbf{bold}						&	\cmd{bfseries}		&	\cmd{textbf\{<text>\}}\\
		\hline\hline
	\end{tabularx}

	\begin{picture}(0,0)
	\thicklines\color{bleuFonceSecondaire}
	\onslide<2>\put(38,-7){\dashbox{1}(40,64)[b]{\parbox{.25\textwidth}{\centering\textbf{applies to all the following text}}}}
	\onslide<3>\put(92,-7){\dashbox{1}(40,64)[b]{\parbox{.25\textwidth}{\centering\textbf{applies to the text in the command}}}}
	\end{picture}
\end{frame}

% Italique
\begin{frame}[fragile,c]{Italics}
	 When using italics to put \emph{emphasis} on parts of a text, you should use the following semantic command instead:
	 
\begin{codesource}
	\emph{text}
\end{codesource}

	\cmd{emph\{<text>\}} commands can be nested in one another. Text in italics becomes upright and vice versa.
	
	\begin{columns}
		\begin{column}{.49\textwidth}
			\vspace{-5.5mm}
\begin{codesource}
	This house lacks a certain
	\emph{je ne sais quoi}\ldots
\end{codesource}
		\end{column}
		\begin{column}{.49\textwidth}
			This house lacks a certain
			\emph{je ne sais quoi}\ldots
		\end{column}
	\end{columns}

	\begin{columns}
		\begin{column}{.49\textwidth}
			\vspace{-5.5mm}
\begin{codesource}
	He said: « \emph{Enough 
	\emph{poutine} for the week!»}
\end{codesource}
		\end{column}
		\begin{column}{.49\textwidth}
			He said: « \emph{Enough \textup{poutine} for the week!»}
		\end{column}
	\end{columns}
	
\end{frame}

% Taille de la police
\begin{frame}{Font size}
	\begin{tabularx}{\textwidth}{l|l}
		\arrayrulecolor{grisPrimaire!40}
		\textbf{Standard commands} 		& 	\textbf{Size}	\\
		\hline
		\cmd{tiny}						&	{\tiny tiny}	\\
		\cmd{scriptsize}				&	{\scriptsize script size}	\\
		\cmd{footnotesize}				&	{\footnotesize footnote size}	\\
		\cmd{small}						&	{\small small}	\\
		\cmd{normalsize}				&	{\normalsize normal size}	\\
		\cmd{large}						&	{\large large}	\\
		\cmd{Large}						&	{\Large larger}	\\
		\cmd{LARGE}						&	{\LARGE largest}	\\
		\cmd{huge}						&	{\huge huge}	\\
		\cmd{Huge}						&	{\Huge humongous}		
	\end{tabularx}
\end{frame}

\subsection{Displaying text}

% Listes
\begin{frame}[fragile]{Lists}
	\begin{itemize}
		\item Two main types of lists:
		\begin{enumerate}
			\item \textbf{unordered} with the \texttt{itemize} environment
			\item \textbf{ordered} with the \texttt{enumerate} environment
		\end{enumerate}
		\item Lists can be nested into one another
		\item Markers are adapted to up to four nesting levels

		\pause
\begin{codesource}
	\begin{itemize}
		\item Two main types of lists:
		\begin{enumerate}
			\item \textbf{unordered} with the \verb=itemize= environment
			\item \textbf{ordered} with the \verb=enumerate= environment
		\end{enumerate}
		\item Lists can be nested into one another
		\item Markers are adapted to up to four nesting levels
	\end{itemize}
\end{codesource}

		\pause
		\item A third list type is available: \texttt{description}
	\end{itemize}
\end{frame}

% Citations
\begin{frame}[fragile,c]{Quotations}
	\framesubtitle<1>{Short Quotes}
	\framesubtitle<2>{Long Quotations}
	\begin{onlyenv}<1>
		We use the \texttt{quote} environment to insert a short, one-paragraph quote in our text.
	
		\begin{columns}
			\begin{column}{.49\textwidth}
				\vspace{-17mm}
\begin{codesource}
	\begin{quote}
		Life is what happens to you while 
		you're busy making other plans. 
		-- John Lennon
	\end{quote}
\end{codesource}
			\end{column}
		
			\begin{column}{.49\textwidth}
				\begin{quote}
					Life is what happens to you while you're busy making other plans. -- John Lennon
				\end{quote}
			\end{column}
		\end{columns}
	\end{onlyenv}

	\begin{onlyenv}<2>
		We use the \texttt{quotation} environment to insert long quotations.
		
		\begin{columns}
			\begin{column}{.49\textwidth}
				\vspace{-38mm}
\begin{codesource}
	\begin{quotation}
		I've missed more than 9000 shots in my 
		career. I've lost almost 300 games. 26 
		times I've been trusted to take the game 
		winning shot and missed.
		
		I've failed over and over and over again 
		in my life. And that is why I succeed. 
		-- Michael Jordan
	\end{quotation}
\end{codesource}	
			\end{column}
		
			\begin{column}{.49\textwidth}
\begin{quotation}
	I've missed more than 9000 shots in my career. 
	I've lost almost 300 games. 26 times I've been 
	trusted to take the game winning shot and missed.
	
	I've failed over and over and over again in my life. 
	And that is why I succeed. -- Michael Jordan
\end{quotation}
			\end{column}
		\end{columns}
	\end{onlyenv}
\end{frame}

% Notes de bas de page
\begin{frame}[fragile,c]{Footnotes}
	\begin{itemize}
		\item A footnote is inserted with the following command:
\begin{codesource}
	\footnote{text}
\end{codesource}
		\item The command must immediately follow the annotated text.
		\item Recommended method:
\begin{codesource}
	... fera remarquer que Pierre Lasou\footnote{%
	Spécialiste en ressources documentaires} %
	fut une grande aide dans la préparation de ...
\end{codesource}
		\item Footnote numbering and display are automatically generated.
	\end{itemize}
\end{frame}

% Code source
\begin{frame}[fragile,c]{Source Code}
	\begin{onlyenv}<1>
		\begin{itemize}
			\item To write source code in our text, we use the \texttt{verbatim} environment.
\begin{codesource}
	\begin{verbatim}
		Text displayed as is in a
		fixed-width font.
	\end{verbatim}
\end{codesource}
			\item To write source code inline in our text, we use the \cmd{verb} command. Its syntax is \cmd{verbcsourcec} where \emph{c} can be any character not found in \emph{source}.
			\item For more thorough inclusions of source code, you should use the \textbf{listings} package.
		\end{itemize}
	\end{onlyenv}

	\begin{onlyenv}<2>
	Example\footnote{taken from the \href{http://r4stats.com/examples/programming/}{r4stats.com} Website.} :
\begin{codesource}
	# ---Writing Your Own Functions (Macros)---
	
	# A good function that just prints.
	mystats <- function(x) {
		print( mean(x, na.rm = TRUE) )
		print(   sd(x, na.rm = TRUE) )
	}
	mystats(myvar)
	
	# A function with vector output.
	mystats  <- function(x) {
		mymean <- mean(x, na.rm = TRUE)
		mysd   <-   sd(x, na.rm = TRUE)
		c(mean = mymean, sd = mysd )
	}
	mystats(myvar)
	myVector <- mystats(myvar)
	myVector
\end{codesource}
	\end{onlyenv}
\end{frame}