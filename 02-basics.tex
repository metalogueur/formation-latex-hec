% Principes de base

\section{The Basics}

\subsection{Document Structure}

% Structure d'un document
\begin{frame}[fragile]{Document structure}

	A \LaTeX\ document always has two parts:
	
\begin{codesource}
	
	\documentclass[11pt,french]{article}
	\usepackage[utf8]{inputenc}
	\usepackage[T1]{fontenc}
	\usepackage{babel}
	\usepackage[autolanguage]{numprint}
	
	\begin{document}
		
		\section{Primo}
		
		Ac class dis donec erat facilisis magna mattis 
		placerat potenti praesent primis sed tellus turpis 
		ut vehicula. Ad amet eleifend eros fames habitant 
		imperdiet integer laoreet leo magna magnis neque 
		netus senectus taciti torquent. 
		
		\section{Deuxio}
		
		Cursus dui egestas eget eros et hac magna massa mollis 
		natoque penatibus sagittis sed tellus urna velit 
		vestibulum vitae vulputate. 
	\end{document}
\end{codesource}

	\begin{picture}(0,0)
		\thicklines\color{bleuFonceSecondaire}
		\onslide<2>\put(1,47){\dashbox{1}(87,15){}}
		\onslide<2>\put(89,53){\Large\textbf{\faArrowLeft\ Preamble}}
		\onslide<3>\put(1,6){\dashbox{1}(87,41){}}
		\onslide<3>\put(89,24){\Large\textbf{\faArrowLeft\ Document body}}
	\end{picture}
\end{frame}

% Préambule : la classe de document
\begin{frame}[fragile]{Preamble}
	\framesubtitle{Document Class}
	The preamble's \textbf{first command} usually is the document class declaration.
	
\begin{codesource}
	\documentclass[options]{class}	
\end{codesource}

	\begin{columns}
		
		\pause
		
		\begin{HECcomparaison}{Main classes}
			\begin{itemize}
				\item article, book, letter, report
				\item memoir, \textbf{hecthese}
				\item slides, beamer, \textbf{hecppt}
			\end{itemize}
		\end{HECcomparaison}
	
		\pause
		
		\begin{HECcomparaison}{Main options}
			\begin{itemize}
				\item 10pt, 11pt, 12pt
				\item oneside, twoside
				\item openright, openany
				\item english, french
			\end{itemize}
		\end{HECcomparaison}
	\end{columns}
\end{frame}

% Préambule : les packages
\begin{frame}[fragile,c]{Preamble}
	\framesubtitle{Packages}
	Packages allow you to \textbf{modify existing commands} and to \textbf{add features} to the system.
	
	They are loaded in the preamble with the \cmd{usepackage[options]\{package\}} command.
	
\begin{codesource}
	\documentclass[options]{class}
	
	\usepackage{package}
	\usepackage[options]{package}
	\usepackage{package1,package2,package3,...}
\end{codesource}

	Each package's documentation can be found on the 
	\href{https://ctan.org/}{Comprehensive \TeX\ Archive Network} Website.
\end{frame}

% Commandes
\begin{frame}[fragile]{Commands}
	\begin{itemize}
		\item Always begin with a \textbackslash
		\item Three main forms:
\begin{codesource}
	\commandname[optional_args]{mandatory_args}
	\commandname*[optional_args]{mandatory_args}
	\commandname
\end{codesource}
		\item Mandatory arguments between \{\ and \}
		\item Optional arguments between [ and ]
		\item Commands without arguments: the command's name ends with any character that isn't a letter or with a blank space.
		\item A command's scope is limited between \{\ and \}.
	\end{itemize}
\end{frame}

% Environnements
\begin{frame}[fragile,c]{Environments}
	\begin{itemize}
		\item Delimited by
\begin{codesource}
	\begin{environment}
		...
	\end{environment}
\end{codesource}
		\item An environment's content is treated differently from the remainder of the text.
		\item Changes apply only to the environment's content.
	\end{itemize}
\end{frame}

\subsection{Writing}

% Rédaction
\begin{frame}[fragile,c]{Writing}
	\begin{itemize}
		\item You write your text in the \texttt{document} environment:
\begin{codesource}
	\begin{document}
		The content of your document goes here...
	\end{document}
\end{codesource}
		\item You write your document in plain text and use commands and environments to structure your text;
		\item You write your text like anywhere else:
			\begin{itemize}
				\item Words are separated by one or more blank spaces;
				\item Paragraphs are separated by one or more empty lines;
				\item All extra white space is deleted on compilation.
			\end{itemize}
	\end{itemize}
\end{frame}

% Caractères spéciaux
\begin{frame}{Reserved Characters}
	\framesubtitle{\TeX's Reserved Characters}
	\begin{description}[\#]
		\item[\#] Argument number in commands
		\item[\$] Math Mode delimiter
		\item[\&] Table column delimiter
		\item[\%] Starts a comment
		\item[\_] Indices (math)
		\item[\textasciicircum] Exponents (math)
		\item[\textasciitilde] No-break space
		\item[\{] Opens a command or an environment definition
		\item[\}] Closes a command or an environment definition
	\end{description}
	\begin{picture}(0,0)
	\thicklines\color{bleuFonceSecondaire}
	\onslide<2>\put(90,5){\dashbox{1}(53,58){}}
	\onslide<2>\put(92,59){\textbf{\MakeUppercase{To use the characters:}}}
	\onslide<2>\put(94,55){\parbox[t]{.3\textwidth}{\centering\bfseries\textbackslash \# \\[5pt] %
			\textbackslash \$ \\[5pt] \textbackslash \& \\[5pt] \textbackslash \% \\[5pt] %
			\textbackslash \_ \\[5pt] \textbackslash textasciicircum \\[4pt] %
			\textbackslash textasciitilde \\[4pt] \textbackslash \{ \\[4pt] %
			\textbackslash \} }}
	\end{picture}
\end{frame}

% Caractères spéciaux - la suite
\begin{frame}[fragile,c]{Reserved Characters}
	\begin{itemize}
		\item Quotation marks
			\begin{itemize}
				\item The quotation marks " found on a keyboard are not used in typesetting.
				\item Single (\lstinline|`|) or double (\lstinline|``|) beginning marks and single (\lstinline|'|) or double (\lstinline|''|) end marks are used to surround quotes.
			\end{itemize}
		\item We type hyphens once (\lstinline|-|), twice (\lstinline|--|) or three times (\lstinline|---|) to produce hyphens, \emph{en dash}es and \emph{em dash}es.
	\end{itemize}
\end{frame}