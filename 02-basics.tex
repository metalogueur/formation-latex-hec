% Principes de base

\section{The Basics}

\subsection{Document Structure}

% Structure d'un document
\begin{frame}[fragile]{Document structure}

	Un document \LaTeX\ est toujours composé de deux parties :
	
\begin{codesource}
	
	\documentclass[11pt,french]{article}
	\usepackage[utf8]{inputenc}
	\usepackage[T1]{fontenc}
	\usepackage{babel}
	\usepackage[autolanguage]{numprint}
	
	\begin{document}
		
		\section{Primo}
		
		Ac class dis donec erat facilisis magna mattis 
		placerat potenti praesent primis sed tellus turpis 
		ut vehicula. Ad amet eleifend eros fames habitant 
		imperdiet integer laoreet leo magna magnis neque 
		netus senectus taciti torquent. 
		
		\section{Deuxio}
		
		Cursus dui egestas eget eros et hac magna massa mollis 
		natoque penatibus sagittis sed tellus urna velit 
		vestibulum vitae vulputate. 
	\end{document}
\end{codesource}

	\begin{picture}(0,0)
		\thicklines\color{bleuFonceSecondaire}
		\onslide<2>\put(1,47){\dashbox{1}(87,15){}}
		\onslide<2>\put(89,53){\Large\textbf{\faArrowLeft\ Préambule}}
		\onslide<3>\put(1,6){\dashbox{1}(87,41){}}
		\onslide<3>\put(89,24){\Large\textbf{\faArrowLeft\ Corps du document}}
	\end{picture}
\end{frame}

% Préambule : la classe de document
\begin{frame}[fragile]{Préambule}
	\framesubtitle{La classe de document}
	La \textbf{première commande} du préambule est normalement la déclaration de la classe.
	
\begin{codesource}
	\documentclass[options]{classe}	
\end{codesource}

	\begin{columns}
		
		\pause
		
		\begin{HECcomparaison}{Principales classes}
			\begin{itemize}
				\item article, book, letter, report
				\item memoir, \textbf{hecthese}
				\item slides, beamer, \textbf{hecppt}
			\end{itemize}
		\end{HECcomparaison}
	
		\pause
		
		\begin{HECcomparaison}{Principales options}
			\begin{itemize}
				\item 10pt, 11pt, 12pt
				\item oneside, twoside
				\item openright, openany
				\item english, french
			\end{itemize}
		\end{HECcomparaison}
	\end{columns}
\end{frame}

% Préambule : les packages
\begin{frame}[fragile,c]{Préambule}
	\framesubtitle{Les \emph{packages}}
	Les \emph{packages} permettent de \textbf{modifier des commandes} ou d’\textbf{ajouter des fonctionnalités} au système.
	
	Ils sont chargés dans le préambule avec la commande \cmd{usepackage[options]\{package\}}.
	
\begin{codesource}
	\documentclass[options]{classe}
	
	\usepackage{package}
	\usepackage[options]{package}
	\usepackage{package1,package2,package3,...}
\end{codesource}

	La documentation de chaque package peut être consultée sur le site du
	\href{https://ctan.org/}{Comprehensive \TeX\ Archive Network}.
\end{frame}

% Commandes
\begin{frame}[fragile]{Commandes}
	\begin{itemize}
		\item Débutent toujours par un \textbackslash
		\item Formes générales:
\begin{codesource}
	\nomcommande[args_optionnels]{args_obligatoires}
	\nomcommande*[args_optionnels]{args_obligatoires}
	\nomcommande
\end{codesource}
		\item Arguments obligatoires entre \{\ et \}
		\item Arguments optionnels entre [ et ]
		\item Commande sans argument : le nom se termine par tout caractère qui n’est pas une lettre (y
		compris l’espace)
		\item Portée d’une commande limitée à la zone entre \{\ et \}.
	\end{itemize}
\end{frame}

% Environnements
\begin{frame}[fragile,c]{Environnements}
	\begin{itemize}
		\item Délimités par
\begin{codesource}
	\begin{environnement}
		...
	\end{environnement}
\end{codesource}
		\item Contenu de l’environnement traité différemment du reste du texte
		\item Changements s’appliquent uniquement à l’intérieur de l’environnement
	\end{itemize}
\end{frame}

\subsection{Rédaction}

% Rédaction
\begin{frame}[fragile,c]{Rédaction}
	\begin{itemize}
		\item On rédige notre texte à l'intérieur de l'environnement \texttt{document}:
\begin{codesource}
	\begin{document}
		Le contenu de votre travail est rédigé ici...
	\end{document}
\end{codesource}
		\item On rédige notre document en texte brut et on utilise les commandes et les environnements
		pour structurer notre texte;
		\item On rédige notre texte comme n'importe où ailleurs:
			\begin{itemize}
				\item Les mots sont séparés par un ou plusieurs espaces;
				\item Les paragraphes sont séparés par une ou plusieurs lignes blanches;
				\item Tous les espaces blancs supplémentaires sont supprimés à la compilation.
			\end{itemize}
	\end{itemize}
\end{frame}

% Caractères spéciaux
\begin{frame}{Caractères spéciaux}
	\framesubtitle{Caractères réservés par \TeX}
	\begin{description}[\#]
		\item[\#] Numéro d'argument dans les commandes
		\item[\$] Délimiteur du mode mathématique
		\item[\&] Délimiteur de colonne dans les tableaux
		\item[\%] Annonce le début d'un commentaire
		\item[\_] Indice (mathématiques)
		\item[\textasciicircum] Exposant (mathématiques)
		\item[\textasciitilde] Espace insécable
		\item[\{] Ouvre une définition de commande ou d'environnement
		\item[\}] Ferme une définition de commande ou d'environnement
	\end{description}
	\begin{picture}(0,0)
	\thicklines\color{bleuFonceSecondaire}
	\onslide<2>\put(90,5){\dashbox{1}(53,58){}}
	\onslide<2>\put(97,59){\textbf{\MakeUppercase{Pour les utiliser:}}}
	\onslide<2>\put(94,55){\parbox[t]{.3\textwidth}{\centering\bfseries\textbackslash \# \\[5pt] %
			\textbackslash \$ \\[5pt] \textbackslash \& \\[5pt] \textbackslash \% \\[5pt] %
			\textbackslash \_ \\[5pt] \textbackslash textasciicircum \\[4pt] %
			\textbackslash textasciitilde \\[4pt] \textbackslash \{ \\[4pt] %
			\textbackslash \} }}
	\end{picture}
\end{frame}

% Diacritiques et LaTeX
\begin{frame}[fragile,c]{Diacritiques dans \LaTeX}
	\LaTeX\ ne supporte pas les diacritiques de manière native.
	\begin{columns}				
		\begin{column}{.49\textwidth}
			\vspace{-5.2mm}
\begin{codesource}
	 	\'{E}crire \`{a} la fran\c{c}aise
	 	peut \^{e}tre vraiment p\'{e}nible
	 	si on ne conna\^{i}t pas le truc\ldots
\end{codesource}
		\end{column}
		\begin{column}{.49\textwidth}
			Écrire à la française peut être vraiment pénible si on ne connaît pas
			le truc\ldots
		\end{column}
	\end{columns}

	On peut apprendre la \href{https://en.wikibooks.org/wiki/LaTeX/Special_Characters#Escaped_codes}{liste des commandes} 
	par coeur\ldots ou on peut ajouter des fonctionnalités à \LaTeX\ pour le franciser.
\end{frame}

% LaTeX en français - préambule pour pdfLaTeX
\begin{frame}[fragile]{\LaTeX\ en français -- préambule pour pdf\LaTeX}
	Il faut charger un certain nombre de \emph{packages} pour franciser \LaTeX.
	
\begin{codesource}
	\documentclass[french]{hecthese}
	\usepackage[utf8]{inputenc}
	\usepackage[T1]{fontenc}
	\usepackage{babel}
	\usepackage[autolanguage]{numprint}
	\usepackage{icomma}
\end{codesource}

	\pause
	\begin{description}[inputenc et fontenc]
		\item[babel] traduction des mots-clés prédéfinis, typographie française, coupure de mots,
			document multilingue
			
		\pause
		\item[inputenc et fontenc] lettres accentuées dans le code source
		
		\pause
		\item[icomma] virgule comme séparateur décimal
		
		\pause
		\item[numprint] espace comme séparateur de milliers
	\end{description}
\end{frame}

% Caractères spéciaux - la suite
\begin{frame}[fragile]{Caractères spéciaux}
	\framesubtitle{La suite\ldots}
	\begin{itemize}
		\item Guillemets
			\begin{itemize}
				\item On ouvre les guillemets anglais simples avec un accent grave (\lstinline|`|)
					et les doubles avec deux accents graves (\lstinline|``|). On les ferme avec un 
					(\lstinline|'|) ou deux (\lstinline|''|) apostrophes, selon la situation.
				\item On utilise les chevrons (« et ») pour ouvrir et fermer les guillemets français.
					Il faut cependant inscrire la commande suivante à la fin de notre préambule:
\begin{codesource}
	\frenchbsetup{og=«,fg=»}
\end{codesource}				
			\end{itemize}
		\item On inscrit les traits d'union avec un tiret (\lstinline|-|), les traits demi-cadratins avec deux tirets (\lstinline|--|) et les traits cadratins avec trois tirets (\lstinline|---|).
	\end{itemize}
\end{frame}