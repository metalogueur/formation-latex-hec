% version-2
\section{B-A-BA des mathématiques}

% ----------------- %
% Principes de base %
% ----------------- %

\begin{frame}[fragile]

	\frametitle{Principes de base}
	
	\begin{itemize}
		\item Décrire des équations mathématiques requiert un « langage » spécial
		\begin{itemize}
			\item il faut informer {\LaTeX} que l’on passe à ce langage
			\item par le biais de modes mathématiques
		\end{itemize}
	
		\item Important d’utiliser un mode mathématique
		\begin{itemize}
			\item règles de typographie spéciales (constantes vs variables, disposition des équations, 	numérotation, etc.)
			\item espaces entre les symboles et autour des opérateurs gérées automatiquement
		\end{itemize}
	
		\item Vous voulez utiliser le paquetage \textbf{amsmath}
			\begin{codesource}
	\usepackage{amsmath}	
			\end{codesource}

			\begin{itemize}
				\item lire la documentation de ce paquetage pour connaître toutes ses fonctionnalités
			\end{itemize}
	\end{itemize}
\end{frame}

% ------------------- %
% Modes mathématiques %
% ------------------- %

\begin{frame}[fragile]

	\frametitle{Modes mathématiques}
	
	\begin{enumerate}
		\item « En ligne » directement dans le texte comme %
		$(a + b)^2 = a^2 + 2ab + b^2$ en plaçant l’équation entre \$\ et \$
			\begin{codesource}
	«En ligne» directement dans le texte comme $(a + b)^2 = a^2 + 2ab + b^2$
			\end{codesource}
		
		\item « Hors paragraphe » séparé du texte principal comme
		\begin{equation*}
			\int_0^\infty f(x)\, dx =
			\sum_{i = 1}^n \alpha_i e^{x_i} f(x_i)
		\end{equation*}
			en utilisant divers types d’environnements
			\begin{codesource}
	\begin{equation*}
		\int_0^\infty f(x)\, dx =
		\sum_{i = 1}^n \alpha_i e^{x_i} f(x_i)
	\end{equation*}	
			\end{codesource}
	\end{enumerate}
\end{frame}

% ----------------------- %
% Quelques règles de base %
% ----------------------- %

\begin{frame}[c,fragile]

	\frametitle{Quelques règles de base}
	
	\begin{itemize}
		\item En mode mathématique, \TeX\ respecte automatiquement la convention d’écrire les constantes en romain et les variables en italique
		\begin{columns}
			\column{.4\textwidth}
			\vspace{-1.4em}
			\begin{codesource}
	$z = 2a + 3y$	
			\end{codesource}
			\column{.4\textwidth}
				$z = 2a + 3y$
		\end{columns}
	
		\item Espacement entre les éléments géré automatiquement, peu importe le code source
		\begin{columns}
			\column{.4\textwidth}
			\vspace{-1.4em}
				\begin{codesource}
	$z=2 a+3 y$
				\end{codesource}
			\column{.4\textwidth}
				$z=2 a+3 y$
		\end{columns}
	
		\item Utiliser la commande \lstinline|\text{}| de \texttt{amsmath} pour composer du
			texte à l’intérieur du mode mathématique
		\begin{columns}
			\column{.4\textwidth}
			\vspace{-1.4em}
				\begin{codesource}
	$x = 0 \text{ si } y < 2$
				\end{codesource}
			\column{.4\textwidth}
				$x = 0 \text{ si } y < 2$
		\end{columns}
	\end{itemize}
\end{frame}

% ---------- %
% Avant-goût %
% ---------- %

\begin{frame}[fragile]

	\frametitle{Avant-goût}
	
	Pouvez-vous interpréter ce code?
	
	\begin{codesource}
	\begin{equation*}
		\Gamma(\alpha) =
		\sum_{j = 0}^\infty \int_j^{j + 1}
		x^{\alpha - 1} e^{-x}\, dx
	\end{equation*}
	\end{codesource}

	\pause

	Fort probablement!
	
	\begin{equation*}
		\Gamma(\alpha) =
		\sum_{j = 0}^\infty \int_j^{j + 1}
		x^{\alpha - 1} e^{-x}\, dx
	\end{equation*}
\end{frame}