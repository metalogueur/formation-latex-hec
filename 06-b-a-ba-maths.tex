\section{B-A-BA des mathématiques}

% ----------------- %
% Principes de base %
% ----------------- %

\begin{frame}[fragile]

	\frametitle{Principes de base}
	
	\begin{itemize}
		\item Décrire des équations mathématiques requiert un « langage » spécial
		\begin{itemize}
			\item il faut informer {\LaTeX} que l’on passe à ce langage
			\item par le biais de modes mathématiques
		\end{itemize}
	
		\item Important d’utiliser un mode mathématique
		\begin{itemize}
			\item règles de typographie spéciales (constantes vs variables, disposition des équations, 	numérotation, etc.)
			\item espaces entre les symboles et autour des opérateurs gérées automatiquement
		\end{itemize}
	
		\item Vous voulez utiliser le paquetage \textbf{amsmath} \\
			\lstinline|\usepackage{amsmath}|
			\begin{itemize}
				\item lire la documentation de ce paquetage pour connaître toutes ses fonctionnalités
			\end{itemize}
	\end{itemize}
\end{frame}

% ------------------- %
% Modes mathématiques %
% ------------------- %

\begin{frame}[fragile]

	\frametitle{Modes mathématiques}
	
	\begin{enumerate}
		\item « En ligne » directement dans le texte comme
		$(𝑎 + 𝑏)2 = 𝑎2 + 2𝑎𝑏 + 𝑏2$ en plaçant l’équation entre \$  \$ \\
		\lstinline[literate={«}{{\og}}1{»}{{\fg{}}}1]|«En ligne» directement dans le texte|
		\lstinline|comme $(a + b)^2 = a^2 + 2ab + b^2$|
		
	\end{enumerate}
\end{frame}