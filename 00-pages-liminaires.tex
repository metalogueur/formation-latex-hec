% Pages liminaires
\scriptsize

% Page titre
\begin{frame}
	Benoit Hamel \\
	Technicien en documentation, soutien technique \\
	Bibliothèque HEC Montréal
	\vfill
	{
		\Huge\bfseries
		Rédaction avec \\
		\texttt{\textbackslash title\{\textrm{\LaTeX}\}}
	}
	\vfill
	Deuxième partie : notions avancées \\
	Édition HEC Montréal, revue et augmentée (version française)
\end{frame}

% Page de la licence
\begin{frame}
	\faCopyright\ 2016 Vincent Goulet pour la 
	\href{https://ctan.org/pkg/formation-latex-ul}{version originale}. La liste des sources qui ont 
	servi à l'élaboration de cette formation se trouve à la fin du présent document.
	
	\faCreativeCommons\ Cette création est mise à disposition selon le contrat 
	\href{http://creativecommons.org/licenses/by-sa/4.0/deed.fr}{%
	Attribution-Partage dans les mêmes conditions 4.0 International de Creative Commons}. 
	En vertu de ce contrat, vous êtes libre de :
	
	\begin{itemize}
		\item partager -- reproduire, distribuer et communiquer l’oeuvre;
		\item remixer -- adapter l’oeuvre;
		\item utiliser cette oeuvre à des fins commerciales.
	\end{itemize}

	Selon les conditions suivantes :
	
	\begin{itemize}
		\item Attribution -- Vous devez créditer l’oeuvre, intégrer un lien vers le contrat et indiquer si des modifications ont été effectuées à l’oeuvre. Vous devez indiquer ces informations par tous les moyens possibles, mais vous ne pouvez suggérer que l’Offrant vous soutient ou soutient la façon dont vous avez utilisé son oeuvre.
		\item Partage dans les mêmes conditions -- Dans le cas où vous modifiez, transformez ou créez à partir du matériel composant l’oeuvre
		originale, vous devez diffuser l’oeuvre modifiée dans les même conditions, c’est-à-dire avec le même contrat avec lequel l’oeuvre originale a été diffusée.
	\end{itemize}
\end{frame}

% Table des matières
\begin{frame}{Sommaire de la formation}
	\tableofcontents
\end{frame}