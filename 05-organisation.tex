% Organisation d'un document

\section{Organisation d'un document}

\subsection{Parties d'un document}

% Choix d'une classe
\begin{frame}[c]{Choix d'une classe}
	La première chose que l'on doit faire lorsqu'on débute la rédaction d'un document \LaTeX,
	c'est de choisir une classe de document.
	
	\begin{table}[c]
		\begin{tabularx}{\textwidth}{lllll}
			\arrayrulecolor{grisPrimaire!40}\hline\hline
			\textbf{Classe} & \textbf{Divisions} & \textbf{Disposition} & \textbf{Entête} &	\textbf{Pied de page} \\
			\hline
			\texttt{article}			&	parties, sections, \ldots				&	recto		&	vide			&	folio centré \\
			\texttt{report}				&	parties, chapitres, sections, \ldots	&	recto		&	vide			&	folio centré \\
			\texttt{book}				&	parties, chapitres, sections, \ldots	&	recto verso	&
			folio, titres	&	vide \\
			\texttt{hecthese}	&	chapitres, sections, sous-sections		&	recto verso	&
			vide			&	folio centré \\
			\hline\hline
		\end{tabularx}
	\end{table}
\end{frame}

% Résumé
\begin{frame}[fragile,c]{Résumé}
	\begin{itemize}
		\item Classes \textbf{article}, \textbf{report} ou \textbf{memoir}: résumé créé avec
		l'environnement \lstinline|abstract|
\begin{codesource}
	\begin{abstract}
		...
	\end{abstract}
\end{codesource}

		\item Classe \textbf{hecthese} : résumés français et anglais traités comme des chapitres
		normaux (non numérotés)
	\end{itemize}
\end{frame}

% Sections
\begin{frame}[fragile]{Sections}
	\begin{itemize}
		\item Découpage du document en sections avec les commandes suivantes :
\begin{codesource}
	\part[titre court]{titre au long}
	\chapter[titre court]{titre au long}
	\section[titre court]{titre au long}
	\subsection[titre court]{titre au long}
	
	\subsubsection[titre court]{titre au long} 	% à éviter dans un livre
	
	\paragraph[titre court]{titre au long} 		% ne jamais utiliser
	\subparagraph[titre court]{titre au long} 	% ne jamais JAMAIS utiliser
\end{codesource}

		\item Numérotation automatique
		\item Commande suivie d'un * = section non numérotée
		\item Titre court en argument optionnel
	\end{itemize}
\end{frame}

% Annexes
\begin{frame}[fragile,c]{Annexes}
	\begin{itemize}
		\item Les annexes sont des sections ou des chapitres avec une numérotation alphanumérique (A,
		A.1, \ldots).
		\item Les sections suivantes sont identifiées comme des annexes par la commande 
			\cmd{appendix}.
		\item Dans le titre, «Chapitre» est changé pour «Annexe».
	\end{itemize}
\end{frame}

% Structure logique d'un livre
\begin{frame}[fragile]{Structure logique d'un livre}
	\framesubtitle{Classes book, memoir, hecthese}

\begin{onlyenv}<1>
\begin{codesource}
	\frontmatter
\end{codesource}	
	\begin{itemize}
		\item préface, table des matières, etc.
		\item numérotation des pages en chiffres romains (i, ii, \ldots)
		\item chapitres non numérotés
	\end{itemize}
\begin{codesource}
	\mainmatter
\end{codesource}	
	\begin{itemize}
		\item le contenu à proprement parler
		\item numérotation des pages à partir de 1 en chiffres arabes
		\item chapitres numérotés
	\end{itemize}
\end{onlyenv}

\begin{onlyenv}<2>
\begin{codesource}
	\backmatter
\end{codesource}
	\begin{itemize}
		\item tout le reste (bibliographie, index, etc.)
		\item numérotation des pages se poursuit
		\item chapitres non numérotés
	\end{itemize}
\end{onlyenv}
\end{frame}

\subsection{Table des matières et renvois}

% Table des matières
\begin{frame}[fragile,c]{Table des matières}
	
	\begin{itemize}
		\item La table des matières est produite automatiquement avec \cmd{tableofcontents}.
		\item Requiert \textbf{plusieurs} compilations.
		\item Les sections non numérotées ne sont pas incluses.
		\item Avec le \emph{package} \textbf{hyperref}, \cmd{tableofcontents} produit également la table des matières du fichier .pdf.
		\pause
		\item La classe memoir fournit également \cmd{tableofcontents*} qui n’insère pas la table des matières dans la table des matières.
		\pause
		\item \cmd{listoffigures} produit la liste des figures.
		\item \cmd{listoftables} produit la liste des tableaux.
	\end{itemize}

\end{frame}

% Étiquettes et renvois automatiques
\begin{frame}[fragile]{Étiquettes et renvois automatiques}
	\framesubtitle{Parce que l'ordinateur le fera mieux que vous\ldots}
	\begin{onlyenv}<1>
		\begin{itemize}
			\item Ne \textbf{jamais} renvoyer manuellement à un numéro de section, d’équation, de tableau, etc.
			\item «Nommer» un élément avec \cmd{label}
			\item Faire référence par son nom avec \cmd{ref}
			\item Requiert 2 à 3 compilations
		\end{itemize}
	
\begin{codesource}
	\section{Définitions}
		\label{sec:definitions}
	
		Lorem ipsum dolor sit amet, consectetur adipiscing elit, 
		sed do eiusmod tempor incididunt ut labore et dolore magna aliqua. 
		Ut enim ad minim veniam, quis nostrud exercitation ullamco laboris 
		nisi ut aliquip ex ea commodo consequat.
	
	\section{Historique}
		Tel que vu à la section \ref{sec:definitions}...
\end{codesource}
	\end{onlyenv}
	\begin{onlyenv}<2>
		\begin{itemize}
			\item Le \emph{package} \textbf{hyperref} insère des hyperliens vers des renvois dans les fichiers .pdf.
			\item La commande \cmd{autoref\{\}} permet de:
				\begin{enumerate}
					\item nommer automatiquement le type de renvoi (section, équation, tableau, etc.);
					\item transformer en hyperlien le texte \textbf{et} le numéro de la référence.
\begin{codesource}
	Tel que vu à la \autoref{sec:definitions}...
\end{codesource}
				\end{enumerate}
			\item La commande \cmd{pageref\{\}} renvoie à la page de la référence.
			\item Le \emph{package} \textbf{amsmath} fournit la commande \cmd{eqref\{\}} pour
				référencer les équations.
		\end{itemize}
	\end{onlyenv}
\end{frame}