\documentclass[aspectratio=1610,compress,t,gabaritb,english,french]{hecppt}

%% Packages
\usepackage{fontawesome}
\usepackage{metalogo}
\usepackage{listings}
\usepackage{tabularx}
\usepackage{colortbl}
\usepackage{hyperref}

%% Commandes
\newcommand{\cmd}[1]{%
	\texttt{\textbackslash #1}
}

%% Environnements
\lstnewenvironment{codesource}{%
	\lstset{%
		basicstyle=\tiny,
		language=[LaTeX]TeX,
		backgroundcolor=\color{bleuPaleSecondaire!10},
		tabsize=2,
		% frame=leftline,
		% numbers=left,
		% numberstyle=\tiny,
		literate=%
		{à}{{\`a}}1
		{á}{{\'a}}1
		{é}{{\'e}}1
		{ç}{{\c c}}1
		{«}{{\og}}1
		{»}{{\fg}}1		
	}
}{}

%% Options des packages
\hypersetup{colorlinks=true,%
	urlcolor=bleuFoncePrimaire,%
	linkcolor=bleuFoncePrimaire,%
	pdfauthor=Benoit Hamel,%
	pdftitle=Rédaction avec LaTeX : Principes de base}
\frenchbsetup{og=«,fg=»}
\setlength{\parskip}{1ex}

\renewcommand{\frenchtablename}{Tableau}

%% Ajustement des espaces entre les éléments de la table des matières
\makeatletter
\patchcmd{\beamer@sectionintoc}
{\vfill}
{}
{}
{}
\makeatother 

%% Métadonnées du document

\title{Writing with \\ \texttt{\textbackslash title}\{\textrm{\LaTeX}\} }
\subtitle{Advanced Notions}
\HECauteur{Benoit Hamel}{Benoit Hamel}
\date[2018-02-28]{2018-02-28}
\subject{} % Sujet inséré dans les métadonnées du pdf
\keywords{} % Mots-clés insérés dans les métadonnées du pdf

\begin{document}

\pageTitre

% Pages liminaires
\scriptsize

% Page titre
\begin{frame}
	Benoit Hamel \\
	Library technician, technical support \\
	HEC Montréal Library
	\vfill
	{
		\Huge\bfseries
		Writing with \\
		\texttt{\textbackslash title\{\textrm{\LaTeX}\}}
	}
	\vfill
	Part Two : Advanced Notions \\
	HEC Montréal Edition, revised and extended (english version)
\end{frame}

% Page de la licence
\begin{frame}
	\faCopyright\ 2016 Vincent Goulet for the 
	\href{https://ctan.org/pkg/formation-latex-ul}{original version}. A list of sources that have been used
	for elaborating this training session can be found at the end of this document.
	
	\faCreativeCommons\ This work is provided under the  
	\href{http://creativecommons.org/licenses/by-sa/4.0/deed.en}{%
	Creative Commons Attribution-ShareAlike 4.0 International (CC BY-SA 4.0)} license. 
	According to the license, you are free to:
	
	\begin{itemize}
		\item share -- copy and redistribute the material in any medium or format;
		\item adapt -- remix, transform, and build upon the material
		for any purpose, even commercially.
	\end{itemize}

	Under the following terms:
	
	\begin{itemize}
		\item Attribution -- You must give appropriate credit, provide a link to the license, and indicate if changes were made. You may do so in any reasonable manner, but not in any way that suggests the licensor endorses you or your use.
		\item ShareAlike -- If you remix, transform, or build upon the material, you must distribute your contributions under the same license as the original.
		\item No additional restrictions -- You may not apply legal terms or technological measures that legally restrict others from doing anything the license permits.
	\end{itemize}
\end{frame}

% Table des matières
\begin{frame}{Training Session Summary}
	\tableofcontents
\end{frame}
\section{Floats}

\begin{frame}[c]{Floats}

	It was already said that the strength of \TeX\ and \LaTeX\ is typography and that it was better to let the systems do their work automatically.
	
	Tables and figures (images and graphics) are an excellent example of the systems' power.
\end{frame}

\subsection{Tables}

% Tableaux (introduction)
\begin{frame}[fragile,c]{Tables}
	\framesubtitle{Introduction}
	
	\begin{itemize}
		\item Building tables in \LaTeX\ can be tricky.
		\item There isn't one, nor two, but many ways to build tables.
		\item \LaTeX\ provides two environments: \texttt{tabular} and \texttt{tabular*}.
	\end{itemize}

	\begin{columns}
		\begin{column}{.49\textwidth}
\begin{codesource}
	\begin{tabular}{columns}
		cell1 & cell2 & cell3 \\
		cell4 & cell5 & cell6 \\
		cell7 & cell8 & cell9
	\end{tabular}
\end{codesource}
		\end{column}
		\begin{column}{.49\textwidth}
\begin{codesource}
	\begin{tabular*}{width}{columns}
		cell1 & cell2 & cell3 \\
		cell4 & cell5 & cell6 \\
		cell7 & cell8 & cell9
	\end{tabular*}
\end{codesource}
		\end{column}
	\end{columns}

	\begin{itemize}
		\item We will also take a look at a third environment, \texttt{tabularx}, provided by its eponymic package.
		\item \texttt{tabularx}'s syntax is the same as \texttt{tabular}'s.
	\end{itemize}
\end{frame}

% Tableaux (construction)
\begin{frame}[fragile]{Tables}
	\framesubtitle{Building}
	Let's take a look at the last frame's tables:
	
	\begin{columns}
		\begin{column}{.49\textwidth}
			\begin{codesource}
				\begin{tabular}{columns}
					cell1 & cell2 & cell3 \\
					cell4 & cell5 & cell6 \\
					cell7 & cell8 & cell9
				\end{tabular}
			\end{codesource}
		\end{column}
		\begin{column}{.49\textwidth}
			\begin{codesource}
				\begin{tabular*}{width}{columns}
					cell1 & cell2 & cell3 \\
					cell4 & cell5 & cell6 \\
					cell7 & cell8 & cell9
				\end{tabular*}
			\end{codesource}
		\end{column}
	\end{columns}

	\begin{onlyenv}<2>
		\begin{itemize}
			\item We define the \textbf{number of cells} and their \textbf{horizontal alignment} in the \texttt{columns} argument.
			\begin{itemize}
				\scriptsize
				\item Possible options are \texttt{l} (\emph{left}), \texttt{c} (\emph{center}),
					and \texttt{r} (\emph{right}).
				\item We define a fixed-width column with \texttt{p\{width\}}.
				\item \texttt{tabularx} also takes the \texttt{X} option, which adjusts cell width according to the table width.
				\item The \texttt{|} symbol is used to insert a vertical line between cells.
			\end{itemize}
		\end{itemize}
	\end{onlyenv}

	\begin{onlyenv}<3>
		\begin{itemize}
			\item A table's \textbf{width} depends of the environment:
			\begin{itemize}
				\scriptsize
				\item \texttt{tabular} : table width = content width;
				\item \texttt{tabular*} and \texttt{tabularx} : width determined by the \texttt{width} argument.
			\end{itemize}
		\end{itemize}
	\end{onlyenv}

	\begin{onlyenv}<4>
		\begin{itemize}
			\item Cells from a specific \textbf{row} are separated by the \texttt{\&} symbol.
			\item A row ends with \textbackslash\textbackslash, \textbf{except for the last row}.
			\item A horizontal line can be inserted between rows with \cmd{hline}.
			\item The \cmd{multicolumn\{cols\}\{pos\}\{text\}} command is used to merge cells in a row.
			\begin{itemize}
				\scriptsize
				\item \texttt{cols} : a cell's column span;
				\item \texttt{pos} : horizontal alignment (\texttt{l},\texttt{c},\texttt{r});
				\item \texttt{text} : cell content.
			\end{itemize}
		\end{itemize}
	\end{onlyenv}
	
\end{frame}

% Tableaux (exemple concret)
\begin{frame}[fragile,c]{Tables}
	\framesubtitle{Example}
\begin{codesource}
	\begin{tabularx}{\textwidth}{X|rrr|r|rrr}
		\textbf{Teams}		&	\multicolumn{7}{c}{\textbf{Statistics}} \\
		\hline\hline
		NFC North			&	W	&	L	&	T	&	PCT		&	PF	&	PA	&	Net Pts \\
		\hline
		Minnesota Vikings	&	13	&	3	&	0	&	.813	&	382	&	252	&	130 \\
		Detroit Lions		&	9	&	7	&	0	&	.563	&	410	&	376	&	34 \\
		Green Bay Packers	&	7	&	9	& 	0	&	.438	&	320	&	384	&	-64 \\
		Chicago Bears		&	5	&	11	&	0	&	.313	&	264	&	320	&	-56
	\end{tabularx}
\end{codesource}

	\begin{tabularx}{\textwidth}{X|rrr|r|rrr}
		\textbf{Teams}		&	\multicolumn{7}{c}{\textbf{Statistics}} \\
		\hline\hline
		NFC North			&	W	&	L	&	T	&	PCT		&	PF	&	PA	&	Net Pts \\
		\hline
		Minnesota Vikings	&	13	&	3	&	0	&	.813	&	382	&	252	&	130 \\
		Detroit Lions		&	9	&	7	&	0	&	.563	&	410	&	376	&	34 \\
		Green Bay Packers	&	7	&	9	& 	0	&	.438	&	320	&	384	&	-64 \\
		Chicago Bears		&	5	&	11	&	0	&	.313	&	264	&	320	&	-56
	\end{tabularx}
\end{frame}

% Tableaux flottants
\begin{frame}[fragile]{Floating tables}
	
	\begin{onlyenv}<1-2>
		\begin{itemize}
			\item The \texttt{tabular}, \texttt{tabular*} and \texttt{tabularx} insert tables in a document where they have been written in the text.
			\item \LaTeX\ can determine the best place to insert tables with the \texttt{table} environment.
\begin{codesource}
	\begin{table}[location]
		\begin{tabularx}{\textwidth}{lccc}
			...
		\end{tabularx}
		\caption{text}
	\end{table}
\end{codesource}

			\pause
			\item The optional \texttt{location} argument takes one or more of the following options:
				\begin{description}[b]
					\item[t] Table inserted on \emph{\textbf{t}op} of the page
					\item[b] Table inserted at the \emph{\textbf{b}ottom} of the page
					\item[p] Table inserted in a reserved \emph{\textbf{p}age}
					\item[h] Table inserted \emph{\textbf{h}ere}, meaning it's inserted where it was written in the text
				\end{description}
			\item Use \cmd{caption} to insert a caption below of above a table.
			\item \cmd{listoftables} generates a list of all the \texttt{table} environments inserted in the text.
		\end{itemize}
	\end{onlyenv}

	\begin{onlyenv}<3>
		\begin{codesource}
\begin{table}
	\begin{tabularx}{\textwidth}{X|rrr|r|rrr}
		Teams				&	W	&	L	&	T	&	PCT		&	PF	&	PA	&	Net Pts \\
		\hline
		Minnesota Vikings	&	13	&	3	&	0	&	.813	&	382	&	252	&	130 \\
		Detroit Lions		&	9	&	7	&	0	&	.563	&	410	&	376	&	34 \\
		Green Bay Packers	&	7	&	9	& 	0	&	.438	&	320	&	384	&	-64 \\
		Chicago Bears		&	5	&	11	&	0	&	.313	&	264	&	320	&	-56
	\end{tabularx}
	\caption{The NFL NFC North 2017 Season Statistics}
\end{table}
		\end{codesource}
		
		\begin{table}
			\begin{tabularx}{\textwidth}{X|rrr|r|rrr}
				Teams				&	W	&	L	&	T	&	PCT		&	PF	&	PA	&	Net Pts \\
				\hline
				Minnesota Vikings	&	13	&	3	&	0	&	.813	&	382	&	252	&	130 \\
				Detroit Lions		&	9	&	7	&	0	&	.563	&	410	&	376	&	34 \\
				Green Bay Packers	&	7	&	9	& 	0	&	.438	&	320	&	384	&	-64 \\
				Chicago Bears		&	5	&	11	&	0	&	.313	&	264	&	320	&	-56
			\end{tabularx}
			\caption{The NFL NFC North 2017 Season Statistics}
		\end{table}
	\end{onlyenv}
\end{frame}

\subsection{Figures}

% Insertion d'images
\begin{frame}[fragile,c]{Inserting images}
	\begin{itemize}
		\item To insert images in a \LaTeX\ document , we need three commands:
\begin{codesource}
	%% Preamble	
	\usepackage{graphicx}
	\graphicspath{{dir1}{dir2}...}
	
	%% Document body
	\includegraphics[options]{imagefile}
\end{codesource}

		\pause
		\item The \textbf{graphicx} package must be loaded in the preamble.
		\item The \cmd{graphicspath} command is used to specify in which directories the image files can be found.
		\item The \cmd{includegraphics} command inserts the image in the document.
		\item The options from \cmd{includegraphics} determine, among other things, the image's size, rotation, origin, etc. Refer to the  \href{http://mirrors.ctan.org/macros/latex/required/graphics/grfguide.pdf}{graphicx documentation} to see all available options.
	\end{itemize}
\end{frame}

% Environnement picture
\begin{frame}[fragile]{Inserting graphics}
	
	We can draw graphics in \LaTeX\ with the \texttt{picture} environment
	\footnote{\url{https://en.wikibooks.org/wiki/LaTeX/Picture\#Plotting_graphs}}.
	
	\begin{columns}
		\begin{column}{.49\textwidth}
\begin{codesource}
	\setlength{\unitlength}{1cm}
	\begin{picture}(0,0)(-3,2)
	\put(-1.5,0){\vector(1,0){3}}
	\put(2.7,-0.1){$\chi$}
	\put(0,-1.5){\vector(0,1){3}}
	\multiput(-2.5,1)(0.4,0){13}
	{\line(1,0){0.2}}
	\multiput(-2.5,-1)(0.4,0){13}
	{\line(1,0){0.2}}
	\put(0.2,1.4)
	{$\beta=v/c=\tanh\chi$}
	\qbezier(0,0)(0.8853,0.8853)
	(2,0.9640)
	\qbezier(0,0)(-0.8853,-0.8853)
	(-2,-0.9640)	
	\end{picture}
\end{codesource}	
		\end{column}
		
		\begin{column}{.49\textwidth}
			\setlength{\unitlength}{1cm}
			\begin{picture}(0,0)(-3,2)
			\put(-1.5,0){\vector(1,0){3}}
			\put(2.7,-0.1){$\chi$}
			\put(0,-1.5){\vector(0,1){3}}
			\multiput(-2.5,1)(0.4,0){13}
			{\line(1,0){0.2}}
			\multiput(-2.5,-1)(0.4,0){13}
			{\line(1,0){0.2}}
			\put(0.2,1.4)
			{$\beta=v/c=\tanh\chi$}
			\qbezier(0,0)(0.8853,0.8853)
			(2,0.9640)
			\qbezier(0,0)(-0.8853,-0.8853)
			(-2,-0.9640)	
			\end{picture}
		\end{column}
	\end{columns}	
	
	For a more advanced usage of graphics, you can use the 
	\href{https://ctan.org/pkg/pgf}{\textbf{TikZ PGF}} package.
\end{frame}

% Images et graphiques flottants
\begin{frame}[fragile,c]{Floating images and graphics}
	\begin{itemize}
		\item As for tables, it is better to let \TeX\ and \LaTeX\ determine where it is best to insert images and graphics.
		\item This can be done with the \texttt{figure} environment.
		\vspace{-2.2mm}
		\begin{columns}
			\begin{column}{.4\textwidth}
\begin{codesource}
	\begin{figure}[location]
		\includegraphics[options]{file}
		\caption{text}
	\end{figure}
\end{codesource}
			\end{column}
			\begin{column}{.4\textwidth}
\begin{codesource}
	\begin{figure}[location]
		\begin{picture}(width,height)(x,y)
			...
		\end{picture}
		\caption{text}
	\end{figure}
\end{codesource}
			\end{column}
		\end{columns}
	
		\pause
		\item The optional \texttt{location} argument takes the options values as \texttt{table}: \texttt{t},\texttt{b},\texttt{p},\texttt{h}.
		\item \cmd{caption} inserts a captions below of above an image or graphic.
		\item \cmd{listoffigures} generates a list of all the \texttt{figure} environments inserted in the text.
	\end{itemize}
\end{frame}
% Mathématiques

\section{Maths}

% Introduction
\begin{frame}[c]{Maths in \LaTeX}
	\framesubtitle{Introduction}

	\begin{itemize}
		\item Maths are \textbf{THE} reason why \TeX\ exists. \TeX\ exists because it is otherwise very difficult to render complex equations in a document.
		\item The \emph{American Mathematical Society} supports \TeX\ and \LaTeX\ from the beginning. It has built numerous packages to facilitate the writing and rendering of maths.
		\item An \textbf{essential} package that you \textbf{have to use} is
		\href{https://ctan.org/pkg/amsmath}{\texttt{amsmath}}.
		\item \LaTeX\ takes care of all typographic conventions:
		\begin{itemize}
			\scriptsize
			\item constants vs variables, equation layout and numbering;
			\item spaces between symbols and operators.
		\end{itemize}
		\item To use maths in \LaTeX, you have to put it in ``Math Mode''.
	\end{itemize}
\end{frame}

\subsection{Math Modes}

% Modes mathématiques
\begin{frame}[fragile]{Math Modes}
	There is two ways of writing equations in \LaTeX:
	
	\begin{enumerate}
		\item ``Inline'', directly in the text like $(a + b)^2 = a^2 + 2ab + b^2$ by placing the equation between \$\ and \$.
\begin{codesource}
	``Inline'', directly in the text like $(a + b)^2 = a^2 + 2ab + b^2$ by placing 
	the equations between \$\ and \$.
\end{codesource}
		\item In their own ``paragraph'', separated from the text like
			\begin{equation*}
				\int_0^\infty f(x)\, dx =
				\sum_{i = 1}^n \alpha_i e^{x_i} f(x_i)
			\end{equation*}
			by using different types of environments.
\begin{codesource}
	In their own ``paragraph'', separated from the text like
	\begin{equation*}
		\int_0^\infty f(x)\, dx =
		\sum_{i = 1}^n \alpha_i e^{x_i} f(x_i)
	\end{equation*}
	by using different types of environments.
\end{codesource}
	\end{enumerate}
\end{frame}

% Environnements mathématiques standards

\begin{frame}[fragile,c]{Math Environments}
	\framesubtitle{\LaTeX\ Standard Environments}
	There are several \LaTeX\ environments you can use to write equations:
	\begin{itemize}
		\item One-line equations:
\begin{codesource}
	\begin{displaymath} equation...	\end{displaymath}
	\begin{equation} equation... \end{equation}
	\begin{equation*} equation... \end{equation*}
\end{codesource}
		\item Multiline equations:
\begin{codesource}
	\begin{eqnarray} equation...  \end{eqnarray}
	\begin{eqnarray*} equation... \end{eqnarray*}
\end{codesource}
	\end{itemize}

	\pause
	For multiline equations, you should use the \textbf{amsmath} package's environments. They are more versatile, easier to use and they give a better rendering of equations.
\end{frame}

% Environnements mathématiques de amsmath

\begin{frame}[fragile,c]{Math Environments}
	\framesubtitle{\textbf{amsmath} package's Environments}
	\begin{description}[aaaaaaaaaaaaaaaaaaa]
		\item[\texttt{multline, multline*}] For single equations too long to fit on one line.
		\item[\texttt{align, align*}] For multiple equations aligned on a single marker (usually the = sign).
		\item[\texttt{gather, gather*}] For multiple equations, horizontally centered.
		\item[\texttt{falign, falign*}] Like \texttt{align}, but separates both sides of the equation to fit the line width.
		\item[\texttt{alignat, alignat*}] The opposite of \texttt{falign}: no space separates both sides of the equation.
		\item[\texttt{split}] For single equations too long to fit on one line; allows the alignment of the equation on a single marker.
	\end{description}
\end{frame}

% Environnements mathématiques (exemples)

\begin{frame}[fragile]{Math Environments}
	\framesubtitle{Examples}

	\begin{onlyenv}<1>
\begin{codesource}
	\begin{equation}
		a = b
	\end{equation}
\end{codesource}	
	\begin{equation}
		a = b
	\end{equation}
	
\begin{codesource}
	\begin{equation*}
		a = b
	\end{equation*}
\end{codesource}	
	\begin{equation*}
		a = b
	\end{equation*}
	
\begin{codesource}
	\begin{multline}
		a + b + c + d + e + f \\
		+ i + j + k + l + m + n
	\end{multline}
\end{codesource}
	\begin{multline}
	a + b + c + d + e + f \\
	+ o + p + q + r + s + t
	\end{multline}	
	\end{onlyenv}

	\begin{onlyenv}<2>
\begin{codesource}
	\begin{align}
		a_1 &= b_1 + c_1 \\
		a_2 &= b_2 + c_2 - d_2 + e_2
	\end{align}
\end{codesource}
	\begin{align}
		a_1 &= b_1 + c_1 \\
		a_2 &= b_2 + c_2 - d_2 + e_2
	\end{align}
	
\begin{codesource}
	\begin{gather}
		a_1 = b_1 + c_1 \\
		a_2 = b_2 + c_2 - d_2 + e_2
	\end{gather}
\end{codesource}
	\begin{gather}
		a_1 = b_1 + c_1 \\
		a_2 = b_2 + c_2 - d_2 + e_2
	\end{gather}
	\end{onlyenv}

	\begin{onlyenv}<3>
\begin{codesource}
	\begin{equation}
		\begin{split}
			a &= b + c - d \\
			&\phantom{=} + e - f \\
			&= g + h \\
			&= i
		\end{split}
	\end{equation}
\end{codesource}
		\begin{equation}
			\begin{split}
				a &= b + c - d \\
				&\phantom{=} + e - f \\
				&= g + h \\
				&= i
			\end{split}
		\end{equation}
	\end{onlyenv}
\end{frame}

\subsection{Symbols}

% Principaux éléments du mode mathématique
\begin{frame}[fragile,c]{Main elements of Math Mode}
	\begin{itemize}
		\item Basic math symbols:
		 	\texttt{+ - = < > / : ! ' | [ ] ( ) \{ \}}
		\item Exponents are written with \textasciicircum. \lstinline|x^2| becomes $x^2$.
		\item Indices are written with the underscore \_. \lstinline|a_n| becomes $a_n$.
		\item Exponents and indices can be combined: \lstinline|x_i^k| becomes $x_i^k$.
		\item Exponents and indices can be grouped with \{ and \}. \lstinline|A_{i_s, k^n}^{y_i}|
			becomes $A_{i_s, k^n}^{y_i}$.
	\end{itemize}
\end{frame}

% Fractions
\begin{frame}[fragile,c]{Fractions}
	\begin{itemize}
		\item Fractions are written with \cmd{frac\{numerator\}\{denominator\}}.
		\begin{columns}
			\begin{column}{.4\textwidth}
			\vspace{-4.5mm}
\begin{codesource}
	% Fraction size inside text
	Let $z_1 = \frac{x}{y}$ and
	$z_2 = xy$...
\end{codesource}
			\end{column}
			\begin{column}{.4\textwidth}
				Let $z_1 = \frac{x}{y}$ and
				$z_2 = xy$...
			\end{column}
		\end{columns}
	
		\pause
		
		\begin{columns}
			\begin{column}{.4\textwidth}
				\vspace{-4.5mm}
\begin{codesource}
	% Fraction size outside text
	Let
	\begin{equation*}
		z_1 = \frac{x}{y}
	\end{equation*}
	and $z_2 = xy$...
\end{codesource}
			\end{column}
			\begin{column}{.4\textwidth}
				Let
				\begin{equation*}
					z_1 = \frac{x}{y}
				\end{equation*}
				and $z_2 = xy$...
			\end{column}
		\end{columns}
	
		\pause
		
		\begin{columns}
			\begin{column}{.4\textwidth}
				\vspace{-4.5mm}
\begin{codesource}
	% Combined sizes
	Let
	\begin{equation*}
		z = \frac{\frac{x}{2} + 1}{y}.
	\end{equation*}
\end{codesource}
			\end{column}
			\begin{column}{.4\textwidth}
				Let
				\begin{equation*}
					z = \frac{\frac{x}{2} + 1}{y}.
				\end{equation*}
			\end{column}
		\end{columns}
	\end{itemize}
\end{frame}

\begin{frame}[fragile]{Roots}
	\begin{itemize}
		\item Roots are written with \cmd{sqrt[n]\{arg\}}.
		\begin{itemize}
			\scriptsize
			\item The default root (if \texttt{n} as not been defined) is the square root.
			\item The root sign is automatically fitted to \texttt{arg}.
		\end{itemize}
		\begin{columns}
			\begin{column}{.4\textwidth}
			\vspace{-4.5mm}
\begin{codesource}
	\sqrt{2}
	
		
	\sqrt{625}
	
		
	\sqrt[3]{8}
	
	
	\sqrt[n]{x + y + z}
	
	
	\sqrt{\frac{x + y}{x^2 - y^2}}
\end{codesource}
			\end{column}
			\begin{column}{.4\textwidth}
				\begin{equation*}					
					\sqrt{2}		
				\end{equation*}	
				\begin{equation*}
					\sqrt{625}
				\end{equation*}		
				\begin{equation*}				
					\sqrt[3]{8}				
				\end{equation*}
				\begin{equation*}
					\sqrt[n]{x + y + z}
				\end{equation*}
				\begin{equation*}
					\sqrt{\frac{x + y}{x^2 - y^2}}
				\end{equation*}
			\end{column}
		\end{columns}
	\end{itemize}
\end{frame}

\begin{frame}[fragile,c]{Sums and Integrals}
	\begin{itemize}
		\item Sums are written with \cmd{sum}.
		\item Integrals are written with \cmd{int}
		\item Lower and upper limits are written with indices (\_) and exponents (\textasciicircum).
			\begin{columns}
				\column{.4\textwidth}					
\begin{codesource}
	\sum_{i = 0}^n x_1
\end{codesource}
				\column{.4\textwidth}
					\begin{equation*}
						\sum_{i = 0}^n x_1
					\end{equation*}
			\end{columns}
			\begin{columns}
				\column{.4\textwidth}
\begin{codesource}
	\int_0^{10} f(x)\, dx
\end{codesource}
				\column{.4\textwidth}
					\begin{equation*}
						\int_0^{10} f(x)\, dx
					\end{equation*}
			\end{columns}
		\item The \textbf{amsmath} package also provides the \cmd{iint}
			and \cmd{iiint} to generate multiple integrals like $\iint$ and $\iiint$.
	\end{itemize}
\end{frame}

\begin{frame}[c]{Functions, operators, etc.}
	Since in Math Mode letters are considered variables, we can't manually write functions. \LaTeX\ defines commands for these functions:
	
	\begin{center}
		\begin{tabular}{lllllll}
			\cmd{arccos} & \cmd{cosh} & \cmd{det} & \cmd{inf} & \cmd{limsup} & \cmd{Pr} & \cmd{tan} \\
			\cmd{arcsin} & \cmd{cot} & \cmd{dim} & \cmd{ker} & \cmd{ln} & \cmd{sec} & \cmd{tanh} \\
			\cmd{arctan} & \cmd{coth} & \cmd{exp} & \cmd{lg} & \cmd{log} & \cmd{sin} & 	\\
			\cmd{arg} &	\cmd{csc} & \cmd{gcd} & \cmd{lim} & \cmd{max} & \cmd{sinh} &	\\
			\cmd{cos} & \cmd{deg} & \cmd{hom} & \cmd{liminf} & \cmd{min} & \cmd{sup} &	
		\end{tabular}
	\end{center}

	\pause
	
	There are also commands for \textbf{greek letters}, \textbf{text} and \textbf{spaces}, \textbf{continuation dots}, \textbf{calligraphic letters}, \textbf{binary operators} and \textbf{relations}, \textbf{arrows}, \textbf{accents} and many more!
	
	Refer to the \textbf{amsmath} package documentation and the
	\href{http://tug.ctan.org/info/symbols/comprehensive/symbols-a4.pdf}{Comprehensive \LaTeX\ Symbol List} -- 338 pages of pleasant reading! -- to learn about all the functionalities.
	
\end{frame}
% Bibliographies et citations

\section{Bibliographies et citations}

\subsection{Types de bibliographies}

% Bibliographie manuelle
\begin{frame}[fragile,c]{Bibliographie manuelle}
	\begin{itemize}
		\item On peut se «tricoter» une bibliographie à la main avec l'environnement \texttt{thebibliography}.
\begin{codesource}
	\begin{thebibliography}{libellé le plus long}
		\bibitem[libellé]{id_citation} Entrée bibliographique #1
		\bibitem[libellé]{id_citation} Entrée bibliographique #2
		[...]
	\end{thebibliography}
\end{codesource}
		
		\pause
		
		\item Chaque entrée bibliographique est rédigée avec la commande \cmd{bibitem}.
		\begin{itemize}
			\scriptsize
			\item Le \texttt{libellé} est ce qu'on retrouvera dans la référence à l'intérieur du texte. S'il n'y a pas de libellé, \LaTeX\ produira un numéro séquentiel à la place.
			\item \texttt{id\_citation} est l'élément qu'on utilise pour citer une source.
			\item L'\texttt{entrée bibliographique} contient toutes les informations bibliographiques
				de notre source.
		\end{itemize}
	
		\pause
		
		\item Le \texttt{libellé le plus long} à l'ouverture correspond à celui des libellés de tous 
		les \cmd{bibitem} qui est le plus long.
		\item La bibliographie est insérée dans le document là où l'environnement 
			\texttt{thebibliography} est inséré dans le code.
	\end{itemize}
\end{frame}

\begin{frame}[fragile,c]{Bibliographie manuelle}
	\framesubtitle{Un exemple\ldots}
\begin{codesource}
	\begin{thebibliography}{99}		
		\bibitem[Kopka and Daly, 2004]{kopkadaly:2004}
			Kopka, Helmut et Patrick W. Daly (2004).
			\newblock Guide to \LaTeX, Fourth Edition,
			\newblock Addison-Wesley,
			\newblock ISBN 978-0-321-17385-0, 597 p.
		\bibitem[Mittelbach et al., 2004]{mittelbach:2004}
			Mittelbach, Frank \emph{et al.} (2004).
			\newblock The \LaTeX\ Companion, Second Edition,
			\newblock Addison-Wesley,
			\newblock ISBN 978-0201362992, 1120p.
		\bibitem[Goossens and Mittelbach, 2007]{goossens:2007}
			Goossens, Michel et Franck Mittelbach (2007).
			\newblock The \LaTeX\ Graphics Companion, Second Edition,
			\newblock Addison-Wesley,
			\newblock ISBN 978-0321508928, 976p.
	\end{thebibliography}
\end{codesource}
\end{frame}

\begin{frame}[c]{Bibliographie automatique}
	\framesubtitle{Une introduction à BiB\TeX}
	
	\begin{itemize}
		\item BiB\TeX\ est un programme (un compilateur) auxiliaire de \LaTeX\ qui construit 
			automatiquement une bibliographie à partir d'une base de données.
		\item Il est \emph{de facto} le système standard de traitement des bibliographies.
		\item Il est stable et simple à utiliser.
		\item C'est généralement le seul format accepté par les revues scientifiques.
		\item Vous pouvez exporter nos références bibliographiques stockées dans \textbf{EndNote}
			directement en format BiB\TeX.
		\item Vous pouvez télécharger des références en format BiB\TeX depuis HECo, Google Scholar,
			ProQuest, Ebsco et de nombreuses autres banques de données de la bibliothèque.
	\end{itemize}
\end{frame}

\begin{frame}[c]{Compilation d'un document avec BiB\TeX}
	\begin{itemize}
		\item À la formation précédente, nous avons schématisé la compilation d'un document comme suit:
	\end{itemize}
	{
		\begin{minipage}[t]{0.25\linewidth}
			\centering
			{\Large\faFileTextO} \\
			code source
		\end{minipage}
		\hfill{\Large\faArrowRight}\hfill
		\begin{minipage}[t]{0.25\linewidth}
			\centering
			{\Large\faCogs} \\
			pdf\LaTeX
		\end{minipage}
		\hfill{\Large\faArrowRight}\hfill
		\begin{minipage}[t]{0.25\linewidth}
			\centering
			{\Large\faFilePdfO} \\
			document .pdf
		\end{minipage}
	}
	
	\pause
	
	\begin{itemize}
		\item Avec BiB\TeX, la séquence de compilations change:		
	\end{itemize}

	{
		\begin{minipage}[t]{0.125\linewidth}
			\centering
			{\Large\faFileTextO} \\
			code source
		\end{minipage}
		\hfill{\Large\faArrowRight}\hfill
		\begin{minipage}[t]{0.125\linewidth}
			\centering
			{\Large\faCogs} \\
			pdf\LaTeX
		\end{minipage}
		\hfill{\Large\faArrowRight}\hfill
		\begin{minipage}[t]{0.125\linewidth}
			\centering
			{\Large\faCogs} \\
			BiB\TeX
		\end{minipage}
		\hfill{\Large\faArrowRight}\hfill
		\begin{minipage}[t]{0.125\linewidth}
			\centering
			{\Large\faCogs} \\
			pdf\LaTeX
		\end{minipage}
		\hfill{\Large\faArrowRight}\hfill
		\begin{minipage}[t]{0.125\linewidth}
			\centering
			{\Large\faCogs} \\
			pdf\LaTeX
		\end{minipage}
		\hfill{\Large\faArrowRight}\hfill
		\begin{minipage}[t]{0.125\linewidth}
			\centering
			{\Large\faFilePdfO} \\
			document .pdf
		\end{minipage}
	}
\end{frame}

\subsection{Création d'une bibliographie}

\subsection{Citations}
% Bibliographie
\scriptsize

\section{Bibliography}

\begin{frame}[c]{Bibliography}
	\framesubtitle{For those who still prefer the scent of ink}
	\setbeamertemplate{bibliography item}[book]
		
	\begin{thebibliography}{99}		
		\bibitem[Kopka and Daly, 2004]{kopkadaly:2004}
			Kopka, Helmut and Patrick W. Daly (2004).
			\newblock Guide to \LaTeX, Fourth Edition,
			\newblock Addison-Wesley,
			\newblock ISBN 978-0-321-17385-0, 597 p.
		\bibitem[Mittelbach et al., 2004]{mittelbach:2004}
			Mittelbach, Frank \emph{et al.} (2004).
			\newblock The \LaTeX\ Companion, Second Edition,
			\newblock Addison-Wesley,
			\newblock ISBN 978-0201362992, 1120p.
		\bibitem[Goossens and Mittelbach, 2007]{goossens:2007}
			Goossens, Michel and Franck Mittelbach (2007).
			\newblock The \LaTeX\ Graphics Companion, Second Edition,
			\newblock Addison-Wesley,
			\newblock ISBN 978-0321508928, 976p.
	\end{thebibliography}

\end{frame}

\begin{frame}[c]{Bibliography}
	\framesubtitle{For the environmentally conscious}
	\setbeamertemplate{bibliography item}[online]
	
	\begin{onlyenv}<1>
		\begin{thebibliography}{99}
			\bibitem[Goulet, 2016]{goulet:2016}
				Goulet, Vincent (2016).
				\newblock formation-latex-ul -- Introductory \LaTeX\ course in French,
				\newblock Comprehensive \TeX\ Archive Network,
				\newblock Viewed on February 22, 2018 at \href{https://ctan.org/pkg/formation-latex-ul}{%
					https://ctan.org/pkg/formation-latex-ul}
			\bibitem[Lees-Miller, 2018]{leesmiller:2018}
				Lees-Miller, John D. (2018).
				\newblock Free \& Interactive Online Introduction to \LaTeX,
				\newblock Overleaf,
				\newblock Viewed on February 22, 2018 at \href{https://www.overleaf.com/latex/learn/free-online-introduction-to-latex-part-1}{%
					https://www.overleaf.com/latex/learn/free-online-introduction-to-latex-part-1}
			\bibitem[ShareLaTeX, 2018]{sharelatex:2018}
				Share\LaTeX\ Documentation,
				\newblock Share\LaTeX,
				\newblock Viewed on February 22, 2018 at \href{https://fr.sharelatex.com/learn/Main_Page}{%
					https://fr.sharelatex.com/learn/Main\_Page}			
		\end{thebibliography}
	\end{onlyenv}

	\begin{onlyenv}<2>
		\begin{thebibliography}{99}
			\bibitem[LaTeX Wikibook]{wikibook}
				\href{https://en.wikibooks.org/wiki/LaTeX}{\LaTeX\ WikiBook}
			\bibitem[ShareLaTeX]{sharelatex}
				\href{https://fr.sharelatex.com/learn}{Share\LaTeX\ Documentation}
			\bibitem[Stack Exchange]{stackex}
				\href{https://tex.stackexchange.com/}{\TeX\ - \LaTeX\ Stack Exchange}
			\bibitem[LaTeX Community]{latexcomm}
				\href{http://latex.org/forum/}{\LaTeX\ Community}
			\bibitem[CTAN]{ctan}
				\href{https://ctan.org/}{Comprehensive \TeX\ Archive Network}
			\bibitem[TeX FAQ]{texfaq}
				\href{http://www.tex.ac.uk/}{UK List of TEX Frequently Asked Questions}
			\bibitem{google}
				Google\ldots
		\end{thebibliography}
	\end{onlyenv}
\end{frame}

% Période de questions
\begin{frame}[c]{Questions and comments}
\LARGE
\begin{block}{Training Session Documentation}
	\ttfamily\href{http://bit.ly/enltxhec2}{http://bit.ly/enltxhec2}
\end{block}
\begin{block}{Training Session Evaluation Survey}
	\ttfamily\href{http://bit.ly/enltxsurvey2}{http://bit.ly/enltxsurvey2}
\end{block}
\begin{block}{\TeX nical Support}
	\ttfamily Benoit Hamel : <benoit.2.hamel@hec.ca>
\end{block}
\end{frame}

\end{document}
