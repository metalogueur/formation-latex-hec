\section{Floats}

\begin{frame}[c]{Floats}

	It was already said that the strength of \TeX\ and \LaTeX\ is typography and that it was better to let the systems do their work automatically.
	
	Tables and figures (images and graphics) are an excellent example of the systems' power.
\end{frame}

\subsection{Tables}

% Tableaux (introduction)
\begin{frame}[fragile,c]{Tables}
	\framesubtitle{Introduction}
	
	\begin{itemize}
		\item Building tables in \LaTeX\ can be tricky.
		\item There isn't one, nor two, but many ways to build tables.
		\item \LaTeX\ provides two environments: \texttt{tabular} and \texttt{tabular*}.
	\end{itemize}

	\begin{columns}
		\begin{column}{.49\textwidth}
\begin{codesource}
	\begin{tabular}{columns}
		cell1 & cell2 & cell3 \\
		cell4 & cell5 & cell6 \\
		cell7 & cell8 & cell9
	\end{tabular}
\end{codesource}
		\end{column}
		\begin{column}{.49\textwidth}
\begin{codesource}
	\begin{tabular*}{width}{columns}
		cell1 & cell2 & cell3 \\
		cell4 & cell5 & cell6 \\
		cell7 & cell8 & cell9
	\end{tabular*}
\end{codesource}
		\end{column}
	\end{columns}

	\begin{itemize}
		\item We will also take a look at a third environment, \texttt{tabularx}, provided by its eponymic package.
		\item \texttt{tabularx}'s syntax is the same as \texttt{tabular}'s.
	\end{itemize}
\end{frame}

% Tableaux (construction)
\begin{frame}[fragile]{Tables}
	\framesubtitle{Building}
	Let's take a look at the last frame's tables:
	
	\begin{columns}
		\begin{column}{.49\textwidth}
			\begin{codesource}
				\begin{tabular}{columns}
					cell1 & cell2 & cell3 \\
					cell4 & cell5 & cell6 \\
					cell7 & cell8 & cell9
				\end{tabular}
			\end{codesource}
		\end{column}
		\begin{column}{.49\textwidth}
			\begin{codesource}
				\begin{tabular*}{width}{columns}
					cell1 & cell2 & cell3 \\
					cell4 & cell5 & cell6 \\
					cell7 & cell8 & cell9
				\end{tabular*}
			\end{codesource}
		\end{column}
	\end{columns}

	\begin{onlyenv}<2>
		\begin{itemize}
			\item We define the \textbf{number of cells} and their \textbf{horizontal alignment} in the \texttt{columns} argument.
			\begin{itemize}
				\scriptsize
				\item Possible options are \texttt{l} (\emph{left}), \texttt{c} (\emph{center}),
					and \texttt{r} (\emph{right}).
				\item We define a fixed-width column with \texttt{p\{width\}}.
				\item \texttt{tabularx} also takes the \texttt{X} option, which adjusts cell width according to the table width.
				\item The \texttt{|} symbol is used to insert a vertical line between cells.
			\end{itemize}
		\end{itemize}
	\end{onlyenv}

	\begin{onlyenv}<3>
		\begin{itemize}
			\item A table's \textbf{width} depends of the environment:
			\begin{itemize}
				\scriptsize
				\item \texttt{tabular} : table width = content width;
				\item \texttt{tabular*} and \texttt{tabularx} : width determined by the \texttt{width} argument.
			\end{itemize}
		\end{itemize}
	\end{onlyenv}

	\begin{onlyenv}<4>
		\begin{itemize}
			\item Cells from a specific \textbf{row} are separated by the \texttt{\&} symbol.
			\item A row ends with \textbackslash\textbackslash, \textbf{except for the last row}.
			\item A horizontal line can be inserted between rows with \cmd{hline}.
			\item The \cmd{multicolumn\{cols\}\{pos\}\{text\}} command is used to merge cells in a row.
			\begin{itemize}
				\scriptsize
				\item \texttt{cols} : a cell's column span;
				\item \texttt{pos} : horizontal alignment (\texttt{l},\texttt{c},\texttt{r});
				\item \texttt{text} : cell content.
			\end{itemize}
		\end{itemize}
	\end{onlyenv}
	
\end{frame}

% Tableaux (exemple concret)
\begin{frame}[fragile,c]{Tables}
	\framesubtitle{Example}
\begin{codesource}
	\begin{tabularx}{\textwidth}{X|rrr|r|rrr}
		\textbf{Teams}		&	\multicolumn{7}{c}{\textbf{Statistics}} \\
		\hline\hline
		NFC North			&	W	&	L	&	T	&	PCT		&	PF	&	PA	&	Net Pts \\
		\hline
		Minnesota Vikings	&	13	&	3	&	0	&	.813	&	382	&	252	&	130 \\
		Detroit Lions		&	9	&	7	&	0	&	.563	&	410	&	376	&	34 \\
		Green Bay Packers	&	7	&	9	& 	0	&	.438	&	320	&	384	&	-64 \\
		Chicago Bears		&	5	&	11	&	0	&	.313	&	264	&	320	&	-56
	\end{tabularx}
\end{codesource}

	\begin{tabularx}{\textwidth}{X|rrr|r|rrr}
		\textbf{Teams}		&	\multicolumn{7}{c}{\textbf{Statistics}} \\
		\hline\hline
		NFC North			&	W	&	L	&	T	&	PCT		&	PF	&	PA	&	Net Pts \\
		\hline
		Minnesota Vikings	&	13	&	3	&	0	&	.813	&	382	&	252	&	130 \\
		Detroit Lions		&	9	&	7	&	0	&	.563	&	410	&	376	&	34 \\
		Green Bay Packers	&	7	&	9	& 	0	&	.438	&	320	&	384	&	-64 \\
		Chicago Bears		&	5	&	11	&	0	&	.313	&	264	&	320	&	-56
	\end{tabularx}
\end{frame}

% Tableaux flottants
\begin{frame}[fragile]{Floating tables}
	
	\begin{onlyenv}<1-2>
		\begin{itemize}
			\item The \texttt{tabular}, \texttt{tabular*} and \texttt{tabularx} insert tables in a document where they have been written in the text.
			\item \LaTeX\ can determine the best place to insert tables with the \texttt{table} environment.
\begin{codesource}
	\begin{table}[location]
		\begin{tabularx}{\textwidth}{lccc}
			...
		\end{tabularx}
		\caption{text}
	\end{table}
\end{codesource}

			\pause
			\item The optional \texttt{location} argument takes one or more of the following options:
				\begin{description}[b]
					\item[t] Table inserted on \emph{\textbf{t}op} of the page
					\item[b] Table inserted at the \emph{\textbf{b}ottom} of the page
					\item[p] Table inserted in a reserved \emph{\textbf{p}age}
					\item[h] Table inserted \emph{\textbf{h}ere}, meaning it's inserted where it was written in the text
				\end{description}
			\item Use \cmd{caption} to insert a caption below of above a table.
			\item \cmd{listoftables} generates a list of all the \texttt{table} environments inserted in the text.
		\end{itemize}
	\end{onlyenv}

	\begin{onlyenv}<3>
		\begin{codesource}
\begin{table}
	\begin{tabularx}{\textwidth}{X|rrr|r|rrr}
		Teams				&	W	&	L	&	T	&	PCT		&	PF	&	PA	&	Net Pts \\
		\hline
		Minnesota Vikings	&	13	&	3	&	0	&	.813	&	382	&	252	&	130 \\
		Detroit Lions		&	9	&	7	&	0	&	.563	&	410	&	376	&	34 \\
		Green Bay Packers	&	7	&	9	& 	0	&	.438	&	320	&	384	&	-64 \\
		Chicago Bears		&	5	&	11	&	0	&	.313	&	264	&	320	&	-56
	\end{tabularx}
	\caption{The NFL NFC North 2017 Season Statistics}
\end{table}
		\end{codesource}
		
		\begin{table}
			\begin{tabularx}{\textwidth}{X|rrr|r|rrr}
				Teams				&	W	&	L	&	T	&	PCT		&	PF	&	PA	&	Net Pts \\
				\hline
				Minnesota Vikings	&	13	&	3	&	0	&	.813	&	382	&	252	&	130 \\
				Detroit Lions		&	9	&	7	&	0	&	.563	&	410	&	376	&	34 \\
				Green Bay Packers	&	7	&	9	& 	0	&	.438	&	320	&	384	&	-64 \\
				Chicago Bears		&	5	&	11	&	0	&	.313	&	264	&	320	&	-56
			\end{tabularx}
			\caption{The NFL NFC North 2017 Season Statistics}
		\end{table}
	\end{onlyenv}
\end{frame}

\subsection{Figures}

% Insertion d'images
\begin{frame}[fragile,c]{Inserting images}
	\begin{itemize}
		\item To insert images in a \LaTeX\ document , we need three commands:
\begin{codesource}
	%% Preamble	
	\usepackage{graphicx}
	\graphicspath{{dir1}{dir2}...}
	
	%% Document body
	\includegraphics[options]{imagefile}
\end{codesource}

		\pause
		\item The \textbf{graphicx} package must be loaded in the preamble.
		\item The \cmd{graphicspath} command is used to specify in which directories the image files can be found.
		\item The \cmd{includegraphics} command inserts the image in the document.
		\item The options from \cmd{includegraphics} determine, among other things, the image's size, rotation, origin, etc. Refer to the  \href{http://mirrors.ctan.org/macros/latex/required/graphics/grfguide.pdf}{graphicx documentation} to see all available options.
	\end{itemize}
\end{frame}

% Environnement picture
\begin{frame}[fragile]{Inserting graphics}
	
	We can draw graphics in \LaTeX\ with the \texttt{picture} environment
	\footnote{\url{https://en.wikibooks.org/wiki/LaTeX/Picture\#Plotting_graphs}}.
	
	\begin{columns}
		\begin{column}{.49\textwidth}
\begin{codesource}
	\setlength{\unitlength}{1cm}
	\begin{picture}(0,0)(-3,2)
	\put(-1.5,0){\vector(1,0){3}}
	\put(2.7,-0.1){$\chi$}
	\put(0,-1.5){\vector(0,1){3}}
	\multiput(-2.5,1)(0.4,0){13}
	{\line(1,0){0.2}}
	\multiput(-2.5,-1)(0.4,0){13}
	{\line(1,0){0.2}}
	\put(0.2,1.4)
	{$\beta=v/c=\tanh\chi$}
	\qbezier(0,0)(0.8853,0.8853)
	(2,0.9640)
	\qbezier(0,0)(-0.8853,-0.8853)
	(-2,-0.9640)	
	\end{picture}
\end{codesource}	
		\end{column}
		
		\begin{column}{.49\textwidth}
			\setlength{\unitlength}{1cm}
			\begin{picture}(0,0)(-3,2)
			\put(-1.5,0){\vector(1,0){3}}
			\put(2.7,-0.1){$\chi$}
			\put(0,-1.5){\vector(0,1){3}}
			\multiput(-2.5,1)(0.4,0){13}
			{\line(1,0){0.2}}
			\multiput(-2.5,-1)(0.4,0){13}
			{\line(1,0){0.2}}
			\put(0.2,1.4)
			{$\beta=v/c=\tanh\chi$}
			\qbezier(0,0)(0.8853,0.8853)
			(2,0.9640)
			\qbezier(0,0)(-0.8853,-0.8853)
			(-2,-0.9640)	
			\end{picture}
		\end{column}
	\end{columns}	
	
	For a more advanced usage of graphics, you can use the 
	\href{https://ctan.org/pkg/pgf}{\textbf{TikZ PGF}} package.
\end{frame}

% Images et graphiques flottants
\begin{frame}[fragile,c]{Floating images and graphics}
	\begin{itemize}
		\item As for tables, it is better to let \TeX\ and \LaTeX\ determine where it is best to insert images and graphics.
		\item This can be done with the \texttt{figure} environment.
		\vspace{-2.2mm}
		\begin{columns}
			\begin{column}{.4\textwidth}
\begin{codesource}
	\begin{figure}[location]
		\includegraphics[options]{file}
		\caption{text}
	\end{figure}
\end{codesource}
			\end{column}
			\begin{column}{.4\textwidth}
\begin{codesource}
	\begin{figure}[location]
		\begin{picture}(width,height)(x,y)
			...
		\end{picture}
		\caption{text}
	\end{figure}
\end{codesource}
			\end{column}
		\end{columns}
	
		\pause
		\item The optional \texttt{location} argument takes the options values as \texttt{table}: \texttt{t},\texttt{b},\texttt{p},\texttt{h}.
		\item \cmd{caption} inserts a captions below of above an image or graphic.
		\item \cmd{listoffigures} generates a list of all the \texttt{figure} environments inserted in the text.
	\end{itemize}
\end{frame}