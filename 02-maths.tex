% Mathématiques

\section{Mathématiques}

% Introduction
\begin{frame}[c]{Mathématiques et \LaTeX}
	\framesubtitle{Introduction}

	\begin{itemize}
		\item Les mathématiques sont \textbf{LA} raison de l'existence de \TeX. \TeX\ existe parce qu'il est très difficile de typographier correctement des équations complexes dans un document.
		\item L'\emph{American Mathematical Society} supporte \TeX\ et \LaTeX\ depuis le début. Elle a conçu plusieurs \emph{packages} pour faciliter la transcription et la typographie des mathématiques.
		\item Un package \textbf{essentiel} que vous \textbf{devez} utiliser est 
		\href{https://ctan.org/pkg/amsmath}{\texttt{amsmath}}.
		\item \LaTeX\ gère automatiquement les conventions typographiques:
		\begin{itemize}
			\scriptsize
			\item constantes vs variables, disposition des équations, numérotation;
			\item espaces entre les symboles et autour des opérateurs.
		\end{itemize}
		\item Pour utiliser les mathématiques avec \LaTeX, il faut mettre celui-ci en «mode mathématiques».
	\end{itemize}
\end{frame}

\subsection{Modes mathématiques}



\subsection{Symboles}

\subsection{Équations sur plusieurs lignes}