% Mathématiques

\section{Mathématiques}

% Introduction
\begin{frame}[c]{Mathématiques et \LaTeX}
	\framesubtitle{Introduction}

	\begin{itemize}
		\item Les mathématiques sont \textbf{LA} raison de l'existence de \TeX. \TeX\ existe parce qu'il est très difficile de typographier correctement des équations complexes dans un document.
		\item L'\emph{American Mathematical Society} supporte \TeX\ et \LaTeX\ depuis le début. Elle a conçu plusieurs \emph{packages} pour faciliter la transcription et la typographie des mathématiques.
		\item Un package \textbf{essentiel} que vous \textbf{devez} utiliser est 
		\href{https://ctan.org/pkg/amsmath}{\texttt{amsmath}}.
		\item \LaTeX\ gère automatiquement les conventions typographiques:
		\begin{itemize}
			\scriptsize
			\item constantes vs variables, disposition des équations, numérotation;
			\item espaces entre les symboles et autour des opérateurs.
		\end{itemize}
		\item Pour utiliser les mathématiques avec \LaTeX, il faut mettre celui-ci en «mode mathématiques».
	\end{itemize}
\end{frame}

\subsection{Modes mathématiques}

% Modes mathématiques
\begin{frame}[fragile]{Modes mathématiques}
	Il existe deux méthodes d'écrire des équations avec \LaTeX:
	
	\begin{enumerate}
		\item «En ligne», directement dans le texte comme $(a + b)^2 = a^2 + 2ab + b^2$ en
		plaçant l'équation entre \$\ et \$.
\begin{codesource}
	«En ligne», directement dans le texte comme $(a + b)^2 = a^2 + 2ab + b^2$ en
	plaçant l'équation entre \$\ et \$.
\end{codesource}
		\item «Hors paragraphe» séparé du texte principal comme
			\begin{equation*}
				\int_0^\infty f(x)\, dx =
				\sum_{i = 1}^n \alpha_i e^{x_i} f(x_i)
			\end{equation*}
			en utilisant divers types d'environnements.
\begin{codesource}
	«Hors paragraphe» séparé du texte principal comme
	\begin{equation*}
		\int_0^\infty f(x)\, dx =
		\sum_{i = 1}^n \alpha_i e^{x_i} f(x_i)
	\end{equation*}
	en utilisant divers types d'environnements.
\end{codesource}
	\end{enumerate}
\end{frame}

% Environnements mathématiques standards

\begin{frame}[fragile,c]{Environnements mathématiques}
	\framesubtitle{Environnements standards \LaTeX}
	Il existe de nombreux environnements pour écrire des équations avec \LaTeX:
	\begin{itemize}
		\item Équations sur une seule ligne:
\begin{codesource}
	\begin{displaymath} équation...	\end{displaymath}
	\begin{equation} équation... \end{equation}
	\begin{equation*} équation... \end{equation*}
\end{codesource}
		\item Équations sur plusieurs lignes:
\begin{codesource}
	\begin{eqnarray} équation...  \end{eqnarray}
	\begin{eqnarray*} équation... \end{eqnarray*}
\end{codesource}
	\end{itemize}

	\pause
	On préférera cependant utiliser les environnements du \emph{package} \textbf{amsmath} pour les équations sur plusieurs lignes. Ils sont plus polyvalents, plus simples à utiliser et ils donnent un meilleur rendu.
\end{frame}

% Environnements mathématiques de amsmath

\begin{frame}[fragile,c]{Environnements mathématiques}
	\framesubtitle{Environnements du package \textbf{amsmath}}
	\begin{description}[aaaaaaaaaaaaaaaaaaa]
		\item[\texttt{multline, multline*}] Pour les équations trop longues pour entrer sur une ligne.
		\item[\texttt{align, align*}] Pour les équations multiples alignées sur un marqueur (généralement le signe =).
		\item[\texttt{gather, gather*}] Pour les équations multiples, centrées horizontalement.
		\item[\texttt{falign, falign*}] Pareil que \texttt{align}, mais sépare les deux côtés d'une équation pour remplir toute la ligne.
		\item[\texttt{alignat, alignat*}] Le contraire de \texttt{falign} : aucun espace ne sépare les deux côtés d'une équation.
		\item[\texttt{split}] Pour les équations trop longues pour entrer sur une ligne; permet l'alignement de chaque ligne avec un marqueur.
	\end{description}
\end{frame}

% Environnements mathématiques (exemples)

\begin{frame}[fragile]{Environnements mathématiques}
	\framesubtitle{Exemples}

	\begin{onlyenv}<1>
\begin{codesource}
	\begin{equation}
		a = b
	\end{equation}
\end{codesource}	
	\begin{equation}
		a = b
	\end{equation}
	
\begin{codesource}
	\begin{equation*}
		a = b
	\end{equation*}
\end{codesource}	
	\begin{equation*}
		a = b
	\end{equation*}
	
\begin{codesource}
	\begin{multline}
		a + b + c + d + e + f \\
		+ i + j + k + l + m + n
	\end{multline}
\end{codesource}
	\begin{multline}
	a + b + c + d + e + f \\
	+ o + p + q + r + s + t
	\end{multline}	
	\end{onlyenv}

	\begin{onlyenv}<2>
\begin{codesource}
	\begin{align}
		a_1 &= b_1 + c_1 \\
		a_2 &= b_2 + c_2 - d_2 + e_2
	\end{align}
\end{codesource}
	\begin{align}
		a_1 &= b_1 + c_1 \\
		a_2 &= b_2 + c_2 - d_2 + e_2
	\end{align}
	
\begin{codesource}
	\begin{gather}
		a_1 = b_1 + c_1 \\
		a_2 = b_2 + c_2 - d_2 + e_2
	\end{gather}
\end{codesource}
	\begin{gather}
		a_1 = b_1 + c_1 \\
		a_2 = b_2 + c_2 - d_2 + e_2
	\end{gather}
	\end{onlyenv}

	\begin{onlyenv}<3>
\begin{codesource}
	\begin{equation}
		\begin{split}
			a &= b + c - d \\
			&\phantom{=} + e - f \\
			&= g + h \\
			&= i
		\end{split}
	\end{equation}
\end{codesource}
		\begin{equation}
			\begin{split}
				a &= b + c - d \\
				&\phantom{=} + e - f \\
				&= g + h \\
				&= i
			\end{split}
		\end{equation}
	\end{onlyenv}
\end{frame}

\subsection{Symboles}

% Principaux éléments du mode mathématique


\subsection{Équations sur plusieurs lignes}