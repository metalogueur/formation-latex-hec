% Mathématiques

\section{Maths}

% Introduction
\begin{frame}[c]{Maths in \LaTeX}
	\framesubtitle{Introduction}

	\begin{itemize}
		\item Maths are \textbf{THE} reason why \TeX\ exists. \TeX\ exists because it is otherwise very difficult to render complex equations in a document.
		\item The \emph{American Mathematical Society} supports \TeX\ and \LaTeX\ from the beginning. It has built numerous packages to facilitate the writing and rendering of maths.
		\item An \textbf{essential} package that you \textbf{have to use} is
		\href{https://ctan.org/pkg/amsmath}{\texttt{amsmath}}.
		\item \LaTeX\ takes care of all typographic conventions:
		\begin{itemize}
			\scriptsize
			\item constants vs variables, equation layout and numbering;
			\item spaces between symbols and operators.
		\end{itemize}
		\item To use maths in \LaTeX, you have to put it in ``Math Mode''.
	\end{itemize}
\end{frame}

\subsection{Math Modes}

% Modes mathématiques
\begin{frame}[fragile]{Math Modes}
	There is two ways of writing equations in \LaTeX:
	
	\begin{enumerate}
		\item ``Inline'', directly in the text like $(a + b)^2 = a^2 + 2ab + b^2$ by placing the equation between \$\ and \$.
\begin{codesource}
	``Inline'', directly in the text like $(a + b)^2 = a^2 + 2ab + b^2$ by placing 
	the equations between \$\ and \$.
\end{codesource}
		\item In their own ``paragraph'', separated from the text like
			\begin{equation*}
				\int_0^\infty f(x)\, dx =
				\sum_{i = 1}^n \alpha_i e^{x_i} f(x_i)
			\end{equation*}
			by using different types of environments.
\begin{codesource}
	In their own ``paragraph'', separated from the text like
	\begin{equation*}
		\int_0^\infty f(x)\, dx =
		\sum_{i = 1}^n \alpha_i e^{x_i} f(x_i)
	\end{equation*}
	by using different types of environments.
\end{codesource}
	\end{enumerate}
\end{frame}

% Environnements mathématiques standards

\begin{frame}[fragile,c]{Math Environments}
	\framesubtitle{\LaTeX\ Standard Environments}
	There are several \LaTeX\ environments you can use to write equations:
	\begin{itemize}
		\item One-line equations:
\begin{codesource}
	\begin{displaymath} equation...	\end{displaymath}
	\begin{equation} equation... \end{equation}
	\begin{equation*} equation... \end{equation*}
\end{codesource}
		\item Multiline equations:
\begin{codesource}
	\begin{eqnarray} equation...  \end{eqnarray}
	\begin{eqnarray*} equation... \end{eqnarray*}
\end{codesource}
	\end{itemize}

	\pause
	For multiline equations, you should use the \textbf{amsmath} package's environments. They are more versatile, easier to use and they give a better rendering of equations.
\end{frame}

% Environnements mathématiques de amsmath

\begin{frame}[fragile,c]{Math Environments}
	\framesubtitle{\textbf{amsmath} package's Environments}
	\begin{description}[aaaaaaaaaaaaaaaaaaa]
		\item[\texttt{multline, multline*}] For single equations too long to fit on one line.
		\item[\texttt{align, align*}] For multiple equations aligned on a single marker (usually the = sign).
		\item[\texttt{gather, gather*}] For multiple equations, horizontally centered.
		\item[\texttt{falign, falign*}] Like \texttt{align}, but separates both sides of the equation to fit the line width.
		\item[\texttt{alignat, alignat*}] The opposite of \texttt{falign}: no space separates both sides of the equation.
		\item[\texttt{split}] For single equations too long to fit on one line; allows the alignment of the equation on a single marker.
	\end{description}
\end{frame}

% Environnements mathématiques (exemples)

\begin{frame}[fragile]{Math Environments}
	\framesubtitle{Examples}

	\begin{onlyenv}<1>
\begin{codesource}
	\begin{equation}
		a = b
	\end{equation}
\end{codesource}	
	\begin{equation}
		a = b
	\end{equation}
	
\begin{codesource}
	\begin{equation*}
		a = b
	\end{equation*}
\end{codesource}	
	\begin{equation*}
		a = b
	\end{equation*}
	
\begin{codesource}
	\begin{multline}
		a + b + c + d + e + f \\
		+ i + j + k + l + m + n
	\end{multline}
\end{codesource}
	\begin{multline}
	a + b + c + d + e + f \\
	+ o + p + q + r + s + t
	\end{multline}	
	\end{onlyenv}

	\begin{onlyenv}<2>
\begin{codesource}
	\begin{align}
		a_1 &= b_1 + c_1 \\
		a_2 &= b_2 + c_2 - d_2 + e_2
	\end{align}
\end{codesource}
	\begin{align}
		a_1 &= b_1 + c_1 \\
		a_2 &= b_2 + c_2 - d_2 + e_2
	\end{align}
	
\begin{codesource}
	\begin{gather}
		a_1 = b_1 + c_1 \\
		a_2 = b_2 + c_2 - d_2 + e_2
	\end{gather}
\end{codesource}
	\begin{gather}
		a_1 = b_1 + c_1 \\
		a_2 = b_2 + c_2 - d_2 + e_2
	\end{gather}
	\end{onlyenv}

	\begin{onlyenv}<3>
\begin{codesource}
	\begin{equation}
		\begin{split}
			a &= b + c - d \\
			&\phantom{=} + e - f \\
			&= g + h \\
			&= i
		\end{split}
	\end{equation}
\end{codesource}
		\begin{equation}
			\begin{split}
				a &= b + c - d \\
				&\phantom{=} + e - f \\
				&= g + h \\
				&= i
			\end{split}
		\end{equation}
	\end{onlyenv}
\end{frame}

\subsection{Symbols}

% Principaux éléments du mode mathématique
\begin{frame}[fragile,c]{Main elements of Math Mode}
	\begin{itemize}
		\item Basic math symbols:
		 	\texttt{+ - = < > / : ! ' | [ ] ( ) \{ \}}
		\item Exponents are written with \textasciicircum. \lstinline|x^2| becomes $x^2$.
		\item Indices are written with the underscore \_. \lstinline|a_n| becomes $a_n$.
		\item Exponents and indices can be combined: \lstinline|x_i^k| becomes $x_i^k$.
		\item Exponents and indices can be grouped with \{ and \}. \lstinline|A_{i_s, k^n}^{y_i}|
			becomes $A_{i_s, k^n}^{y_i}$.
	\end{itemize}
\end{frame}

% Fractions
\begin{frame}[fragile,c]{Fractions}
	\begin{itemize}
		\item Fractions are written with \cmd{frac\{numerator\}\{denominator\}}.
		\begin{columns}
			\begin{column}{.4\textwidth}
			\vspace{-4.5mm}
\begin{codesource}
	% Fraction size inside text
	Let $z_1 = \frac{x}{y}$ and
	$z_2 = xy$...
\end{codesource}
			\end{column}
			\begin{column}{.4\textwidth}
				Let $z_1 = \frac{x}{y}$ and
				$z_2 = xy$...
			\end{column}
		\end{columns}
	
		\pause
		
		\begin{columns}
			\begin{column}{.4\textwidth}
				\vspace{-4.5mm}
\begin{codesource}
	% Fraction size outside text
	Let
	\begin{equation*}
		z_1 = \frac{x}{y}
	\end{equation*}
	and $z_2 = xy$...
\end{codesource}
			\end{column}
			\begin{column}{.4\textwidth}
				Let
				\begin{equation*}
					z_1 = \frac{x}{y}
				\end{equation*}
				and $z_2 = xy$...
			\end{column}
		\end{columns}
	
		\pause
		
		\begin{columns}
			\begin{column}{.4\textwidth}
				\vspace{-4.5mm}
\begin{codesource}
	% Combined sizes
	Let
	\begin{equation*}
		z = \frac{\frac{x}{2} + 1}{y}.
	\end{equation*}
\end{codesource}
			\end{column}
			\begin{column}{.4\textwidth}
				Let
				\begin{equation*}
					z = \frac{\frac{x}{2} + 1}{y}.
				\end{equation*}
			\end{column}
		\end{columns}
	\end{itemize}
\end{frame}

\begin{frame}[fragile]{Roots}
	\begin{itemize}
		\item Roots are written with \cmd{sqrt[n]\{arg\}}.
		\begin{itemize}
			\scriptsize
			\item The default root (if \texttt{n} as not been defined) is the square root.
			\item The root sign is automatically fitted to \texttt{arg}.
		\end{itemize}
		\begin{columns}
			\begin{column}{.4\textwidth}
			\vspace{-4.5mm}
\begin{codesource}
	\sqrt{2}
	
		
	\sqrt{625}
	
		
	\sqrt[3]{8}
	
	
	\sqrt[n]{x + y + z}
	
	
	\sqrt{\frac{x + y}{x^2 - y^2}}
\end{codesource}
			\end{column}
			\begin{column}{.4\textwidth}
				\begin{equation*}					
					\sqrt{2}		
				\end{equation*}	
				\begin{equation*}
					\sqrt{625}
				\end{equation*}		
				\begin{equation*}				
					\sqrt[3]{8}				
				\end{equation*}
				\begin{equation*}
					\sqrt[n]{x + y + z}
				\end{equation*}
				\begin{equation*}
					\sqrt{\frac{x + y}{x^2 - y^2}}
				\end{equation*}
			\end{column}
		\end{columns}
	\end{itemize}
\end{frame}

\begin{frame}[fragile,c]{Sums and Integrals}
	\begin{itemize}
		\item Sums are written with \cmd{sum}.
		\item Integrals are written with \cmd{int}
		\item Lower and upper limits are written with indices (\_) and exponents (\textasciicircum).
			\begin{columns}
				\column{.4\textwidth}					
\begin{codesource}
	\sum_{i = 0}^n x_1
\end{codesource}
				\column{.4\textwidth}
					\begin{equation*}
						\sum_{i = 0}^n x_1
					\end{equation*}
			\end{columns}
			\begin{columns}
				\column{.4\textwidth}
\begin{codesource}
	\int_0^{10} f(x)\, dx
\end{codesource}
				\column{.4\textwidth}
					\begin{equation*}
						\int_0^{10} f(x)\, dx
					\end{equation*}
			\end{columns}
		\item The \textbf{amsmath} package also provides the \cmd{iint}
			and \cmd{iiint} to generate multiple integrals like $\iint$ and $\iiint$.
	\end{itemize}
\end{frame}

\begin{frame}[c]{Functions, operators, etc.}
	Since in Math Mode letters are considered variables, we can't manually write functions. \LaTeX\ defines commands for these functions:
	
	\begin{center}
		\begin{tabular}{lllllll}
			\cmd{arccos} & \cmd{cosh} & \cmd{det} & \cmd{inf} & \cmd{limsup} & \cmd{Pr} & \cmd{tan} \\
			\cmd{arcsin} & \cmd{cot} & \cmd{dim} & \cmd{ker} & \cmd{ln} & \cmd{sec} & \cmd{tanh} \\
			\cmd{arctan} & \cmd{coth} & \cmd{exp} & \cmd{lg} & \cmd{log} & \cmd{sin} & 	\\
			\cmd{arg} &	\cmd{csc} & \cmd{gcd} & \cmd{lim} & \cmd{max} & \cmd{sinh} &	\\
			\cmd{cos} & \cmd{deg} & \cmd{hom} & \cmd{liminf} & \cmd{min} & \cmd{sup} &	
		\end{tabular}
	\end{center}

	\pause
	
	There are also commands for \textbf{greek letters}, \textbf{text} and \textbf{spaces}, \textbf{continuation dots}, \textbf{calligraphic letters}, \textbf{binary operators} and \textbf{relations}, \textbf{arrows}, \textbf{accents} and many more!
	
	Refer to the \textbf{amsmath} package documentation and the
	\href{http://tug.ctan.org/info/symbols/comprehensive/symbols-a4.pdf}{Comprehensive \LaTeX\ Symbol List} -- 338 pages of pleasant reading! -- to learn about all the functionalities.
	
\end{frame}