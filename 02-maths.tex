% Mathématiques

\section{Mathématiques}

% Introduction
\begin{frame}[c]{Mathématiques et \LaTeX}
	\framesubtitle{Introduction}

	\begin{itemize}
		\item Les mathématiques sont \textbf{LA} raison de l'existence de \TeX. \TeX\ existe parce qu'il est très difficile de typographier correctement des équations complexes dans un document.
		\item L'\emph{American Mathematical Society} supporte \TeX\ et \LaTeX\ depuis le début. Elle a conçu plusieurs \emph{packages} pour faciliter la transcription et la typographie des mathématiques.
		\item Un package \textbf{essentiel} que vous \textbf{devez} utiliser est 
		\href{https://ctan.org/pkg/amsmath}{\texttt{amsmath}}.
		\item \LaTeX\ gère automatiquement les conventions typographiques:
		\begin{itemize}
			\scriptsize
			\item constantes vs variables, disposition des équations, numérotation;
			\item espaces entre les symboles et autour des opérateurs.
		\end{itemize}
		\item Pour utiliser les mathématiques avec \LaTeX, il faut mettre celui-ci en «mode mathématiques».
	\end{itemize}
\end{frame}

\subsection{Modes mathématiques}

% Modes mathématiques
\begin{frame}[fragile]{Modes mathématiques}
	Il existe deux méthodes d'écrire des équations avec \LaTeX:
	
	\begin{enumerate}
		\item «En ligne», directement dans le texte comme $(a + b)^2 = a^2 + 2ab + b^2$ en
		plaçant l'équation entre \$\ et \$.
\begin{codesource}
	«En ligne», directement dans le texte comme $(a + b)^2 = a^2 + 2ab + b^2$ en
	plaçant l'équation entre \$\ et \$.
\end{codesource}
		\item «Hors paragraphe» séparé du texte principal comme
			\begin{equation*}
				\int_0^\infty f(x)\, dx =
				\sum_{i = 1}^n \alpha_i e^{x_i} f(x_i)
			\end{equation*}
			en utilisant divers types d'environnements.
\begin{codesource}
	«Hors paragraphe» séparé du texte principal comme
	\begin{equation*}
		\int_0^\infty f(x)\, dx =
		\sum_{i = 1}^n \alpha_i e^{x_i} f(x_i)
	\end{equation*}
	en utilisant divers types d'environnements.
\end{codesource}
	\end{enumerate}
\end{frame}

% Environnements mathématiques standards

\begin{frame}[fragile,c]{Environnements mathématiques}
	\framesubtitle{Environnements standards \LaTeX}
	Il existe de nombreux environnements pour écrire des équations avec \LaTeX:
	\begin{itemize}
		\item Équations sur une seule ligne:
\begin{codesource}
	\begin{displaymath} équation...	\end{displaymath}
	\begin{equation} équation... \end{equation}
	\begin{equation*} équation... \end{equation*}
\end{codesource}
		\item Équations sur plusieurs lignes:
\begin{codesource}
	\begin{eqnarray} équation...  \end{eqnarray}
	\begin{eqnarray*} équation... \end{eqnarray*}
\end{codesource}
	\end{itemize}

	\pause
	On préférera cependant utiliser les environnements du \emph{package} \textbf{amsmath} pour les équations sur plusieurs lignes. Ils sont plus polyvalents, plus simples à utiliser et ils donnent un meilleur rendu.
\end{frame}

% Environnements mathématiques de amsmath

\begin{frame}[fragile,c]{Environnements mathématiques}
	\framesubtitle{Environnements du package \textbf{amsmath}}
	\begin{description}[aaaaaaaaaaaaaaaaaaa]
		\item[\texttt{multline, multline*}] Pour les équations trop longues pour entrer sur une ligne.
		\item[\texttt{align, align*}] Pour les équations multiples alignées sur un marqueur (généralement le signe =).
		\item[\texttt{gather, gather*}] Pour les équations multiples, centrées horizontalement.
		\item[\texttt{falign, falign*}] Pareil que \texttt{align}, mais sépare les deux côtés d'une équation pour remplir toute la ligne.
		\item[\texttt{alignat, alignat*}] Le contraire de \texttt{falign} : aucun espace ne sépare les deux côtés d'une équation.
		\item[\texttt{split}] Pour les équations trop longues pour entrer sur une ligne; permet l'alignement de chaque ligne avec un marqueur.
	\end{description}
\end{frame}

% Environnements mathématiques (exemples)

\begin{frame}[fragile]{Environnements mathématiques}
	\framesubtitle{Exemples}

	\begin{onlyenv}<1>
\begin{codesource}
	\begin{equation}
		a = b
	\end{equation}
\end{codesource}	
	\begin{equation}
		a = b
	\end{equation}
	
\begin{codesource}
	\begin{equation*}
		a = b
	\end{equation*}
\end{codesource}	
	\begin{equation*}
		a = b
	\end{equation*}
	
\begin{codesource}
	\begin{multline}
		a + b + c + d + e + f \\
		+ i + j + k + l + m + n
	\end{multline}
\end{codesource}
	\begin{multline}
	a + b + c + d + e + f \\
	+ o + p + q + r + s + t
	\end{multline}	
	\end{onlyenv}

	\begin{onlyenv}<2>
\begin{codesource}
	\begin{align}
		a_1 &= b_1 + c_1 \\
		a_2 &= b_2 + c_2 - d_2 + e_2
	\end{align}
\end{codesource}
	\begin{align}
		a_1 &= b_1 + c_1 \\
		a_2 &= b_2 + c_2 - d_2 + e_2
	\end{align}
	
\begin{codesource}
	\begin{gather}
		a_1 = b_1 + c_1 \\
		a_2 = b_2 + c_2 - d_2 + e_2
	\end{gather}
\end{codesource}
	\begin{gather}
		a_1 = b_1 + c_1 \\
		a_2 = b_2 + c_2 - d_2 + e_2
	\end{gather}
	\end{onlyenv}

	\begin{onlyenv}<3>
\begin{codesource}
	\begin{equation}
		\begin{split}
			a &= b + c - d \\
			&\phantom{=} + e - f \\
			&= g + h \\
			&= i
		\end{split}
	\end{equation}
\end{codesource}
		\begin{equation}
			\begin{split}
				a &= b + c - d \\
				&\phantom{=} + e - f \\
				&= g + h \\
				&= i
			\end{split}
		\end{equation}
	\end{onlyenv}
\end{frame}

\subsection{Symboles}

% Principaux éléments du mode mathématique
\begin{frame}[fragile,c]{Principaux éléments du mode mathématique}
	\begin{itemize}
		\item Symboles mathématiques «de base»:
		 	\texttt{+ - = < > / : ! ' | [ ] ( ) \{ \}}
		\item On écrit les exposants avec la commande \textasciicircum. \lstinline|x^2| devient $x^2$.
		\item On écrit les indices avec la commande \_. \lstinline|a_n| devient $a_n$.
		\item On peut combiner exposants et indices : \lstinline|x_i^k| devient $x_i^k$.
		\item On regroupe les exposants et les indices avec \{ et \}. \lstinline|A_{i_s, k^n}^{y_i}|
			devient $A_{i_s, k^n}^{y_i}$.
	\end{itemize}
\end{frame}

% Fractions
\begin{frame}[fragile,c]{Fractions}
	\begin{itemize}
		\item On rédige des fractions avec la commande \cmd{frac\{numérateur\}\{dénominateur\}}.
		\begin{columns}
			\begin{column}{.4\textwidth}
			\vspace{-4.5mm}
\begin{codesource}
	% Taille au fil du texte
	On a $z_1 = \frac{x}{y}$ et
	$z_2 = xy$.
\end{codesource}
			\end{column}
			\begin{column}{.4\textwidth}
				On a $z_1 = \frac{x}{y}$ et
				$z_2 = xy$.
			\end{column}
		\end{columns}
	
		\pause
		
		\begin{columns}
			\begin{column}{.4\textwidth}
				\vspace{-4.5mm}
\begin{codesource}
	% taille hors paragraphe
	On a
	\begin{equation*}
		z_1 = \frac{x}{y}
	\end{equation*}
	et $z_2 = xy$.
\end{codesource}
			\end{column}
			\begin{column}{.4\textwidth}
				On a
				\begin{equation*}
					z_1 = \frac{x}{y}
				\end{equation*}
				et $z_2 = xy$.
			\end{column}
		\end{columns}
	
		\pause
		
		\begin{columns}
			\begin{column}{.4\textwidth}
				\vspace{-4.5mm}
\begin{codesource}
	% Deux tailles combinées
	Soit
	\begin{equation*}
		z = \frac{\frac{x}{2} + 1}{y}.
	\end{equation*}
\end{codesource}
			\end{column}
			\begin{column}{.4\textwidth}
				Soit
				\begin{equation*}
					z = \frac{\frac{x}{2} + 1}{y}.
				\end{equation*}
			\end{column}
		\end{columns}
	\end{itemize}
\end{frame}

\begin{frame}[fragile]{Racines}
	\begin{itemize}
		\item On rédige des racines avec la commande \cmd{sqrt[n]\{radicande\}}.
		\begin{itemize}
			\scriptsize
			\item Le radical par défaut (si on ne spécifie par l'argument \texttt{n}) est la racine carrée.
			\item Le radical s'adapte toujours au radicande.
		\end{itemize}
		\begin{columns}
			\begin{column}{.4\textwidth}
			\vspace{-4.5mm}
\begin{codesource}
	\sqrt{2}
	
		
	\sqrt{625}
	
		
	\sqrt[3]{8}
	
	
	\sqrt[n]{x + y + z}
	
	
	\sqrt{\frac{x + y}{x^2 - y^2}}
\end{codesource}
			\end{column}
			\begin{column}{.4\textwidth}
				\begin{equation*}					
					\sqrt{2}		
				\end{equation*}	
				\begin{equation*}
					\sqrt{625}
				\end{equation*}		
				\begin{equation*}				
					\sqrt[3]{8}				
				\end{equation*}
				\begin{equation*}
					\sqrt[n]{x + y + z}
				\end{equation*}
				\begin{equation*}
					\sqrt{\frac{x + y}{x^2 - y^2}}
				\end{equation*}
			\end{column}
		\end{columns}
	\end{itemize}
\end{frame}

\begin{frame}[fragile,c]{Sommes et intégrales}
	\begin{itemize}
		\item On écrit des sommes avec la commande \cmd{sum}.
		\item On écrit des intégrales avec la commande \cmd{int}
		\item On saisit les limites inférieures et supérieures avec des indices (\_) et 
			des exposants (\textasciicircum).
			\begin{columns}
				\column{.4\textwidth}					
\begin{codesource}
	\sum_{i = 0}^n x_1
\end{codesource}
				\column{.4\textwidth}
					\begin{equation*}
						\sum_{i = 0}^n x_1
					\end{equation*}
			\end{columns}
			\begin{columns}
				\column{.4\textwidth}
\begin{codesource}
	\int_0^{10} f(x)\, dx
\end{codesource}
				\column{.4\textwidth}
					\begin{equation*}
						\int_0^{10} f(x)\, dx
					\end{equation*}
			\end{columns}
		\item Le \emph{package} \textbf{amsmath} fournit également les commandes \cmd{iint}
			et \cmd{iiint} pour afficher des intégrales multiples comme $\iint$ et $\iiint$.
	\end{itemize}
\end{frame}

\begin{frame}[c]{Fonctions, opérateurs, etc.}
	Puisque le mode mathématique considère les lettres comme des variables, on ne peut pas écrire 
	les fonctions textuellement. \LaTeX\ définit donc des commandes pour ces fonctions :
	
	\begin{center}
		\begin{tabular}{lllllll}
			\cmd{arccos} & \cmd{cosh} & \cmd{det} & \cmd{inf} & \cmd{limsup} & \cmd{Pr} & \cmd{tan} \\
			\cmd{arcsin} & \cmd{cot} & \cmd{dim} & \cmd{ker} & \cmd{ln} & \cmd{sec} & \cmd{tanh} \\
			\cmd{arctan} & \cmd{coth} & \cmd{exp} & \cmd{lg} & \cmd{log} & \cmd{sin} & 	\\
			\cmd{arg} &	\cmd{csc} & \cmd{gcd} & \cmd{lim} & \cmd{max} & \cmd{sinh} &	\\
			\cmd{cos} & \cmd{deg} & \cmd{hom} & \cmd{liminf} & \cmd{min} & \cmd{sup} &	
		\end{tabular}
	\end{center}

	\pause
	
	Il existe aussi des commandes pour les \textbf{lettres grecques}, le \textbf{texte} et les \textbf{espaces}, les \textbf{points de suspension}, les \textbf{lettres modifiées}, les
	\textbf{opérateurs binaires} et les \textbf{relations}, les \textbf{flèches}, les \textbf{accents} et bien plus encore!
	
	Consultez la documentation du \emph{package} \textbf{amsmath} ainsi que la 
	\href{http://tug.ctan.org/info/symbols/comprehensive/symbols-a4.pdf}{Comprehensive \LaTeX\ Symbol List} -- 338 pages de bonheur! -- pour connaître l'étendue de toutes les fonctionnalités.
	
\end{frame}