\documentclass[aspectratio=1610,compress,t,gabaritb,french,english]{hecppt}

%% Packages
\usepackage{fontawesome}
\usepackage{metalogo}
\usepackage{listings}
\usepackage{tabularx}
\usepackage{colortbl}
\usepackage{makecell}
\usepackage{hyperref}

%% Commandes
\newcommand{\cmd}[1]{%
	\texttt{\textbackslash #1}
}

\newcommand{\lien}[2]{%
	\href{#1}{#2 \faExternalLink}
}

%% Environnements
\lstnewenvironment{codesource}{%
	\lstset{%
		basicstyle=\tiny,
		language=[LaTeX]TeX,
		backgroundcolor=\color{bleuPaleSecondaire!10},
		tabsize=2,
		% frame=leftline,
		% numbers=left,
		% numberstyle=\tiny,
		literate=%
		{à}{{\`a}}1
		{é}{{\'e}}1
		{ç}{{\c c}}1
		{«}{{\og}}1
		{»}{{\fg}}1
	}
}{}

%% Options des packages
\hypersetup{colorlinks=true,%
	urlcolor=bleuFoncePrimaire,%
	linkcolor=bleuFoncePrimaire,%
	pdfauthor=Benoit Hamel,%
	pdftitle=Rédaction avec LaTeX : Principes de base}
\frenchbsetup{og=«,fg=»}
\setlength{\parskip}{1ex}
\renewcommand{\cellalign}{tl}

%% Métadonnées du document

\title{Writing with \\ \texttt{\textbackslash title}\{\textrm{\LaTeX}\} }
\subtitle{The Basics}
\HECauteur{Benoit Hamel}{Benoit Hamel}
\date[2018-02-28]{2018-02-28}
\subject{} % Sujet inséré dans les métadonnées du pdf
\keywords{} % Mots-clés insérés dans les métadonnées du pdf

\begin{document}

\pageTitre

% Pages liminaires
\scriptsize

% Page titre
\begin{frame}
	Benoit Hamel \\
	Library technician, technical support \\
	HEC Montréal Library
	\vfill
	{
		\Huge\bfseries
		Writing with \\
		\texttt{\textbackslash title\{\textrm{\LaTeX}\}}
	}
	\vfill
	Part One : The Basics \\
	HEC Montréal Edition, revised and extended (english version)
\end{frame}

% Page de la licence
\begin{frame}
	\faCopyright\ 2016 Vincent Goulet for the 
	\lien{https://ctan.org/pkg/formation-latex-ul}{original version}. A list of sources that have been used
	for elaborating this training session can be found at the end of this document.
	
	\faCreativeCommons\ This work is provided under the  
	\lien{http://creativecommons.org/licenses/by-sa/4.0/deed.en}{%
	Creative Commons Attribution-ShareAlike 4.0 International (CC BY-SA 4.0)} license. 
	According to the license, you are free to:
	
	\begin{itemize}
		\item share -- copy and redistribute the material in any medium or format;
		\item adapt -- remix, transform, and build upon the material
		for any purpose, even commercially.
	\end{itemize}

	Under the following terms:
	
	\begin{itemize}
		\item Attribution -- You must give appropriate credit, provide a link to the license, and indicate if changes were made. You may do so in any reasonable manner, but not in any way that suggests the licensor endorses you or your use.
		\item ShareAlike -- If you remix, transform, or build upon the material, you must distribute your contributions under the same license as the original.
		\item No additional restrictions -- You may not apply legal terms or technological measures that legally restrict others from doing anything the license permits.
	\end{itemize}
\end{frame}

% Table des matières
\begin{frame}{Training Session Summary}
	\begin{columns}[onlytextwidth]
		\begin{column}{.49\textwidth}
			\tableofcontents[sections={1-3}]
		\end{column}
		\begin{column}{.49\textwidth}
			\tableofcontents[sections={4-8}]
		\end{column}
	\end{columns}
\end{frame}
% Présentation de TeX et LaTeX
\small

\section{Présentation de \TeX\ et \LaTeX}

\subsection{Qu'est-ce que \TeX\ et \LaTeX?}

\begin{frame}[c,label=fr:commencement]{Au commencement (1978), il y eut \TeX\ldots}
	\includegraphics[width=\textwidth,keepaspectratio=true]{knuth-tex-commencement.jpg}
\end{frame}

\begin{frame}[c]{Qu'est-ce que \TeX?}

	\begin{itemize}
		\item Un système de mise en page (\emph{typesetting}) et de préparation de documents;
		\item «Le système le plus puissant pour produire des ouvrages scientifiques et
		techniques d'une grande qualité typographique»\footnote{Kopka \& Daly, p. 6};
		\item Un système mature, stable et complet, considéré comme exempt de bogues;
		\item Un ensemble de commandes très primitives parfaites pour la typographie et
		des fonctions de programmation;
		\item «\emph{typesetter-level program}».
	\end{itemize}

\end{frame}

\begin{frame}[c,label=fr:sixiemejour]{Au sixième jour (1983), il y eut \LaTeX\ldots}
	\includegraphics[width=\textwidth,keepaspectratio=true]{creation-of-latex.jpg}
\end{frame}

\begin{frame}{Qu'est-ce que \LaTeX?}
	\begin{itemize}
		\item Un ensemble de macro-commandes pour faciliter l'utilisation de \TeX.
		\item Ne requiert aucune connaissance préalable de la typographie en général et de \TeX\ en particulier.
		\item Langage de balisage (\emph{Markup Language}) typographique et logique pour indiquer la mise en forme du texte (pensez au HTML).
		\item Langage multiplateforme, identique d'un système d'exploitation à l'autre, et extensible par l'ajout de \emph{packages}.
		\item «\emph{author-level program}»
	\end{itemize}
\end{frame}

\subsection{Processus de création d'un document \LaTeX}

% Rédiger avec une nouvelle perspective
\begin{frame}[c]{Rédiger avec une nouvelle perspective}
	
	\begin{itemize}
		\item Vous rédigez votre document en texte brut et utilisez des commandes pour décrire
			\textbf{ce que votre texte représente} et \textbf{non pas ce à quoi il doit ressembler}.
		\item Vous vous concentrez sur votre \textbf{contenu}.
		\item Vous laissez \LaTeX\ faire son travail, c'est-à-dire s'occuper du \textbf{contenant}.
	\end{itemize}
	
\end{frame}

% Processus de création d'un document LaTeX
\begin{frame}[c]{Processus de création d'un document \LaTeX}
	\Huge
	\begin{minipage}[t]{0.25\linewidth}
		\centering
		\faFileTextO
	\end{minipage}
	\hfill\faArrowRight\hfill
	\begin{minipage}[t]{0.25\linewidth}
		\centering
		\faCogs
	\end{minipage}
	\hfill\faArrowRight\hfill
	\begin{minipage}[t]{0.25\linewidth}
		\centering
		\faFilePdfO
	\end{minipage}

	\begin{picture}(0,0)
		\footnotesize\thicklines\color{bleuFonceSecondaire}
		\onslide<2>\put(0,-10){\dashbox{1}(35,40)[b]{\parbox{.2\textwidth}{\centering\textbf{rédaction du texte et balisage avec éditeur de texte\smallskip}}}}
		\onslide<3>\put(54,-10){\dashbox{1}(35,40)[b]{\parbox{.2\textwidth}{\centering\textbf{compilation avec un moteur \TeX\ à partir de la ligne de commande\smallskip}}}}
		\onslide<4>\put(108,-10){\dashbox{1}(35,40)[b]{\parbox{.2\textwidth}{\centering\textbf{visualisation avec une visionneuse externe\smallskip}}}}
	\end{picture}
\end{frame}
% Création d'un document LaTeX

\section{Création d'un document \LaTeX}

\subsection{Structure d'un document}

% Un document LaTeX dans sa plus simple expression
\begin{frame}[c,fragile]{Un document \LaTeX\ dans sa plus simple expression}

	Dans un éditeur de texte, ouvrez un nouveau fichier et saisissez le code suivant :
	
\begin{codesource}	
	\documentclass{article}
	
	\begin{document}		
		Ceci est mon premier document LaTeX et j'en suis fier.
	\end{document}

\end{codesource}

	Sauvegardez votre fichier avec l'extension \texttt{.tex} et compilez-le. Observez le résultat.
	
\end{frame}

% Les parties d'un document : la déclaration de la classe de document
\begin{frame}[c,fragile]{Les parties d'un document}
	\framesubtitle{La déclaration de la classe de document}
	\begin{itemize}
		\item Un document commence toujours par la \textbf{commande} \cmd{documentclass}.
		
\begin{codesource}
	\documentclass[options]{classe}
\end{codesource}

		\item La \lien{https://en.wikibooks.org/wiki/LaTeX/Document_Structure\#Document_classes}{classe de document}
			détermine le type de document.		
		\item Plusieurs options de classe peuvent être utilisées pour changer la mise en page.
	\end{itemize}
\end{frame}

% Les parties d'un document : le corps du document
\begin{frame}[c,fragile]{Les parties d'un document}
	\framesubtitle{Le corps du document}
	Le contenu du document est rédigé dans l'\textbf{environnement} \texttt{document},
	entre les commandes \cmd{begin\{document\}} et \cmd{end\{document\}}.
\begin{codesource}
	\documentclass[options]{classe}
	
	\begin{document}
		Le contenu du document est rédigé ici...
	\end{document}
\end{codesource}
\end{frame}

% Les parties d'un document : le préambule
\begin{frame}[c,fragile]{Les parties d'un document}
	\framesubtitle{Le préambule}
	Tout ce qui se trouve avant la commande \cmd{begin\{document\}} constitue le \textbf{préambule} du document.
	
\begin{codesource}
	\documentclass[options]{classe}
	
	%% Ici se trouve le préambule du document...
	
	\begin{document}
		Ici se trouve le contenu du document...
	\end{document}
\end{codesource}

	Dans le préambule se trouvent:
	\begin{itemize}
		\item des \emph{packages};
		\item des commandes de configuration;
		\item des commandes et des environnements personnalisés;
		\item des métadonnées.
	\end{itemize}
\end{frame}

% Création d'un document plus complexe
\begin{frame}[c]{Création d'un document plus complexe}
	\begin{itemize}
		\item Ouvrez le premier document \texttt{.tex} que vous avez créé.
		\item Rendez-vous sur le \lien{http://www.hec.ca/nouvelles/index.html}{site des nouvelles de HEC Montréal}.
		\item Copiez-collez l'intégralité du contenu d'un article dans votre document.
		\item Sauvegardez et compilez votre document, puis observez le résultat.
	\end{itemize}
\end{frame}

\subsection{Personnalisation de \LaTeX}

% Préambule : les packages
\begin{frame}[fragile,c]{Préambule}
	\framesubtitle{Les \emph{packages}}
	Les \emph{packages} permettent de \textbf{modifier des commandes} ou d’\textbf{ajouter des fonctionnalités} au système.

	Ils sont chargés dans le préambule avec la commande \cmd{usepackage[options]\{package\}}.

\begin{codesource}
	\documentclass[options]{classe}
	
	\usepackage{package}
	\usepackage[options]{package}
	\usepackage{package1,package2,package3,...}
\end{codesource}

	La documentation de chaque package peut être consultée sur le site du
	\lien{https://ctan.org/}{Comprehensive \TeX\ Archive Network}.
\end{frame}

% Commandes
\begin{frame}[fragile]{Commandes}
	\begin{itemize}
		\item Débutent toujours par un \textbackslash
		\item Formes générales:
\begin{codesource}
	\nomcommande[args_optionnels]{args_obligatoires}
	\nomcommande*[args_optionnels]{args_obligatoires}
	\nomcommande
\end{codesource}
		\item Arguments obligatoires entre \{\ et \}
		\item Arguments optionnels entre [ et ]
		\item Commande sans argument : le nom se termine par tout caractère qui n’est pas une lettre (y
		compris l’espace)
		\item Portée d’une commande limitée à la zone entre \{\ et \}.
	\end{itemize}
\end{frame}

% Environnements
\begin{frame}[fragile,c]{Environnements}
	\begin{itemize}
		\item Délimités par
\begin{codesource}
	\begin{environnement}
		...
	\end{environnement}
\end{codesource}
		\item Contenu de l’environnement traité différemment du reste du texte
		\item Changements s’appliquent uniquement à l’intérieur de l’environnement
	\end{itemize}
\end{frame}

% Commandes et environnements personnalisés
\begin{frame}[c]{Commandes et environnements personnalisés}
	\begin{itemize}
		\item Vous pouvez \textbf{créer} de nouvelles commandes avec \cmd{newcommand}.
		\item Vous pouvez \textbf{modifier} des commandes existantes avec \cmd{renewcommand}.
		\item Vous pouvez \textbf{créer} de nouveaux environnements avec \cmd{newenvironment}.
		\item Vous pouvez \textbf{modifier} des environnements existants avec \cmd{renewenvironment}.
	\end{itemize}
\end{frame}

% LaTeX en français - préambule pour pdfLaTeX
\begin{frame}[fragile]{\LaTeX\ en français -- préambule pour pdf\LaTeX}
	Il faut charger un certain nombre de \emph{packages} pour franciser \LaTeX.

\begin{codesource}
	\documentclass[french]{hecthese}
	\usepackage[utf8]{inputenc}
	\usepackage[T1]{fontenc}
	\usepackage{babel}
	\usepackage[autolanguage]{numprint}
	\usepackage{icomma}
\end{codesource}

	\pause
	\begin{description}[inputenc et fontenc]
		\item[babel] traduction des mots-clés prédéfinis, typographie française, coupure de mots,
		document multilingue
		
		\pause
		\item[inputenc et fontenc] lettres accentuées dans le code source
		
		\pause
		\item[icomma] virgule comme séparateur décimal
		
		\pause
		\item[numprint] espace comme séparateur de milliers
	\end{description}
\end{frame}
\section{Writing}

\subsection{Basic formatting}

% Titre, auteur et date du document
\begin{frame}[fragile]{Title, author and date}
	\begin{itemize}
		\item Automatic formatting
\begin{codesource}
	\documentclass{article}
	
	\title{Document title}
	\author{Author name}
	\date{A date}
	
	\begin{document}
		\maketitle
		
		% Document content...
	\end{document}
\end{codesource}
		\item Manual formatting
\begin{codesource}
	\documentclass{article}
	
	\begin{document}
		\begin{titlepage}
			% Title page built manually...
		\end{titlepage}
	\end{document}
\end{codesource}
	\end{itemize}
\end{frame}

% Paragraphes, sauts de lignes et espaces blancs
\begin{frame}[c]{Paragraphs, line breaks and white space}
	\begin{itemize}
		\item \LaTeX\ automatically deletes all extra white spaces.
		\item Line breaks are created with \textbackslash\textbackslash.
		\item There needs to be at least one blank line between paragraphs in
			the code in order to distinguish them in the text.
	\end{itemize}
\end{frame}

% Caractères spéciaux
\begin{frame}{Reserved characters}
	\framesubtitle{\TeX\ reserved characters}
	\begin{description}[\#]
		\item[\#] Argument identifier in commands
		\item[\$] Math mode delimiter
		\item[\&] Column delimiter in tables
		\item[\%] Start of a comment
		\item[\_] Indice (math)
		\item[\textasciicircum] Exponent (math)
		\item[\textasciitilde] No-break space
		\item[\{] Opens a command or environment definition
		\item[\}] Closes a command or environment definition
	\end{description}
	\begin{picture}(0,0)
	\thicklines\color{bleuFonceSecondaire}
	\onslide<2>\put(90,5){\dashbox{1}(53,58){}}
	\onslide<2>\put(97,59){\textbf{\MakeUppercase{To use them:}}}
	\onslide<2>\put(94,55){\parbox[t]{.3\textwidth}{\centering\bfseries\textbackslash \# \\[5pt] %
			\textbackslash \$ \\[5pt] \textbackslash \& \\[5pt] \textbackslash \% \\[5pt] %
			\textbackslash \_ \\[5pt] \textbackslash textasciicircum \\[4pt] %
			\textbackslash textasciitilde \\[4pt] \textbackslash \{ \\[4pt] %
			\textbackslash \} }}
	\end{picture}
\end{frame}

% Caractères spéciaux - la suite
\begin{frame}[fragile]{Reserved characters}
	\framesubtitle{Part deux\ldots}
	\begin{itemize}
		\item Quote marks
			\begin{itemize}
				\item We open english single quotes with (\lstinline|`|)
					and double quotes with (\lstinline|``|). We close them with one
					(\lstinline|'|) or two (\lstinline|''|) apostrophes, depending on the case.
				\item On utilise les chevrons (« et ») pour ouvrir et fermer les guillemets français.
					Il faut cependant inscrire la commande suivante à la fin de notre préambule:
\begin{codesource}
	\frenchbsetup{og=«,fg=»}
\end{codesource}				
			\end{itemize}
		\item On inscrit les traits d'union avec un tiret (\lstinline|-|), les traits demi-cadratins avec deux tirets (\lstinline|--|) et les traits cadratins avec trois tirets (\lstinline|---|).
	\end{itemize}
\end{frame}

% Commentaires
\begin{frame}[c]{Commentaires}
\begin{itemize}
\item Pour se retrouver dans le code (ou des documents longs), il est conseillé 
d'y insérer des commentaires.
\item Ceux-ci commencent toujours avec le symbole \texttt{\%}.
\item Ils s'affichent dans le code, mais pas dans le document final.
\end{itemize}
\end{frame}

\subsection{Apparence du texte}

% Polices de caractères
\begin{frame}[c]{Polices de caractères}
	\begin{itemize}
		\item Par défaut, tous les documents \LaTeX\ utilisent la même police, \textrm{Computer Modern}.
		\item Privilégier les polices de grande qualité et très complètes (lettres accentuées, grand choix de symboles)
		\item Peu de polices sont adaptées pour les mathématiques : Palatino, Times, Lucida (\$) sont des choix sûrs
		\item Dans la classe \textbf{hecthese}, les paquetages mathptmx et mathpazo sont chargés par défaut afin	d’offrir les polices de caractères Times et Palatino.
	\end{itemize}
\end{frame}

% Changement d'attribut de la police
\begin{frame}{Changement d'attribut de la police}
	\begin{tabularx}{\textwidth}{XXX}
		\arrayrulecolor{grisPrimaire!40}\hline\hline
		\multicolumn{3}{l}{\textbf{familles}}	\\
		\hline
		\textrm{romain}						&	\cmd{rmfamily}		&	\cmd{textrm\{<texte>\}}\\
		\texttt{largeur fixe}				&	\cmd{ttfamily}		&	\cmd{texttt\{<texte>\}}\\
		sans empattements					&	\cmd{sffamily}		&	\cmd{textsf\{<texte>\}}\\
		\hline
		\multicolumn{3}{l}{\textbf{formes}}	\\
		\hline
		droit								&	\cmd{upshape}		&	\cmd{textup\{<texte>\}}\\
		\emph{italique}						&	\cmd{itshape}		&	\cmd{textit\{<texte>\}}\\
		\textsl{penché}						&	\cmd{slshape}		&	\cmd{textsl\{<texte>\}}\\
		\textrm{\textsc{petites capitales}}	&	\cmd{scshape}		&	\cmd{textsc\{<texte>\}}\\
		\hline
		\multicolumn{3}{l}{\textbf{séries}}	\\
		\hline
		\textmd{moyen}						&	\cmd{mdseries}		&	\cmd{textmd\{<texte>\}}\\
		\textbf{gras}						&	\cmd{bfseries}		&	\cmd{textbf\{<texte>\}}\\
		\hline\hline
	\end{tabularx}

	\begin{picture}(0,0)
		\thicklines\color{bleuFonceSecondaire}
		\onslide<2>\put(38,-7){\dashbox{1}(40,64)[b]{\parbox{.25\textwidth}{\centering\textbf{s'applique à tout le texte qui suit}}}}
		\onslide<3>\put(92,-7){\dashbox{1}(40,64)[b]{\parbox{.25\textwidth}{\centering\textbf{s'applique au texte en argument}}}}
	\end{picture}
\end{frame}

% Taille de la police
\begin{frame}{Taille de la police}
	\begin{tabularx}{\textwidth}{l|l}
		\arrayrulecolor{grisPrimaire!40}
		\textbf{Commandes standards} 	& 	\textbf{Rendu}	\\
		\hline
		\cmd{tiny}						&	{\tiny vraiment petit}	\\
		\cmd{scriptsize}				&	{\scriptsize encore plus petit}	\\
		\cmd{footnotesize}				&	{\footnotesize plus petit}	\\
		\cmd{small}						&	{\small petit}	\\
		\cmd{normalsize}				&	{\normalsize normal}	\\
		\cmd{large}						&	{\large grand}	\\
		\cmd{Large}						&	{\Large plus grand}	\\
		\cmd{LARGE}						&	{\LARGE encore plus grand}	\\
		\cmd{huge}						&	{\huge énorme}	\\
		\cmd{Huge}						&	{\Huge encore plus énorme}		
	\end{tabularx}
\end{frame}

% Caractères gras, italiques et soulignés
\begin{frame}[c]{Caractères gras, italiques et soulignés}
	\begin{itemize}
		\item Caractères \textbf{gras} : \cmd{textbf\{\}}
		\item Caractères \emph{italiques} :
		\begin{itemize}
			\item \cmd{textit\{\}}
			\item \cmd{emph\{\}} -- commande à privilégier
		\end{itemize}
		\item Caractères \underline{soulignés} : \cmd{underline\{\}}
	\end{itemize}
\end{frame}


\section{Disposition du texte}

% Alignement du texte
\begin{frame}[fragile]{Alignement du texte}
	\begin{itemize}
		\item Par défaut, le texte est pleinement justifié.
		\item Pour aligner le texte à gauche, on utilise l'environnement \texttt{flushleft}.
\begin{codesource}
\begin{flushleft}
	Le texte sera aligné à gauche.
\end{flushleft}
\end{codesource}
		\item On utilise l'environnement \texttt{center} pour centrer le texte.
\begin{codesource}
	\begin{center}
		Le texte sera centré.
	\end{center}
\end{codesource}
		\item Pour aligner le texte à droite, on utilise l'environnement \texttt{flushright}.
\begin{codesource}
	\begin{flushright}
		Le texte sera aligné à droite.
	\end{flushright}
\end{codesource}
	\end{itemize}
\end{frame}

% Listes
\begin{frame}[fragile]{Listes}
	\framesubtitle{Listes à puces et listes numérotées}
	\begin{itemize}
		\item Les listes à puces sont construites avec l'environnement \texttt{itemize}.
\begin{codesource}
	\begin{itemize}
		\item Premier item
		\item Deuxième item
		\item etc.
	\end{itemize}
\end{codesource}
		\item Les listes numérotées sont construites avec l'environnement \texttt{enumerate}.
\begin{codesource}
	\begin{enumerate}
		\item Premier item
		\item Deuxième item
		\item etc.
	\end{enumerate}
\end{codesource}
		\item La commande \cmd{item} est utilisée pour lister les items.
		\item On peut imbriquer jusqu'à quatre niveaux de listes.
	\end{itemize}
\end{frame}

\begin{frame}[c, fragile]{Listes}
	\framesubtitle{Listes de définitions}
	
	On crée une liste de définitions avec l'environnement \texttt{description}.
	
\begin{codesource}
	\begin{description}
		\item[Premier terme] Définition du premier terme.
		\item[Deuxième terme] Définition du deuxième terme.
	\end{description}
\end{codesource}

	\begin{description}
		\item[Premier terme] Définition du premier terme. Auctor est gravida habitasse leo lobortis mollis nec platea posuere
		 sollicitudin tempus.
		\item[Deuxième terme] Définition du deuxième terme. Aenean consequat dictumst dignissim duis facilisis himenaeos id
		 pharetra placerat porta posuere primis senectus tortor.
	\end{description}
\end{frame}

% Citations
\begin{frame}[fragile,c]{Citations}
\framesubtitle<1>{Citations courtes}
\framesubtitle<2>{Citations longues}
\begin{onlyenv}<1>
	On utilise l'environnement \texttt{quote} pour insérer une citation courte (un paragraphe)
	dans le texte.
	
	\begin{columns}
		\begin{column}{.49\textwidth}
			\vspace{-17mm}
\begin{codesource}
	\begin{quote}
		Life is what happens to you while 
		you're busy making other plans. 
		-- John Lennon
	\end{quote}
\end{codesource}
		\end{column}
		
		\begin{column}{.49\textwidth}
			\begin{quote}
				Life is what happens to you while you're busy making other plans. -- John Lennon
			\end{quote}
		\end{column}
	\end{columns}
\end{onlyenv}

\begin{onlyenv}<2>
	On utilise l'environnement \texttt{quotation} pour insérer une citation longue (plus
	d'un paragraphe).
	
	\begin{columns}
		\begin{column}{.49\textwidth}
			\vspace{-38mm}
\begin{codesource}
	\begin{quotation}
		I've missed more than 9000 shots in my 
		career. I've lost almost 300 games. 26 
		times I've been trusted to take the game 
		winning shot and missed.
		
		I've failed over and over and over again 
		in my life. And that is why I succeed. 
		-- Michael Jordan
	\end{quotation}
\end{codesource}	
		\end{column}
		
		\begin{column}{.49\textwidth}
			\begin{quotation}
				I've missed more than 9000 shots in my career. 
				I've lost almost 300 games. 26 times I've been 
				trusted to take the game winning shot and missed.
				
				I've failed over and over and over again in my life. 
				And that is why I succeed. -- Michael Jordan
			\end{quotation}
		\end{column}
	\end{columns}
\end{onlyenv}
\end{frame}

% Notes de bas de page
\begin{frame}[fragile,c]{Notes de bas de page}
\begin{itemize}
\item Une note de bas de page est insérée avec la commande suivante:
\begin{codesource}
	\footnote{texte de la note}
\end{codesource}
\item La commande doit suivre immédiatement le texte à annoter.
\item Méthode recommandée :
\begin{codesource}
	... fera remarquer que Pierre Lasou\footnote{%
		Spécialiste en ressources documentaires} %
	fut une grande aide dans la préparation de ...
\end{codesource}
\item La numérotation et la disposition sont automatiques.
\end{itemize}
\end{frame}

% Code source
\begin{frame}[fragile,c]{Code source}
\begin{onlyenv}<1>
\begin{itemize}
\item Pour rédiger du code source en bloc, on utilise l'environnement \texttt{verbatim}
\begin{codesource}
	\begin{verbatim}
        Texte disposé tel qu'il est saisi
        dans une police à largeur fixe.
	\end{verbatim}
\end{codesource}
\item Pour rédiger du code source à même le texte, on utilise la commande \cmd{verb}, dont la
syntaxe est \cmd{verbcsourcec} où \emph{c} est un caractère quelconque ne se trouvant pas dans \emph{source}.
\begin{codesource}
	Du texte avec \verb|du code|.
\end{codesource}
\item Pour un usage plus intensif, consultez la documentation du \emph{package} \textbf{listings}.
\end{itemize}
\end{onlyenv}

\begin{onlyenv}<2>
Un exemple\footnote{tiré du site \href{http://r4stats.com/examples/programming/}{r4stats.com}.} :
\begin{codesource}
# ---Writing Your Own Functions (Macros)---

# A good function that just prints.
mystats <- function(x) {
	print( mean(x, na.rm = TRUE) )
	print(   sd(x, na.rm = TRUE) )
}
mystats(myvar)

# A function with vector output.
mystats  <- function(x) {
	mymean <- mean(x, na.rm = TRUE)
	mysd   <-   sd(x, na.rm = TRUE)
	c(mean = mymean, sd = mysd )
}
mystats(myvar)
myVector <- mystats(myvar)
myVector
\end{codesource}
\end{onlyenv}
\end{frame}
% Organisation d'un document

\section{Organisation d'un document}

\subsection{Parties d'un document}

% Choix d'une classe
\begin{frame}[c]{Choix d'une classe}
	La première chose que l'on doit faire lorsqu'on débute la rédaction d'un document \LaTeX,
	c'est de choisir une classe de document.
	
	\begin{table}[c]
		\begin{tabularx}{\textwidth}{lllll}
			\arrayrulecolor{grisPrimaire!40}\hline\hline
			\textbf{Classe} & \textbf{Divisions} & \textbf{Disposition} & \textbf{Entête} &	\textbf{Pied de page} \\
			\hline
			\texttt{article}			&	parties, sections, \ldots				&	recto		&	vide			&	folio centré \\
			\texttt{report}				&	parties, chapitres, sections, \ldots	&	recto		&	vide			&	folio centré \\
			\texttt{book}				&	parties, chapitres, sections, \ldots	&	recto verso	&
			folio, titres	&	vide \\
			\texttt{hecthese}	&	chapitres, sections, sous-sections		&	recto verso	&
			vide			&	folio centré \\
			\hline\hline
		\end{tabularx}
	\end{table}
\end{frame}

% Résumé
\begin{frame}[fragile,c]{Résumé}
	\begin{itemize}
		\item Classes \textbf{article}, \textbf{report} ou \textbf{memoir}: résumé créé avec
		l'environnement \lstinline|abstract|
\begin{codesource}
	\begin{abstract}
		...
	\end{abstract}
\end{codesource}

		\item Classe \textbf{hecthese} : résumés français et anglais traités comme des chapitres
		normaux (non numérotés)
	\end{itemize}
\end{frame}

% Sections
\begin{frame}[fragile]{Sections}
	\begin{itemize}
		\item Découpage du document en sections avec les commandes suivantes :
\begin{codesource}
	\part[titre court]{titre au long}
	\chapter[titre court]{titre au long}
	\section[titre court]{titre au long}
	\subsection[titre court]{titre au long}
	
	\subsubsection[titre court]{titre au long} 	% à éviter dans un livre
	
	\paragraph[titre court]{titre au long} 		% ne jamais utiliser
	\subparagraph[titre court]{titre au long} 	% ne jamais JAMAIS utiliser
\end{codesource}

		\item Numérotation automatique
		\item Commande suivie d'un * = section non numérotée
		\item Titre court en argument optionnel
	\end{itemize}
\end{frame}

% Annexes
\begin{frame}[fragile,c]{Annexes}
	\begin{itemize}
		\item Les annexes sont des sections ou des chapitres avec une numérotation alphanumérique (A,
		A.1, \ldots).
		\item Les sections suivantes sont identifiées comme des annexes par la commande 
			\cmd{appendix}.
		\item Dans le titre, «Chapitre» est changé pour «Annexe».
	\end{itemize}
\end{frame}

% Structure logique d'un livre
\begin{frame}[fragile]{Structure logique d'un livre}
	\framesubtitle{Classes book, memoir, hecthese}

\begin{onlyenv}<1>
\begin{codesource}
	\frontmatter
\end{codesource}	
	\begin{itemize}
		\item préface, table des matières, etc.
		\item numérotation des pages en chiffres romains (i, ii, \ldots)
		\item chapitres non numérotés
	\end{itemize}
\begin{codesource}
	\mainmatter
\end{codesource}	
	\begin{itemize}
		\item le contenu à proprement parler
		\item numérotation des pages à partir de 1 en chiffres arabes
		\item chapitres numérotés
	\end{itemize}
\end{onlyenv}

\begin{onlyenv}<2>
\begin{codesource}
	\backmatter
\end{codesource}
	\begin{itemize}
		\item tout le reste (bibliographie, index, etc.)
		\item numérotation des pages se poursuit
		\item chapitres non numérotés
	\end{itemize}
\end{onlyenv}
\end{frame}

\subsection{Table des matières et renvois}

% Table des matières
\begin{frame}[fragile,c]{Table des matières}
	
	\begin{itemize}
		\item La table des matières est produite automatiquement avec \cmd{tableofcontents}.
		\item Requiert \textbf{plusieurs} compilations.
		\item Les sections non numérotées ne sont pas incluses.
		\item Avec le \emph{package} \textbf{hyperref}, \cmd{tableofcontents} produit également la table des matières du fichier .pdf.
		\pause
		\item La classe memoir fournit également \cmd{tableofcontents*} qui n’insère pas la table des matières dans la table des matières.
		\pause
		\item \cmd{listoffigures} produit la liste des figures.
		\item \cmd{listoftables} produit la liste des tableaux.
	\end{itemize}

\end{frame}

% Étiquettes et renvois automatiques
\begin{frame}[fragile]{Étiquettes et renvois automatiques}
	\framesubtitle{Parce que l'ordinateur le fera mieux que vous\ldots}
	\begin{onlyenv}<1>
		\begin{itemize}
			\item Ne \textbf{jamais} renvoyer manuellement à un numéro de section, d’équation, de tableau, etc.
			\item «Nommer» un élément avec \cmd{label}
			\item Faire référence par son nom avec \cmd{ref}
			\item Requiert 2 à 3 compilations
		\end{itemize}
	
\begin{codesource}
	\section{Définitions}
		\label{sec:definitions}
	
		Lorem ipsum dolor sit amet, consectetur adipiscing elit, 
		sed do eiusmod tempor incididunt ut labore et dolore magna aliqua. 
		Ut enim ad minim veniam, quis nostrud exercitation ullamco laboris 
		nisi ut aliquip ex ea commodo consequat.
	
	\section{Historique}
		Tel que vu à la section \ref{sec:definitions}...
\end{codesource}
	\end{onlyenv}
	\begin{onlyenv}<2>
		\begin{itemize}
			\item Le \emph{package} \textbf{hyperref} insère des hyperliens vers des renvois dans les fichiers .pdf.
			\item La commande \cmd{autoref\{\}} permet de:
				\begin{enumerate}
					\item nommer automatiquement le type de renvoi (section, équation, tableau, etc.);
					\item transformer en hyperlien le texte \textbf{et} le numéro de la référence.
\begin{codesource}
	Tel que vu à la \autoref{sec:definitions}...
\end{codesource}
				\end{enumerate}
			\item La commande \cmd{pageref\{\}} renvoie à la page de la référence.
			\item Le \emph{package} \textbf{amsmath} fournit la commande \cmd{eqref\{\}} pour
				référencer les équations.
		\end{itemize}
	\end{onlyenv}
\end{frame}
% Classe hecthese

\section{hecthese document class}

\begin{frame}{hecthese document class}
	\begin{itemize}
		\item Document class created specifically for the M.Sc. and Ph.D. students from
			HEC Montréal;
		\item Available at \lien{https://ctan.org/pkg/hecthese}{https://ctan.org/pkg/hecthese};
		\item Layout fully complies with the presentation standards of the 
			\lien{http://www.hec.ca/qualitecomm/caf/guide-redaction-travail-cycles.pdf}{%
			Guidelines for Writing an Academic Work at a Graduate Level};
		\item Based on the \textbf{memoir} document class;
		\item Provides new commands for title page creation and more\ldots
		\item New adapted environments;
		\item You start from a base template (available after installing the package in a working directory);
		\item You use separate files for each chapter of your dissertation or thesis.
	\end{itemize}
\end{frame}
% Bibliographie
\scriptsize

\section{Bibliography}

\begin{frame}[c]{Bibliography}
	\framesubtitle{For those who still prefer the scent of ink}
	\setbeamertemplate{bibliography item}[book]
		
	\begin{thebibliography}{99}		
		\bibitem[Kopka and Daly, 2004]{kopkadaly:2004}
			Kopka, Helmut and Patrick W. Daly (2004).
			\newblock Guide to \LaTeX, Fourth Edition,
			\newblock Addison-Wesley,
			\newblock ISBN 978-0-321-17385-0, 597 p.
		\bibitem[Mittelbach et al., 2004]{mittelbach:2004}
			Mittelbach, Frank \emph{et al.} (2004).
			\newblock The \LaTeX\ Companion, Second Edition,
			\newblock Addison-Wesley,
			\newblock ISBN 978-0201362992, 1120p.
		\bibitem[Goossens and Mittelbach, 2007]{goossens:2007}
			Goossens, Michel and Franck Mittelbach (2007).
			\newblock The \LaTeX\ Graphics Companion, Second Edition,
			\newblock Addison-Wesley,
			\newblock ISBN 978-0321508928, 976p.
	\end{thebibliography}

\end{frame}

\begin{frame}[c]{Bibliography}
	\framesubtitle{For the environmentally conscious}
	\setbeamertemplate{bibliography item}[online]
	
	\begin{onlyenv}<1>
		\begin{thebibliography}{99}
			\bibitem[Goulet, 2016]{goulet:2016}
				Goulet, Vincent (2016).
				\newblock formation-latex-ul -- Introductory \LaTeX\ course in French,
				\newblock Comprehensive \TeX\ Archive Network,
				\newblock Viewed on February 22, 2018 at \href{https://ctan.org/pkg/formation-latex-ul}{%
					https://ctan.org/pkg/formation-latex-ul}
			\bibitem[Lees-Miller, 2018]{leesmiller:2018}
				Lees-Miller, John D. (2018).
				\newblock Free \& Interactive Online Introduction to \LaTeX,
				\newblock Overleaf,
				\newblock Viewed on February 22, 2018 at \href{https://www.overleaf.com/latex/learn/free-online-introduction-to-latex-part-1}{%
					https://www.overleaf.com/latex/learn/free-online-introduction-to-latex-part-1}
			\bibitem[ShareLaTeX, 2018]{sharelatex:2018}
				Share\LaTeX\ Documentation,
				\newblock Share\LaTeX,
				\newblock Viewed on February 22, 2018 at \href{https://fr.sharelatex.com/learn/Main_Page}{%
					https://fr.sharelatex.com/learn/Main\_Page}			
		\end{thebibliography}
	\end{onlyenv}

	\begin{onlyenv}<2>
		\begin{thebibliography}{99}
			\bibitem[LaTeX Wikibook]{wikibook}
				\href{https://en.wikibooks.org/wiki/LaTeX}{\LaTeX\ WikiBook}
			\bibitem[ShareLaTeX]{sharelatex}
				\href{https://fr.sharelatex.com/learn}{Share\LaTeX\ Documentation}
			\bibitem[Stack Exchange]{stackex}
				\href{https://tex.stackexchange.com/}{\TeX\ - \LaTeX\ Stack Exchange}
			\bibitem[LaTeX Community]{latexcomm}
				\href{http://latex.org/forum/}{\LaTeX\ Community}
			\bibitem[CTAN]{ctan}
				\href{https://ctan.org/}{Comprehensive \TeX\ Archive Network}
			\bibitem[TeX FAQ]{texfaq}
				\href{http://www.tex.ac.uk/}{UK List of TEX Frequently Asked Questions}
			\bibitem{google}
				Google\ldots
		\end{thebibliography}
	\end{onlyenv}
\end{frame}

% Période de questions
\begin{frame}[c]{Questions and comments}
\LARGE
\begin{block}{Training Session Documentation}
\ttfamily\href{http://bit.ly/enltxhec1}{http://bit.ly/enltxhec1}
\end{block}
\begin{block}{Training Session Evaluation}
\ttfamily\href{http://bit.ly/enltxsurvey1}{http://bit.ly/enltxsurvey1}
\end{block}
\begin{block}{\TeX nical Support}
\ttfamily Benoit Hamel : <benoit.2.hamel@hec.ca>
\end{block}
\end{frame}

%Crédits
\begin{frame}{Crédits}
	\begin{itemize}
		\item ``At the beginning, there was \TeX'' montage, page \ref{fr:commencement}
		\begin{itemize}
			\scriptsize
			\item \href{https://en.wikipedia.org/wiki/The_Creation_of_the_Sun,_Moon_and_Vegetation}{«La création du Soleil, de la Lune et des plantes», Michel-Ange}
			\item \href{https://fr.m.wikipedia.org/wiki/TeX}{\TeX\ logo}
			\item \href{https://www-cs-faculty.stanford.edu/~knuth/}{Portrait de Donald Knuth}
		\end{itemize}
		\item «On the sixth day, there was \LaTeX » montage, page \ref{fr:sixiemejour}
		\begin{itemize}
			\scriptsize
			\item \href{https://en.wikipedia.org/wiki/The_Creation_of_Adam}{«La création d'Adam», Michel-Ange}
			\item \href{https://fr.m.wikipedia.org/wiki/TeX}{\TeX\ logo}
			\item \href{https://fr.m.wikipedia.org/wiki/LaTeX}{\LaTeX\ logo}
			\item \href{https://www-cs-faculty.stanford.edu/~knuth/}{Portrait de Donald Knuth}
			\item \href{https://en.wikipedia.org/wiki/Leslie_Lamport}{Portrait de Leslie Lamport}
		\end{itemize}
	\end{itemize}
\end{frame}

\end{document}
