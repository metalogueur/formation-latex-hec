\documentclass[aspectratio=1610,compress,t,gabaritb,english,french]{hecppt}

%% Packages
\usepackage{fontawesome}
\usepackage{metalogo}
\usepackage{listings}
\usepackage{tabularx}
\usepackage{colortbl}
\usepackage{hyperref}

%% Environnements
\lstnewenvironment{codesource}{%
	\lstset{%
		basicstyle=\tiny,
		language=[LaTeX]TeX,
		backgroundcolor=\color{bleuPaleSecondaire!10},
		tabsize=2,
		% frame=leftline,
		% numbers=left,
		% numberstyle=\tiny,
		literate=%
		{à}{{\`a}}1
		{é}{{\'e}}1
		{ç}{{\c c}}1
		{«}{{\og}}1
		{»}{{\fg}}1
	}
}{}

%% Options des packages
\hypersetup{colorlinks=true,%
	urlcolor=bleuFoncePrimaire,%
	linkcolor=white,%
	pdfauthor=Benoit Hamel,%
	pdftitle=Rédaction avec LaTeX : Principes de base}
\frenchbsetup{og=«,fg=»}
\setlength{\parskip}{1ex}

%% Métadonnées du document

\title{Rédaction avec \\ \texttt{\textbackslash title}\{\textrm{\LaTeX}\} }
\subtitle{Principes de base}
\HECauteur{Benoit Hamel}{Benoit Hamel}
\date[]{\today}
\subject{} % Sujet inséré dans les métadonnées du pdf
\keywords{} % Mots-clés insérés dans les métadonnées du pdf

\begin{document}

\pageTitre

% Pages liminaires

% Page titre
\begin{frame}
	Benoit Hamel \\
	Technicien en documentation, soutien technique \\
	Bibliothèque HEC Montréal
	\vfill
	{
		\Huge\bfseries
		Rédaction avec \\
		\texttt{\textbackslash title\{\textrm{\LaTeX}\}}
	}
	\vfill
	Édition HEC Montréal, revue et augmentée (version française)
\end{frame}

% Page de la licence
\begin{frame}
	\faCopyright\ 2016 Vincent Goulet pour la 
	\href{https://ctan.org/pkg/formation-latex-ul}{version originale}. La liste des sources qui ont 
	servi à l'élaboration de cette formation se trouve à la fin du présent document.
	
	\faCreativeCommons\ Cette création est mise à disposition selon le contrat 
	\href{http://creativecommons.org/licenses/by-sa/4.0/deed.fr}{%
	Attribution-Partage dans les mêmes conditions 4.0 International de Creative Commons}. 
	En vertu de ce contrat, vous êtes libre de :
	
	\begin{itemize}
		\item partager -- reproduire, distribuer et communiquer l’oeuvre;
		\item remixer -- adapter l’oeuvre;
		\item utiliser cette oeuvre à des fins commerciales.
	\end{itemize}

	Selon les conditions suivantes :
	
	\begin{itemize}
		\item Attribution -- Vous devez créditer l’oeuvre, intégrer un lien vers le contrat et indiquer si des modifications ont été effectuées à l’oeuvre. Vous devez indiquer ces informations par tous les moyens possibles, mais vous ne pouvez suggérer que l’Offrant vous soutient ou soutient la façon dont vous avez utilisé son oeuvre.
		\item Partage dans les mêmes conditions -- Dans le cas où vous modifiez, transformez ou créez à partir du matériel composant l’oeuvre
		originale, vous devez diffuser l’oeuvre modifiée dans les même conditions, c’est à dire avec le même contrat avec lequel l’oeuvre originale a été diffusée.
	\end{itemize}
\end{frame}

% Table des matières
\begin{frame}{Sommaire de la formation}
	\tableofcontents
\end{frame}
% Présentation de TeX et LaTeX
\small

\section{Présentation de \TeX\ et \LaTeX}

\subsection{Qu'est-ce que \TeX\ et \LaTeX?}

\begin{frame}[c,label=fr:commencement]{Au commencement (1978), il y eut \TeX\ldots}
	\includegraphics[width=\textwidth,keepaspectratio=true]{knuth-tex-commencement.jpg}
\end{frame}

\begin{frame}[c]{Qu'est-ce que \TeX?}

	\begin{itemize}
		\item Un système de mise en page (\emph{typesetting}) et de préparation de documents;
		\item «Le système le plus puissant pour produire des ouvrages scientifiques et
		techniques d'une grande qualité typographique»\footnote{Kopka \& Daly, p. 6};
		\item Un système mature, stable et complet, considéré comme exempt de bogues;
		\item Un ensemble de commandes très primitives parfaites pour la typographie et
		des fonctions de programmation;
		\item «\emph{typesetter-level program}».
	\end{itemize}

\end{frame}

\begin{frame}[c,label=fr:sixiemejour]{Au sixième jour (1983), il y eut \LaTeX\ldots}
	\includegraphics[width=\textwidth,keepaspectratio=true]{creation-of-latex.jpg}
\end{frame}

\begin{frame}{Qu'est-ce que \LaTeX?}
	\begin{itemize}
		\item Un ensemble de macro-commandes pour faciliter l'utilisation de \TeX.
		\item Ne requiert aucune connaissance préalable de la typographie en général et de \TeX\ en particulier.
		\item Langage de balisage (\emph{Markup Language}) typographique et logique pour indiquer la mise en forme du texte (pensez au HTML).
		\item Langage multiplateforme, identique d'un système d'exploitation à l'autre, et extensible par l'ajout de \emph{packages}.
		\item «\emph{author-level program}»
	\end{itemize}
\end{frame}

\subsection{Processus de création d'un document \LaTeX}

% Rédiger avec une nouvelle perspective
\begin{frame}[c]{Rédiger avec une nouvelle perspective}
	
	\begin{itemize}
		\item Vous rédigez votre document en texte brut et utilisez des commandes pour décrire
			\textbf{ce que votre texte représente} et \textbf{non pas ce à quoi il doit ressembler}.
		\item Vous vous concentrez sur votre \textbf{contenu}.
		\item Vous laissez \LaTeX\ faire son travail, c'est-à-dire s'occuper du \textbf{contenant}.
	\end{itemize}
	
\end{frame}

% Processus de création d'un document LaTeX
\begin{frame}[c]{Processus de création d'un document \LaTeX}
	\Huge
	\begin{minipage}[t]{0.25\linewidth}
		\centering
		\faFileTextO
	\end{minipage}
	\hfill\faArrowRight\hfill
	\begin{minipage}[t]{0.25\linewidth}
		\centering
		\faCogs
	\end{minipage}
	\hfill\faArrowRight\hfill
	\begin{minipage}[t]{0.25\linewidth}
		\centering
		\faFilePdfO
	\end{minipage}

	\begin{picture}(0,0)
		\footnotesize\thicklines\color{bleuFonceSecondaire}
		\onslide<2>\put(0,-10){\dashbox{1}(35,40)[b]{\parbox{.2\textwidth}{\centering\textbf{rédaction du texte et balisage avec éditeur de texte\smallskip}}}}
		\onslide<3>\put(54,-10){\dashbox{1}(35,40)[b]{\parbox{.2\textwidth}{\centering\textbf{compilation avec un moteur \TeX\ à partir de la ligne de commande\smallskip}}}}
		\onslide<4>\put(108,-10){\dashbox{1}(35,40)[b]{\parbox{.2\textwidth}{\centering\textbf{visualisation avec une visionneuse externe\smallskip}}}}
	\end{picture}
\end{frame}
% Principes de base

\section{Principes de base}

\subsection{Structure d'un document}

% Structure d'un document
\begin{frame}[fragile]{Structure d'un document}

	Un document \LaTeX\ est toujours composé de deux parties :
	
\begin{codesource}
	
	\documentclass[11pt,french]{article}
	\usepackage[utf8]{inputenc}
	\usepackage[T1]{fontenc}
	\usepackage{babel}
	\usepackage[autolanguage]{numprint}
	
	\begin{document}
		
		\section{Primo}
		
		Ac class dis donec erat facilisis magna mattis 
		placerat potenti praesent primis sed tellus turpis 
		ut vehicula. Ad amet eleifend eros fames habitant 
		imperdiet integer laoreet leo magna magnis neque 
		netus senectus taciti torquent. 
		
		\section{Deuxio}
		
		Cursus dui egestas eget eros et hac magna massa mollis 
		natoque penatibus sagittis sed tellus urna velit 
		vestibulum vitae vulputate. 
	\end{document}
\end{codesource}

	\begin{picture}(0,0)
		\thicklines\color{bleuFonceSecondaire}
		\onslide<2>\put(1,47){\dashbox{1}(87,15){}}
		\onslide<2>\put(89,53){\Large\textbf{\faArrowLeft\ Préambule}}
		\onslide<3>\put(1,6){\dashbox{1}(87,41){}}
		\onslide<3>\put(89,24){\Large\textbf{\faArrowLeft\ Corps du document}}
	\end{picture}
\end{frame}

% Préambule : la classe de document
\begin{frame}[fragile]{Préambule}
	\framesubtitle{La classe de document}
	La \textbf{première commande} du préambule est normalement la déclaration de la classe.
	
\begin{codesource}
	\documentclass[options]{classe}	
\end{codesource}

	\begin{columns}
		
		\pause
		
		\begin{HECcomparaison}{Principales classes}
			\begin{itemize}
				\item article, book, letter, report
				\item memoir, \textbf{hecthese}
				\item slides, beamer, \textbf{hecppt}
			\end{itemize}
		\end{HECcomparaison}
	
		\pause
		
		\begin{HECcomparaison}{Principales options}
			\begin{itemize}
				\item 10pt, 11pt, 12pt
				\item oneside, twoside
				\item openright, openany
				\item english, french
			\end{itemize}
		\end{HECcomparaison}
	\end{columns}
\end{frame}

% Préambule : les packages
\begin{frame}[fragile,c]{Préambule}
	\framesubtitle{Les \emph{packages}}
	Les \emph{packages} permettent de \textbf{modifier des commandes} ou d’\textbf{ajouter des fonctionnalités} au système.
	
	Ils sont chargés dans le préambule avec la commande \lstinline|\usepackage[options]{package}|.
	
\begin{codesource}
	\documentclass[options]{classe}
	
	\usepackage{package}
	\usepackage[options]{package}
	\usepackage{package1,package2,package3,...}
\end{codesource}

	La documentation de chaque package peut être consultée sur le site du
	\href{https://ctan.org/}{Comprehensive \TeX\ Archive Network}.
\end{frame}

% Commandes
\begin{frame}[fragile]{Commandes}
	\begin{itemize}
		\item Débutent toujours par un \textbackslash
		\item Formes générales:
\begin{codesource}
	\nomcommande[args_optionnels]{args_obligatoires}
	\nomcommande*[args_optionnels]{args_obligatoires}
	\nomcommande
\end{codesource}
		\item Arguments obligatoires entre \{\ et \}
		\item Arguments optionnels entre [ et ]
		\item Commande sans argument : le nom se termine par tout caractère qui n’est pas une lettre (y
		compris l’espace)
		\item Portée d’une commande limitée à la zone entre \{\ et \}.
	\end{itemize}
\end{frame}

% Environnements
\begin{frame}[fragile,c]{Environnements}
	\begin{itemize}
		\item Délimités par
\begin{codesource}
	\begin{environnement}
		...
	\end{environnement}
\end{codesource}
		\item Contenu de l’environnement traité différemment du reste du texte
		\item Changements s’appliquent uniquement à l’intérieur de l’environnement
	\end{itemize}
\end{frame}

\subsection{Rédaction}

% Rédaction
\begin{frame}[fragile,c]{Rédaction}
	\begin{itemize}
		\item On rédige notre texte à l'intérieur de l'environnement \lstinline|document|:
\begin{codesource}
	\begin{document}
		Le contenu de votre travail est rédigé ici...
	\end{document}
\end{codesource}
		\item On rédige notre document en texte brut et on utilise les commandes et les environnements
		pour structurer notre texte;
		\item On rédige notre texte comme n'importe où ailleurs:
			\begin{itemize}
				\item Les mots sont séparés par un ou plusieurs espaces;
				\item Les paragraphes sont séparés par une ou plusieurs lignes blanches;
				\item Tous les espaces blancs supplémentaires sont supprimés à la compilation.
			\end{itemize}
	\end{itemize}
\end{frame}

% Caractères spéciaux
\begin{frame}{Caractères spéciaux}
	\framesubtitle{Caractères réservés par \TeX}
	\begin{description}[\#]
		\item[\#] Numéro d'argument dans les commandes
		\item[\$] Délimiteur du mode mathématique
		\item[\&] Délimiteur de colonne dans les tableaux
		\item[\%] Annonce le début d'un commentaire
		\item[\_] Indice (mathématiques)
		\item[\textasciicircum] Exposant (mathématiques)
		\item[\textasciitilde] Espace insécable
		\item[\{] Ouvre une définition de commande ou d'environnement
		\item[\}] Ferme une définition de commande ou d'environnement
	\end{description}
	\begin{picture}(0,0)
	\thicklines\color{bleuFonceSecondaire}
	\onslide<2>\put(90,5){\dashbox{1}(53,58){}}
	\onslide<2>\put(97,59){\textbf{\MakeUppercase{Pour les utiliser:}}}
	\onslide<2>\put(94,55){\parbox[t]{.3\textwidth}{\centering\bfseries\textbackslash \# \\[5pt] %
			\textbackslash \$ \\[5pt] \textbackslash \& \\[5pt] \textbackslash \% \\[5pt] %
			\textbackslash \_ \\[5pt] \textbackslash textasciicircum \\[4pt] %
			\textbackslash textasciitilde \\[4pt] \textbackslash \{ \\[4pt] %
			\textbackslash \} }}
	\end{picture}
\end{frame}

% Diacritiques et LaTeX
\begin{frame}[fragile,c]{Diacritiques et ligatures dans \LaTeX}
	\LaTeX\ ne supporte pas les diacritiques de manière native.
	\begin{columns}				
		\begin{column}{.49\textwidth}
			\vspace{-5.2mm}
\begin{codesource}
	 	\'{E}crire \`{a} la fran\c{c}aise
	 	peut \^{e}tre vraiment p\'{e}nible
	 	si on ne conna\^{i}t pas le truc\ldots
\end{codesource}
		\end{column}
		\begin{column}{.49\textwidth}
			Écrire à la française peut être vraiment pénible si on ne connaît pas
			le truc\ldots
		\end{column}
	\end{columns}

	On peut apprendre la \href{https://en.wikibooks.org/wiki/LaTeX/Special_Characters#Escaped_codes}{liste des commandes} 
	par coeur\ldots ou on peut ajouter des fonctionnalités à \LaTeX\ pour le franciser.
\end{frame}

% LaTeX en français - préambule pour pdfLaTeX
\begin{frame}[fragile]{\LaTeX\ en français -- préambule pour pdf\LaTeX}
	Il faut charger un certain nombre de \emph{packages} pour franciser \LaTeX.
	
\begin{codesource}
	\documentclass[french]{hecthese}
	\usepackage[utf8]{inputenc}
	\usepackage[T1]{fontenc}
	\usepackage{babel}
	\usepackage[autolanguage]{numprint}
	\usepackage{icomma}
\end{codesource}

	\pause
	\begin{description}[inputenc et fontenc]
		\item[babel] traduction des mots-clés prédéfinis, typographie française, coupure de mots,
			document multilingue
			
		\pause
		\item[inputenc et fontenc] lettres accentuées dans le code source
		
		\pause
		\item[icomma] virgule comme séparateur décimal
		
		\pause
		\item[numprint] espace comme séparateur de milliers
	\end{description}
\end{frame}

% Caractères spéciaux - la suite
\begin{frame}[fragile]{Caractères spéciaux}
	\framesubtitle{La suite\ldots}
	\begin{itemize}
		\item Guillemets
			\begin{itemize}
				\item On ouvre les guillemets anglais simples avec un accent grave (\lstinline|`|)
					et les doubles avec deux accents graves (\lstinline|``|). On les ferme avec un 
					(\lstinline|'|) ou deux (\lstinline|''|) apostrophes, selon la situation.
				\item On utilise les chevrons (« et ») pour ouvrir et fermer les guillemets français.
					Il faut cependant inscrire la commande suivante à la fin de notre préambule:
\begin{codesource}
	\frenchbsetup{og=«,fg=»}
\end{codesource}				
			\end{itemize}
		\item On inscrit les traits d'union avec un tiret (\lstinline|-|), les traits demi-cadratins avec deux tirets (\lstinline|--|) et les traits cadratins avec trois tirets (\lstinline|---|).
	\end{itemize}
\end{frame}
% Organisation d'un document

\section{Organisation d'un document}

\subsection{Parties d'un document}

% Choix d'une classe
\begin{frame}[c]{Choix d'une classe}
	La première chose que l'on doit faire lorsqu'on débute la rédaction d'un document \LaTeX,
	c'est de choisir une classe de document.
	
	\begin{table}[c]
		\begin{tabularx}{\textwidth}{lllll}
			\arrayrulecolor{grisPrimaire!40}\hline\hline
			\textbf{Classe} & \textbf{Divisions} & \textbf{Disposition} & \textbf{Entête} &	\textbf{Pied de page} \\
			\hline
			\texttt{article}			&	parties, sections, \ldots				&	recto		&	vide			&	folio centré \\
			\texttt{report}				&	parties, chapitres, sections, \ldots	&	recto		&	vide			&	folio centré \\
			\texttt{book}				&	parties, chapitres, sections, \ldots	&	recto verso	&
			folio, titres	&	vide \\
			\texttt{hecthese}	&	chapitres, sections, sous-sections		&	recto verso	&
			vide			&	folio centré \\
			\hline\hline
		\end{tabularx}
	\end{table}
\end{frame}

% Titre et page de titre
\begin{frame}[fragile]{Titre et page de titre}
	Mise en forme automatique :
\begin{codesource}
	% Commandes du préambule
	\title[titre court]{titre au long}
	\author[nom(s) d'auteur(s) court(s)]{noms des auteurs au long}
	\date[date courte]{date au long}
	[...]
	
	% Commande du corps du document
	\maketitle
\end{codesource}

	Mise en forme libre :
		\begin{columns}
			\begin{HECcomparaison}{Classes standards}
\begin{codesource}
	\begin{titlepage}
		...
	\end{titlepage}
\end{codesource}
			\end{HECcomparaison}
			\begin{HECcomparaison}{Classes memoir et hecthese}
\begin{codesource}
	\begin{titlingpage}
		...
	\end{titlingpage}
\end{codesource}	
			\end{HECcomparaison}
		\end{columns}
	
	Dans la classe \textbf{hecthese}, les pages titre sont générées automatiquement.
\end{frame}

% Résumé
\begin{frame}[fragile,c]{Résumé}
	\begin{itemize}
		\item Classes \textbf{article}, \textbf{report} ou \textbf{memoir}: résumé créé avec
		l'environnement \lstinline|abstract|
\begin{codesource}
	\begin{abstract}
		...
	\end{abstract}
\end{codesource}

		\item Classe \textbf{hecthese} : résumés français et anglais traités comme des chapitres
		normaux (non numérotés)
	\end{itemize}
\end{frame}

% Sections
\begin{frame}[fragile]{Sections}
	\begin{itemize}
		\item Découpage du document en sections avec les commandes
\begin{codesource}
	\part[titre court]{titre au long}
	\chapter[titre court]{titre au long}
	\section[titre court]{titre au long}
	\subsection[titre court]{titre au long}
	
	\subsubsection[titre court]{titre au long} 	% à éviter dans un livre
	
	\paragraph[titre court]{titre au long} 		% ne jamais utiliser
	\subparagraph[titre court]{titre au long} 	% ne jamais JAMAIS utiliser
\end{codesource}

		\item Numérotation automatique
		\item Commande suivie d'un * = section non numérotée
		\item Titre court en argument optionnel
	\end{itemize}
\end{frame}

% Annexes
\begin{frame}[fragile,c]{Annexes}
	\begin{itemize}
		\item Les annexes sont des sections ou des chapitres avec une numérotation alphanumérique (A,
		A.1, \ldots).
		\item Les sections suivantes sont identifiées comme des annexes par la commande 
			\lstinline|\appendix|.
		\item Dans le titre, «Chapitre» est changé pour «Annexe».
	\end{itemize}
\end{frame}

% Structure logique d'un livre
\begin{frame}[fragile]{Structure logique d'un livre}
	\framesubtitle{Classes book, memoir, hecthese}

\begin{onlyenv}<1>
\begin{codesource}
	\frontmatter
\end{codesource}	
	\begin{itemize}
		\item préface, table des matières, etc.
		\item numérotation des pages en chiffres romains (i, ii, \ldots)
		\item chapitres non numérotés
	\end{itemize}
\begin{codesource}
	\mainmatter
\end{codesource}	
	\begin{itemize}
		\item le contenu à proprement parler
		\item numérotation des pages à partir de 1 en chiffres arabes
		\item chapitres numérotés
	\end{itemize}
\end{onlyenv}

\begin{onlyenv}<2>
\begin{codesource}
	\backmatter
\end{codesource}
	\begin{itemize}
		\item tout le reste (bibliographie, index, etc.)
		\item numérotation des pages se poursuit
		\item chapitres non numérotés
	\end{itemize}
\end{onlyenv}
\end{frame}

\subsection{Table des matières et renvois}

% Table des matières
\begin{frame}[fragile,c]{Table des matières}
	
	\begin{itemize}
		\item La table des matières est produite automatiquement avec \lstinline|\tableofcontents|.
		\item Requiert \textbf{plusieurs} compilations.
		\item Les sections non numérotées ne sont pas incluses.
		\item Avec le \emph{package} \textbf{hyperref}, \lstinline|\tableofcontents| produit également la table des matières du fichier .pdf.
		\pause
		\item La classe memoir fournit également \lstinline|\tableofcontents*| qui n’insère pas la table des matières dans la table des matières.
		\pause
		\item \textbackslash listoffigures produit la liste des figures.
		\item \textbackslash listoftables produit la liste des tableaux.
	\end{itemize}

\end{frame}

% Étiquettes et renvois automatiques
\begin{frame}[fragile]{Étiquettes et renvois automatiques}
	\framesubtitle{Parce que l'ordinateur le fera mieux que vous\ldots}
	\begin{onlyenv}<1>
		\begin{itemize}
			\item Ne \textbf{jamais} renvoyer manuellement à un numéro de section, d’équation, de tableau, etc.
			\item «Nommer» un élément avec \lstinline|\label|
			\item Faire référence par son nom avec \lstinline|\ref|
			\item Requiert 2 à 3 compilations
		\end{itemize}
	
\begin{codesource}
	\section{Définitions}
		\label{sec:definitions}
	
		Lorem ipsum dolor sit amet, consectetur adipiscing elit, 
		sed do eiusmod tempor incididunt ut labore et dolore magna aliqua. 
		Ut enim ad minim veniam, quis nostrud exercitation ullamco laboris 
		nisi ut aliquip ex ea commodo consequat.
	
	\section{Historique}
		Tel que vu à la section \ref{sec:definitions}...
\end{codesource}
	\end{onlyenv}
	\begin{onlyenv}<2>
		\begin{itemize}
			\item Le \emph{package} \textbf{hyperref} insère des hyperliens vers des renvois dans les fichiers .pdf.
			\item La commande \lstinline|\autoref{}| permet de:
				\begin{enumerate}
					\item nommer automatiquement le type de renvoi (section, équation, tableau, etc.);
					\item transformer en hyperlien le texte \textbf{et} le numéro de la référence.
\begin{codesource}
	Tel que vu à la \autoref{sec:definitions}...
\end{codesource}
				\end{enumerate}
			\item La commande \lstinline|\pageref{}| renvoie à la page de la référence.
			\item Le \emph{package} \textbf{amsmath} fournit la commande \lstinline|\eqref{}| pour
				référencer les équations.
		\end{itemize}
	\end{onlyenv}
\end{frame}
% Bibliographie
\scriptsize

\section{Bibliographie}

\subsection*{Pour les nostalgiques de l'odeur de l'encre}

\begin{frame}[c]{Bibliographie}
	\framesubtitle{Pour les nostalgiques de l'odeur de l'encre}
	\setbeamertemplate{bibliography item}[book]
		
	\begin{thebibliography}{99}		
		\bibitem[Kopka and Daly, 2004]{kopkadaly:2004}
			Kopka, Helmut et Patrick W. Daly (2004).
			\newblock Guide to \LaTeX, Fourth Edition,
			\newblock Addison-Wesley,
			\newblock ISBN 978-0-321-17385-0, 597 p.
		\bibitem[Mittelbach et al., 2004]{mittelbach:2004}
			Mittelbach, Frank \emph{et al.} (2004).
			\newblock The \LaTeX\ Companion, Second Edition,
			\newblock Addison-Wesley,
			\newblock ISBN 978-0201362992, 1120p.
		\bibitem[Goossens and Mittelbach, 2007]{goossens:2007}
			Goossens, Michel et Franck Mittelbach (2007).
			\newblock The \LaTeX\ Graphics Companion, Second Edition,
			\newblock Addison-Wesley,
			\newblock ISBN 978-0321508928, 976p.
	\end{thebibliography}

\end{frame}

\subsection*{Pour les consciencieux de la forêt boréale}

\begin{frame}[c]{Bibliographie}
	\framesubtitle{Pour les consciencieux de la forêt boréale}
	\setbeamertemplate{bibliography item}[online]
	
	\begin{onlyenv}<1>
		\begin{thebibliography}{99}
			\bibitem[Goulet, 2016]{goulet:2016}
				Goulet, Vincent (2016).
				\newblock formation-latex-ul -- Introductory \LaTeX\ course in French,
				\newblock Comprehensive \TeX\ Archive Network,
				\newblock Consulté le 22 février 2018 à \href{https://ctan.org/pkg/formation-latex-ul}{%
					https://ctan.org/pkg/formation-latex-ul}
			\bibitem[Lees-Miller, 2018]{leesmiller:2018}
				Lees-Miller, John D. (2018).
				\newblock Free \& Interactive Online Introduction to \LaTeX,
				\newblock Overleaf,
				\newblock Consulté le 22 février 2018 à \href{https://www.overleaf.com/latex/learn/free-online-introduction-to-latex-part-1}{%
					https://www.overleaf.com/latex/learn/free-online-introduction-to-latex-part-1}
			\bibitem[ShareLaTeX, 2018]{sharelatex:2018}
				Share\LaTeX\ Documentation,
				\newblock Share\LaTeX,
				\newblock Consulté le 22 février à \href{https://fr.sharelatex.com/learn/Main_Page}{%
					https://fr.sharelatex.com/learn/Main\_Page}			
		\end{thebibliography}
	\end{onlyenv}

	\begin{onlyenv}<2>
		\begin{thebibliography}{99}
			\bibitem[LaTeX Wikibook]{wikibook}
				\href{https://en.wikibooks.org/wiki/LaTeX}{\LaTeX\ WikiBook}
			\bibitem[ShareLaTeX]{sharelatex}
				\href{https://fr.sharelatex.com/learn}{Share\LaTeX\ Documentation}
			\bibitem[Stack Exchange]{stackex}
				\href{https://tex.stackexchange.com/}{\TeX\ - \LaTeX\ Stack Exchange}
			\bibitem[LaTeX Community]{latexcomm}
				\href{http://latex.org/forum/}{\LaTeX\ Community}
			\bibitem[CTAN]{ctan}
				\href{https://ctan.org/}{Comprehensive \TeX\ Archive Network}
			\bibitem[TeX FAQ]{texfaq}
				\href{http://www.tex.ac.uk/}{UK List of TEX Frequently Asked Questions}
			\bibitem{google}
				Google\ldots
		\end{thebibliography}
	\end{onlyenv}
\end{frame}

% Période de questions
\begin{frame}[c]{Période de questions}
\LARGE
\begin{block}{Documentation de la formation}
	\ttfamily\href{http://bit.ly/ltxhec2}{http://bit.ly/ltxhec2}
\end{block}
\begin{block}{Évaluation de la formation}
	\ttfamily\href{http://bit.ly/ltxsurvey2}{http://bit.ly/ltxsurvey2}
\end{block}
\begin{block}{Support \TeX nique}
	\ttfamily Benoit Hamel : <benoit.2.hamel@hec.ca>
\end{block}
\end{frame}

\end{document}
