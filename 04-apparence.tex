% Apparence du texte

\section{Apparence du texte}

\subsection{Polices de caractères}

% Polices de caractères
\begin{frame}{Polices de caractères}
	\begin{itemize}
		\item Par défaut, tous les documents \LaTeX\ utilisent la même police, \textrm{Computer Modern}.
		\item Privilégier les polices de grande qualité et très complètes (lettres accentuées, grand choix de symboles)
		\item Peu de polices sont adaptées pour les mathématiques : Palatino, Times, Lucida (\$) sont des choix sûrs
		\item Dans la classe \textbf{hecthese}, les paquetages mathptmx et mathpazo sont chargés par défaut afin	d’offrir les polices de caractères Times et Palatino.
	\end{itemize}
\end{frame}

% Changement d'attribut de la police
\begin{frame}{Changement d'attribut de la police}
	\begin{tabularx}{\textwidth}{XXX}
		\arrayrulecolor{grisPrimaire!40}\hline\hline
		\multicolumn{3}{l}{\textbf{familles}}	\\
		\hline
		\textrm{romain}						&	\textbackslash rmfamily		&	\textbackslash textrm\{\emph{texte}\}\\
		\texttt{largeur fixe}				&	\textbackslash ttfamily		&	\textbackslash texttt\{\emph{texte}\}\\
		sans empattements					&	\textbackslash sffamily		&	\textbackslash textsf\{\emph{texte}\}\\
		\hline
		\multicolumn{3}{l}{\textbf{formes}}	\\
		\hline
		droit								&	\textbackslash upshape		&	\textbackslash textup\{\emph{texte}\}\\
		\emph{italique}						&	\textbackslash itshape		&	\textbackslash textit\{\emph{texte}\}\\
		\textsl{penché}						&	\textbackslash slshape		&	\textbackslash textsl\{\emph{texte}\}\\
		\textrm{\textsc{petites capitales}}	&	\textbackslash scshape		&	\textbackslash textsc\{\emph{texte}\} \\
		\hline
		\multicolumn{3}{l}{\textbf{séries}}	\\
		\hline
		\textmd{moyen}						&	\textbackslash mdseries		&	\textbackslash textmd\{\emph{texte}\}	\\
		\textbf{gras}						&	\textbackslash bfseries		&	\textbackslash textbf\{\emph{texte}\}	\\
		\hline\hline
	\end{tabularx}

	\begin{picture}(0,0)
	\thicklines\color{bleuFonceSecondaire}
	\onslide<2>\put(38,-7){\dashbox{1}(40,64)[b]{\parbox{.25\textwidth}{\centering\textbf{s'applique à tout le texte qui suit}}}}
	\onslide<3>\put(89,-7){\dashbox{1}(40,64)[b]{\parbox{.25\textwidth}{\centering\textbf{s'applique au texte en argument}}}}
	\end{picture}
\end{frame}

% Italique
\begin{frame}[fragile,c]{Italique}
	 Lorsque l'italique est utilisé pour mettre l'\emph{emphase} sur une partie du texte, on privilégie
	 la commande sémantique suivante:
	 
\begin{codesource}
	\emph{texte}
\end{codesource}

	Les commandes \lstinline|\emph{texte}| peuvent être imbriquées une dans l'autre. Le texte mis en italique redevient droit et vice versa.
	
	\begin{columns}
		\begin{column}{.49\textwidth}
			\vspace{-5.5mm}
\begin{codesource}
	C'était un peu \emph{rough}	par 
	moments.
\end{codesource}
		\end{column}
		\begin{column}{.49\textwidth}
			C'était un peu \emph{rough} par moments.
		\end{column}
	\end{columns}

	\begin{columns}
		\begin{column}{.49\textwidth}
			\vspace{-5.5mm}
\begin{codesource}
	Il m'a dit: « \emph{Enough 
	\emph{poutine} for the week!»}
\end{codesource}
		\end{column}
		\begin{column}{.49\textwidth}
			Il m'a dit: « \emph{Enough \textup{poutine} for the week!»}
		\end{column}
	\end{columns}
	
\end{frame}

% Taille de la police
\begin{frame}{Taille de la police}
	\begin{tabularx}{\textwidth}{l|l}
		\arrayrulecolor{grisPrimaire!40}
		\textbf{Commandes standards} 	& 	\textbf{Rendu}	\\
		\hline
		\textbackslash tiny				&	{\tiny vraiment petit}	\\
		\textbackslash scriptsize		&	{\scriptsize encore plus petit}	\\
		\textbackslash footnotesize		&	{\footnotesize plus petit}	\\
		\textbackslash small			&	{\small petit}	\\
		\textbackslash normal			&	{\normalsize normal}	\\
		\textbackslash large			&	{\large grand}	\\
		\textbackslash Large			&	{\Large plus grand}	\\
		\textbackslash LARGE			&	{\LARGE encore plus grand}	\\
		\textbackslash huge				&	{\huge énorme}	\\
		\textbackslash Huge				&	{\Huge encore plus énorme}		
	\end{tabularx}
\end{frame}

\subsection{Disposition du texte}