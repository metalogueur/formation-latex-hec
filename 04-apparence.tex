% Apparence du texte

\section{Apparence du texte}

\subsection{Polices de caractères}

% Polices de caractères
\begin{frame}{Polices de caractères}
	\begin{itemize}
		\item Par défaut, tous les documents \LaTeX\ utilisent la même police, \textrm{Computer Modern}.
		\item Privilégier les polices de grande qualité et très complètes (lettres accentuées, grand choix de symboles)
		\item Peu de polices sont adaptées pour les mathématiques : Palatino, Times, Lucida (\$) sont des choix sûrs
		\item Dans la classe \textbf{hecthese}, les paquetages mathptmx et mathpazo sont chargés par défaut afin	d’offrir les polices de caractères Times et Palatino.
	\end{itemize}
\end{frame}

% Changement d'attribut de la police
\begin{frame}{Changement d'attribut de la police}
	\begin{tabularx}{\textwidth}{XXX}
		\arrayrulecolor{grisPrimaire!40}\hline\hline
		\multicolumn{3}{l}{\textbf{familles}}	\\
		\hline
		\textrm{romain}						&	\cmd{rmfamily}		&	\cmd{textrm\{<texte>\}}\\
		\texttt{largeur fixe}				&	\cmd{ttfamily}		&	\cmd{texttt\{<texte>\}}\\
		sans empattements					&	\cmd{sffamily}		&	\cmd{textsf\{<texte>\}}\\
		\hline
		\multicolumn{3}{l}{\textbf{formes}}	\\
		\hline
		droit								&	\cmd{upshape}		&	\cmd{textup\{<texte>\}}\\
		\emph{italique}						&	\cmd{itshape}		&	\cmd{textit\{<texte>\}}\\
		\textsl{penché}						&	\cmd{slshape}		&	\cmd{textsl\{<texte>\}}\\
		\textrm{\textsc{petites capitales}}	&	\cmd{scshape}		&	\cmd{textsc\{<texte>\}}\\
		\hline
		\multicolumn{3}{l}{\textbf{séries}}	\\
		\hline
		\textmd{moyen}						&	\cmd{mdseries}		&	\cmd{textmd\{<texte>\}}\\
		\textbf{gras}						&	\cmd{bfseries}		&	\cmd{textbf\{<texte>\}}\\
		\hline\hline
	\end{tabularx}

	\begin{picture}(0,0)
	\thicklines\color{bleuFonceSecondaire}
	\onslide<2>\put(38,-7){\dashbox{1}(40,64)[b]{\parbox{.25\textwidth}{\centering\textbf{s'applique à tout le texte qui suit}}}}
	\onslide<3>\put(92,-7){\dashbox{1}(40,64)[b]{\parbox{.25\textwidth}{\centering\textbf{s'applique au texte en argument}}}}
	\end{picture}
\end{frame}

% Italique
\begin{frame}[fragile,c]{Italique}
	 Lorsque l'italique est utilisé pour mettre l'\emph{emphase} sur une partie du texte, on privilégie
	 la commande sémantique suivante:
	 
\begin{codesource}
	\emph{texte}
\end{codesource}

	Les commandes \cmd{emph\{<texte>\}} peuvent être imbriquées une dans l'autre. Le texte mis en italique redevient droit et vice versa.
	
	\begin{columns}
		\begin{column}{.49\textwidth}
			\vspace{-5.5mm}
\begin{codesource}
	C'était un peu \emph{rough}	par 
	moments.
\end{codesource}
		\end{column}
		\begin{column}{.49\textwidth}
			C'était un peu \emph{rough} par moments.
		\end{column}
	\end{columns}

	\begin{columns}
		\begin{column}{.49\textwidth}
			\vspace{-5.5mm}
\begin{codesource}
	Il m'a dit: « \emph{Enough 
	\emph{poutine} for the week!»}
\end{codesource}
		\end{column}
		\begin{column}{.49\textwidth}
			Il m'a dit: « \emph{Enough \textup{poutine} for the week!»}
		\end{column}
	\end{columns}
	
\end{frame}

% Taille de la police
\begin{frame}{Taille de la police}
	\begin{tabularx}{\textwidth}{l|l}
		\arrayrulecolor{grisPrimaire!40}
		\textbf{Commandes standards} 	& 	\textbf{Rendu}	\\
		\hline
		\cmd{tiny}						&	{\tiny vraiment petit}	\\
		\cmd{scriptsize}				&	{\scriptsize encore plus petit}	\\
		\cmd{footnotesize}				&	{\footnotesize plus petit}	\\
		\cmd{small}						&	{\small petit}	\\
		\cmd{normalsize}				&	{\normalsize normal}	\\
		\cmd{large}						&	{\large grand}	\\
		\cmd{Large}						&	{\Large plus grand}	\\
		\cmd{LARGE}						&	{\LARGE encore plus grand}	\\
		\cmd{huge}						&	{\huge énorme}	\\
		\cmd{Huge}						&	{\Huge encore plus énorme}		
	\end{tabularx}
\end{frame}

\subsection{Disposition du texte}

% Listes
\begin{frame}[fragile]{Listes}
	\begin{itemize}
		\item Deux principales sortes de listes:
		\begin{enumerate}
			\item \textbf{à puce} avec l'environnement \cmd{itemize}
			\item \textbf{numérotée} avec l'environnement \cmd{enumerate}
		\end{enumerate}
		\item Possibilité de les imbriquer les unes dans les autres
		\item Marqueurs adaptés automatiquement jusqu'à quatre niveaux

		\pause
\begin{codesource}
	\begin{itemize}
		\item Deux principales sortes de listes:
		\begin{enumerate}
			\item \textbf{à puce} avec l'environnement \verb=itemize=
			\item \textbf{numérotée} avec l'environnement \verb=enumerate=
		\end{enumerate}
		\item Possibilité de les imbriquer les unes dans les autres
		\item Marqueurs adaptés automatiquement jusqu'à quatre niveaux
	\end{itemize}
\end{codesource}

		\pause
		\item Une troisième liste est disponible : \texttt{description}
	\end{itemize}
\end{frame}

% Citations
\begin{frame}[fragile,c]{Citations}
	\framesubtitle<1>{Citations courtes}
	\framesubtitle<2>{Citations longues}
	\begin{onlyenv}<1>
		On utilise l'environnement \texttt{quote} pour insérer une citation courte (un paragraphe)
		dans le texte.
	
		\begin{columns}
			\begin{column}{.49\textwidth}
				\vspace{-17mm}
\begin{codesource}
	\begin{quote}
		Life is what happens to you while 
		you're busy making other plans. 
		-- John Lennon
	\end{quote}
\end{codesource}
			\end{column}
		
			\begin{column}{.49\textwidth}
				\begin{quote}
					Life is what happens to you while you're busy making other plans. -- John Lennon
				\end{quote}
			\end{column}
		\end{columns}
	\end{onlyenv}

	\begin{onlyenv}<2>
		On utilise l'environnement \texttt{quotation} pour insérer une citation longue (plus
		d'un paragraphe)
		
		\begin{columns}
			\begin{column}{.49\textwidth}
				\vspace{-38mm}
\begin{codesource}
	\begin{quotation}
		I've missed more than 9000 shots in my 
		career. I've lost almost 300 games. 26 
		times I've been trusted to take the game 
		winning shot and missed.
		
		I've failed over and over and over again 
		in my life. And that is why I succeed. 
		-- Michael Jordan
	\end{quotation}
\end{codesource}	
			\end{column}
		
			\begin{column}{.49\textwidth}
\begin{quotation}
	I've missed more than 9000 shots in my career. 
	I've lost almost 300 games. 26 times I've been 
	trusted to take the game winning shot and missed.
	
	I've failed over and over and over again in my life. 
	And that is why I succeed. -- Michael Jordan
\end{quotation}
			\end{column}
		\end{columns}
	\end{onlyenv}
\end{frame}

% Notes de bas de page
\begin{frame}[fragile,c]{Notes de bas de page}
	\begin{itemize}
		\item Une note de bas de page est insérée avec la commande suivante:
\begin{codesource}
	\footnote{texte de la note}
\end{codesource}
		\item La commande doit suivre immédiatement le texte à annoter.
		\item Méthode recommandée :
\begin{codesource}
	fera remarquer que Pierre Lasou\footnote{%
	Spécialiste en ressources documentaires} %
	fut une grande aide dans la préparation de ...
\end{codesource}
		\item La numérotation et la disposition sont automatiques.
	\end{itemize}
\end{frame}

% Code source
\begin{frame}[fragile,c]{Code source}
	\begin{onlyenv}<1>
		\begin{itemize}
			\item Pour rédiger du code source en bloc, on utilise l'environnement \texttt{verbatim}
\begin{codesource}
	\begin{verbatim}
		Texte disposé tel qu'il est saisi
		dans une police à largeur fixe.
	\end{verbatim}
\end{codesource}
			\item Pour rédiger du code source à même le texte, on utilise la commande \cmd{verb}, dont la
			syntaxe est \cmd{verbcsourcec} où \emph{c} est un caractère quelconque ne se trouvant pas dans \emph{source}.
			\item Pour un usage plus intensif, consultez la documentation du \emph{package} \textbf{listings}.
		\end{itemize}
	\end{onlyenv}

	\begin{onlyenv}<2>
	Un exemple\footnote{tiré du site \href{http://r4stats.com/examples/programming/}{r4stats.com}.} :
\begin{codesource}
	# ---Writing Your Own Functions (Macros)---
	
	# A good function that just prints.
	mystats <- function(x) {
		print( mean(x, na.rm = TRUE) )
		print(   sd(x, na.rm = TRUE) )
	}
	mystats(myvar)
	
	# A function with vector output.
	mystats  <- function(x) {
		mymean <- mean(x, na.rm = TRUE)
		mysd   <-   sd(x, na.rm = TRUE)
		c(mean = mymean, sd = mysd )
	}
	mystats(myvar)
	myVector <- mystats(myvar)
	myVector
\end{codesource}
	\end{onlyenv}
\end{frame}