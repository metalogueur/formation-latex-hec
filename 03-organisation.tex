% Organisation d'un document

\section{Document Organization}

\subsection{Parts of a Document}

% Choix d'une classe
\begin{frame}[c]{Class Choice}
	The first thing you need to do when writing a \LaTeX\ document is to choose a document class.
	
	\begin{table}[fragile,c]
		\begin{tabularx}{\textwidth}{lllll}
			\arrayrulecolor{grisPrimaire!40}\hline\hline
			\textbf{Class} & \textbf{Divisions} & \textbf{Organization} & \textbf{Header} &	\textbf{Footer} \\
			\hline
			\texttt{article}			&	\makecell{parts, sections,\\ \ldots}				&	one-sided		&	empty			&	centered page number \\
			\texttt{report}				&	\makecell{parts, chapters, \\ sections, \ldots}	&	one-sided		&	empty			&	centered page number \\
			\texttt{book}				&	\makecell{parts, chapters, \\ sections, \ldots}	&	two-sided	&
			\makecell{page numbers, \\ titles}	&	empty \\
			\texttt{hecthese}	&	\makecell{chapters, sections, \\ subsections}		&	two-sided	&
			empty			&	centered page number \\
			\hline\hline
		\end{tabularx}
	\end{table}
\end{frame}

% Titre et page de titre
\begin{frame}[fragile]{Titles and Title Page}
	Automatic layout:
\begin{codesource}
	% Preamble commands
	\title[short title]{long title}
	\author[short author name]{long author name}
	\date[short date]{long date}
	[...]
	
	% Document body command
	\maketitle
\end{codesource}

	Manual layout:
		\begin{columns}
			\begin{HECcomparaison}{Standard classes}
\begin{codesource}
	\begin{titlepage}
		...
	\end{titlepage}
\end{codesource}
			\end{HECcomparaison}
			\begin{HECcomparaison}{memoir and hecthese classes}
\begin{codesource}
	\begin{titlingpage}
		...
	\end{titlingpage}
\end{codesource}	
			\end{HECcomparaison}
		\end{columns}
	
	In the \textbf{hecthese} document class, title pages are automatically generated.
\end{frame}

% Résumé
\begin{frame}[fragile,c]{Abstract}
	\begin{itemize}
		\item \textbf{article}, \textbf{report} or \textbf{memoir} classes : abstract generated with the
		\lstinline|abstract| environment
\begin{codesource}
	\begin{abstract}
		...
	\end{abstract}
\end{codesource}

		\item \textbf{hecthese} class: french and english abstracts treated as normal, unnumbered chapters
	\end{itemize}
\end{frame}

% Sections
\begin{frame}[fragile]{Sections}
	\begin{itemize}
		\item The document is subdivided with the following commands:
\begin{codesource}
	\part[short title]{long title}
	\chapter[short title]{long title}
	\section[short title]{long title}
	\subsection[short title]{long title}
	
	\subsubsection[short title]{long title} 	% avoid using in books
	
	\paragraph[short title]{long title} 		% evil! never use!
	\subparagraph[short title]{long title} 	% EVIL! never EVER use!
\end{codesource}

		\item Automatic numbering
		\item Commands followed by an * = unnumbered section 
		\item Short title as an optional argument
	\end{itemize}
\end{frame}

% Annexes
\begin{frame}[fragile,c]{Appendices}
	\begin{itemize}
		\item Appendices are sections and chapters with an alphanumeric numbering (A,
		A.1, \ldots).
		\item Sections following the \cmd{appendix} command are all considered appendices.
		\item In the title, ``Chapter'' is changed into ``Appendix''.
	\end{itemize}
\end{frame}

% Structure logique d'un livre
\begin{frame}[fragile]{A Book's Logical Structure}
	\framesubtitle{book, memoir, hecthese classes}

\begin{onlyenv}<1>
\begin{codesource}
	\frontmatter
\end{codesource}	
	\begin{itemize}
		\item preface, table of contents, etc.
		\item roman page numbering (i, ii, \ldots)
		\item unnumbered chapters 
	\end{itemize}
\begin{codesource}
	\mainmatter
\end{codesource}	
	\begin{itemize}
		\item the book's content
		\item arabic page numbering, starting at 1
		\item numbered chapters
	\end{itemize}
\end{onlyenv}

\begin{onlyenv}<2>
\begin{codesource}
	\backmatter
\end{codesource}
	\begin{itemize}
		\item everything else (bibliography, index, etc.)
		\item the page numbering continues
		\item unnumbered chapters
	\end{itemize}
\end{onlyenv}
\end{frame}

\subsection{Table of Contents and Referencing}

% Table des matières
\begin{frame}[fragile,c]{Table of Contents}
	
	\begin{itemize}
		\item The table of contents is automatically generated with \cmd{tableofcontents}.
		\item Needs \textbf{more than one} compilation to be generated.
		\item Unnumbered sections are not included.
		\item With the \textbf{hyperref} package, \cmd{tableofcontents} generates the .pdf file's
		table of contents.
		\pause
		\item \cmd{tableofcontents*}, from the memoir document class, doesn't include the table of contents in the table of contents.
		\pause
		\item \cmd{listoffigures} generates the list of figures.
		\item \cmd{listoftables} generates the list of tables.
	\end{itemize}

\end{frame}

% Étiquettes et renvois automatiques
\begin{frame}[fragile]{Labels and Automatic Referencing}
	\framesubtitle{Because your computer will do it better than you\ldots}
	\begin{onlyenv}<1>
		\begin{itemize}
			\item \textbf{Never} refer manually to a section, an equation, a table, etc.
			\item ``Name'' an element with \cmd{label}
			\item Refer to that label with \cmd{ref}
			\item Needs 2 to 3 compilations to generate
		\end{itemize}
	
\begin{codesource}
	\section{Definitions}
		\label{sec:definitions}
	
		Lorem ipsum dolor sit amet, consectetur adipiscing elit, 
		sed do eiusmod tempor incididunt ut labore et dolore magna aliqua. 
		Ut enim ad minim veniam, quis nostrud exercitation ullamco laboris 
		nisi ut aliquip ex ea commodo consequat.
	
	\section{History}
		As seen in Section \ref{sec:definitions}...
\end{codesource}
	\end{onlyenv}
	\begin{onlyenv}<2>
		\begin{itemize}
			\item The \textbf{hyperref} package generates hyperlinks to the references in the .pdf files.
			\item The \cmd{autoref\{\}} command\ldots
				\begin{enumerate}
					\item automatically identifies the reference type (section, equation, table, etc.);
					\item generates a hyperlink with the text \textbf{and} number of the reference.
\begin{codesource}
	As seen in \autoref{sec:definitions}...
\end{codesource}
				\end{enumerate}
			\item The \cmd{pageref\{\}} command refers to a page number.
			\item The  \textbf{amsmath} package provides the \cmd{eqref\{\}} command to refer to equations.
		\end{itemize}
	\end{onlyenv}
\end{frame}