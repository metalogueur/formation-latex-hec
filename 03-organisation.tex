% Organisation d'un document

\section{Organisation d'un document}

\subsection{Parties d'un document}

\begin{frame}[c]{Choix d'une classe}
	La première chose que l'on doit faire lorsqu'on débute la rédaction d'un document \LaTeX,
	c'est de choisir une classe de document.
	
	\begin{table}[c]
		\begin{tabularx}{\textwidth}{lllll}
			\hline\hline
			\textbf{Classe} & \textbf{Divisions} & \textbf{Disposition} & \textbf{Entête} &	\textbf{Pied de page} \\
			\hline
			\texttt{article}			&	parties, sections, \ldots				&	recto		&	vide			&	folio centré \\
			\texttt{report}				&	parties, chapitres, sections, \ldots	&	recto		&	vide			&	folio centré \\
			\texttt{book}				&	parties, chapitres, sections, \ldots	&	recto verso	&
			folio, titres	&	vide \\
			\texttt{hecthese}	&	chapitres, sections, sous-sections		&	recto verso	&
			vide			&	folio centré \\
			\hline\hline
		\end{tabularx}
	\end{table}
\end{frame}

\subsection{Table des matières et renvois}