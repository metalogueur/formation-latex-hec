\section{Text layout}

% Alignement du texte
\begin{frame}[fragile]{Text alignment}
	\begin{itemize}
		\item By default, text is fully justified.
		\item To align text to the left, you use the \texttt{flushleft} environment.
\begin{codesource}
\begin{flushleft}
	Text will be aligned to the left.
\end{flushleft}
\end{codesource}
		\item You use the \texttt{center} environment to center text.
\begin{codesource}
	\begin{center}
		Text will be centered.
	\end{center}
\end{codesource}
		\item To align text to the right, you use the \texttt{flushright} environment.
\begin{codesource}
	\begin{flushright}
		Text will be aligned to the right.
	\end{flushright}
\end{codesource}
	\end{itemize}
\end{frame}

% Listes
\begin{frame}[fragile]{Lists}
	\framesubtitle{Unnumbered and numbered lists}
	\begin{itemize}
		\item Unnumbered lists are built using the \texttt{itemize} environment.
\begin{codesource}
	\begin{itemize}
		\item First item
		\item Second item
		\item etc.
	\end{itemize}
\end{codesource}
		\item Numbered lists are built using the \texttt{enumerate} environment.
\begin{codesource}
	\begin{enumerate}
		\item First item
		\item Second item
		\item etc.
	\end{enumerate}
\end{codesource}
		\item The \cmd{item} command is used to list items.
		\item You can embed lists up to four levels.
	\end{itemize}
\end{frame}

\begin{frame}[c, fragile]{Lists}
	\framesubtitle{Definition lists}
	
	You create definition lists with the \texttt{description} environment.
	
\begin{codesource}
	\begin{description}
		\item[First expression] First expression's definition
		\item[Second expression] Second expression's definition
	\end{description}
\end{codesource}

	\begin{description}
		\item[First expression] First expression's definition. Auctor est gravida habitasse leo lobortis mollis nec platea posuere
		 sollicitudin tempus.
		\item[Second expression] Second expression's definition. Aenean consequat dictumst dignissim duis facilisis himenaeos id
		 pharetra placerat porta posuere primis senectus tortor.
	\end{description}
\end{frame}

% Citations
\begin{frame}[fragile,c]{Quotations}
\framesubtitle<1>{Short quotations}
\framesubtitle<2>{Long quotations}
\begin{onlyenv}<1>
	You use the \texttt{quote} environment to insert short quotations (one paragraph)
	in the text.
	
	\begin{columns}
		\begin{column}{.49\textwidth}
			\vspace{-17mm}
\begin{codesource}
	\begin{quote}
		Life is what happens to you while 
		you're busy making other plans. 
		-- John Lennon
	\end{quote}
\end{codesource}
		\end{column}
		
		\begin{column}{.49\textwidth}
			\begin{quote}
				Life is what happens to you while you're busy making other plans. -- John Lennon
			\end{quote}
		\end{column}
	\end{columns}
\end{onlyenv}

\begin{onlyenv}<2>
	You use the \texttt{quotation} environment to insert long quotations (more than one paragraph)
	in the text.
	
	\begin{columns}
		\begin{column}{.49\textwidth}
			\vspace{-38mm}
\begin{codesource}
	\begin{quotation}
		I've missed more than 9000 shots in my 
		career. I've lost almost 300 games. 26 
		times I've been trusted to take the game 
		winning shot and missed.
		
		I've failed over and over and over again 
		in my life. And that is why I succeed. 
		-- Michael Jordan
	\end{quotation}
\end{codesource}	
		\end{column}
		
		\begin{column}{.49\textwidth}
			\begin{quotation}
				I've missed more than 9000 shots in my career. 
				I've lost almost 300 games. 26 times I've been 
				trusted to take the game winning shot and missed.
				
				I've failed over and over and over again in my life. 
				And that is why I succeed. -- Michael Jordan
			\end{quotation}
		\end{column}
	\end{columns}
\end{onlyenv}
\end{frame}

% Notes de bas de page
\begin{frame}[fragile,c]{Footnotes}
	\begin{itemize}
		\item You insert footnotes with the following command:
\begin{codesource}
	\footnote{footnote text}
\end{codesource}
		\item The command must follow the text that has to be annotated.
		\item Recommended method :
\begin{codesource}
	... fera remarquer que Pierre Lasou\footnote{%
		Spécialiste en ressources documentaires} %
	fut une grande aide dans la préparation de ...
\end{codesource}
		\item Footnote numbering and layout are automatic.
	\end{itemize}
\end{frame}

% Code source
\begin{frame}[fragile,c]{Source code}
	\begin{onlyenv}<1>
		\begin{itemize}
			\item To write source code in blocks, you use the \texttt{verbatim} environment.
\begin{codesource}
	\begin{verbatim}
		Text laid out as is with a
		fixed-width font.
	\end{verbatim}
\end{codesource}
			\item To write source code inside text, you use the \cmd{verb} command. Its syntax
			is \cmd{verbcsourcec} where \emph{c} is a character not used in \emph{source}.
\begin{codesource}
	Text with \verb|some code|.
\end{codesource}
			\item For a more intensive use, please read the \textbf{listings} package's documentation.
		\end{itemize}
	\end{onlyenv}

\begin{onlyenv}<2>
	Example\footnote{taken from the \href{http://r4stats.com/examples/programming/}{r4stats.com} website.} :
\begin{codesource}
# ---Writing Your Own Functions (Macros)---

# A good function that just prints.
mystats <- function(x) {
	print( mean(x, na.rm = TRUE) )
	print(   sd(x, na.rm = TRUE) )
}
mystats(myvar)

# A function with vector output.
mystats  <- function(x) {
	mymean <- mean(x, na.rm = TRUE)
	mysd   <-   sd(x, na.rm = TRUE)
	c(mean = mymean, sd = mysd )
}
mystats(myvar)
myVector <- mystats(myvar)
myVector
\end{codesource}
\end{onlyenv}
\end{frame}