\section{Apparence du texte}

% --------------------- %
% Polices de caractères %
% --------------------- %

\begin{frame}[c]

	\frametitle{Polices de caractères}
	
	\begin{itemize}
		\item Par défaut, tous les documents {\LaTeX} utilisent la même police,
			\textrm{Computer Modern}
		\item Privilégier les polices de grande qualité et très complètes (lettres accentuées, grand 	choix de symboles)
		\item Peu de polices sont adaptées pour les mathématiques
			\begin{itemize}
				\item Palatino, Times, Lucida (\$) sont des choix sûrs
			\end{itemize}
		\item Dans la classe hecthese, les paquetages mathptmx et mathpazo sont chargés par défaut
			afin d'offrir des polices de caractères Times et Palatino.
	\end{itemize}
\end{frame}

% ---------------------------------- %
% Changement d'attribut de la police %
% ---------------------------------- %

\begin{frame}[c]

	\frametitle{Changement d'attribut de la police}
	
	\begin{center}
		\begin{tabular}{lcc}
			\hline\hline		
			\textbf{famille}			&										&	\\
			\hline
			\textrm{romain}				&	\texttt{\textbackslash rmfamily}	&	\texttt{\textbackslash textrm\{}\emph{texte}\texttt{\}} \\
			\texttt{largeur fixe}		&	\texttt{\textbackslash ttfamily}	&	\texttt{\textbackslash texttt\{}\emph{texte}\texttt{\}} \\
			\textsf{sans empattements}	&	\texttt{\textbackslash sffamily}	&	\texttt{\textbackslash textsf\{}\emph{texte}\texttt{\}} \\
			\hline
			\textbf{forme}				&										&	\\
			\hline
			\textup{droit}				&	\texttt{\textbackslash upshape}		&	\texttt{\textbackslash textup\{}\emph{texte}\texttt{\}} \\
			\textit{italique}			&	\texttt{\textbackslash itshape}		&	\texttt{\textbackslash textit\{}\emph{texte}\texttt{\}} \\
			\textsl{penché}				&	\texttt{\textbackslash slshape}		&	\texttt{\textbackslash textsl\{}\emph{texte}\texttt{\}} \\
			\textsc{petites capitales}	&	\texttt{\textbackslash scshape}		&	\texttt{\textbackslash textsc\{}\emph{texte}\texttt{\}} \\
			\hline
			\textbf{série}				&										&	\\
			\hline
			\textmd{moyen}				&	\texttt{\textbackslash mdseries}	&	\texttt{\textbackslash textmd\{}\emph{texte}\texttt{\}} \\
			\textbf{gras}				&	\texttt{\textbackslash bfseries}	&	\texttt{\textbackslash textbf\{}\emph{texte}\texttt{\}} \\
			\hline\hline
		\end{tabular}
	\end{center}
\end{frame}

% ------------------- %
% Taille de la police %
% ------------------- %

\begin{frame}

	\frametitle{Taille de la police}
		
	\begin{center}
		\begin{tabular}{ll}
			\hline\hline
			\textbf{commandes standards}	&									\\
			\hline
			\textbackslash tiny				&	{\tiny vraiment petit} 			\\
			\textbackslash scriptsize		&	{\scriptsize encore plus petit}	\\
			\textbackslash footnotesize		&	{\footnotesize plus petit}		\\
			\textbackslash small			&	{\small petit}					\\
			\textbackslash normalsize		&	{\normalsize normal}			\\
			\textbackslash large			&	{\large grand}					\\
			\textbackslash Large			&	{\Large plus grand}				\\
			\textbackslash LARGE			&	{\LARGE encore plus grand}		\\
			\textbackslash huge				&	{\huge énorme} 					\\
			\textbackslash Huge				&	{\Huge encore plus énorme}		\\
			\hline\hline
		\end{tabular}
	\end{center}
\end{frame}

% -------- %
% Italique %
% -------- %

\begin{frame}[fragile]

	\frametitle{Italique}
	
	\begin{itemize}
		\item Une des propriétés \emph{les plus utilisées} dans le texte
		\item Commande sémantique : \\
		\lstinline|\emph{texte}|
		
		\pause
		
		\item Par défaut : texte en italique dans texte droit et vice versa
	\end{itemize}
\end{frame}