% version-2
\section{Apparence du texte}

% --------------------- %
% Polices de caractères %
% --------------------- %

\begin{frame}[c]

	\frametitle{Polices de caractères}
	
	\begin{itemize}
		\item Par défaut, tous les documents {\LaTeX} utilisent la même police,
			\textrm{Computer Modern}
		\item Privilégier les polices de grande qualité et très complètes (lettres accentuées, grand 	choix de symboles)
		\item Peu de polices sont adaptées pour les mathématiques
			\begin{itemize}
				\item Palatino, Times, Lucida (\$) sont des choix sûrs
			\end{itemize}
		\item Dans la classe hecthese, les paquetages \texttt{mathptmx} et \texttt{mathpazo} sont chargés par défaut afin d'offrir des polices de caractères Times et Palatino.
	\end{itemize}
\end{frame}

% ---------------------------------- %
% Changement d'attribut de la police %
% ---------------------------------- %

\begin{frame}[c]

	\frametitle{Changement d'attribut de la police}
	
		\begin{tabularx}{\textwidth}{lYY}
			\hline\hline		
			\textbf{famille}					&										&	\\
			\hline
			\textrm{romain}						&	\texttt{\textbackslash rmfamily}	&	\texttt{\textbackslash textrm\{}\emph{texte}\texttt{\}} \\
			\texttt{largeur fixe}				&	\texttt{\textbackslash ttfamily}	&	\texttt{\textbackslash texttt\{}\emph{texte}\texttt{\}} \\
			\textsf{sans empattements}			&	\texttt{\textbackslash sffamily}	&	\texttt{\textbackslash textsf\{}\emph{texte}\texttt{\}} \\
			\hline
			\textbf{forme}						&										&	\\
			\hline
			\textup{droit}						&	\texttt{\textbackslash upshape}		&	\texttt{\textbackslash textup\{}\emph{texte}\texttt{\}} \\
			\textit{italique}					&	\texttt{\textbackslash itshape}		&	\texttt{\textbackslash textit\{}\emph{texte}\texttt{\}} \\
			\textsl{penché}						&	\texttt{\textbackslash slshape}		&	\texttt{\textbackslash textsl\{}\emph{texte}\texttt{\}} \\
			\textrm{\textsc{petites capitales}}	&	\texttt{\textbackslash scshape}		&	\texttt{\textbackslash textsc\{}\emph{texte}\texttt{\}} \\
			\hline
			\textbf{série}						&										&	\\
			\hline
			\textmd{moyen}						&	\texttt{\textbackslash mdseries}	&	\texttt{\textbackslash textmd\{}\emph{texte}\texttt{\}} \\
			\textbf{gras}						&	\texttt{\textbackslash bfseries}	&	\texttt{\textbackslash textbf\{}\emph{texte}\texttt{\}} \\
			\hline\hline
		\end{tabularx}
\end{frame}

% ------------------- %
% Taille de la police %
% ------------------- %

\begin{frame}

	\frametitle{Taille de la police}
		
		\begin{tabularx}{\textwidth}{lY}
			\hline\hline
			\textbf{commandes standards}	&									\\
			\hline
			\textbackslash tiny				&	{\tiny vraiment petit} 			\\
			\textbackslash scriptsize		&	{\scriptsize encore plus petit}	\\
			\textbackslash footnotesize		&	{\footnotesize plus petit}		\\
			\textbackslash small			&	{\small petit}					\\
			\textbackslash normalsize		&	{\normalsize normal}			\\
			\textbackslash large			&	{\large grand}					\\
			\textbackslash Large			&	{\Large plus grand}				\\
			\textbackslash LARGE			&	{\LARGE encore plus grand}		\\
			\textbackslash huge				&	{\huge énorme} 					\\
			\textbackslash Huge				&	{\Huge encore plus énorme}		\\
			\hline\hline
		\end{tabularx}
\end{frame}

% -------- %
% Italique %
% -------- %

\begin{frame}[fragile]

	\frametitle{Italique}
	
	\begin{itemize}
		\item Commande sémantique : \lstinline|\emph{texte}|
		
		\pause
		
		\item Par défaut : texte en italique dans texte droit et vice versa
		\begin{columns}
			\column{.4\textwidth}
			\vspace{-1.4em}
	\begin{codesource}
	C'était un peu \emph{rough}
	par moments	
	\end{codesource}
			\column{.4\textwidth}
				C'était un peu \emph{rough} par moments
		\end{columns}
		
		\begin{columns}
			\column{.4\textwidth}
			\vspace{-1.4em}
	\begin{codesource}
	Il m'a dit: «\emph{Enough
	\emph{poutine} for the week!}»	
	\end{codesource} 
			\column{.4\textwidth}
				Il m’a dit : « \emph{Enough {\em poutine} for the week !} »
		\end{columns}
	
		\pause
		
		\item Pas de commande pour souligner en {\LaTeX}\dots
	\end{itemize}
\end{frame}

% -------- %
% Exercice %
% -------- %

\begin{frame}[fragile,c]

	\frametitle{Exercice \thenoExercice}
	
	À partir de votre document \texttt{.tex}, faites les étapes suivantes. À la fin de chaque étape,
	compilez votre document et observez les résultats.
	
	\begin{enumerate}
		\item Agrémentez votre texte de différentes familles, formes et séries de polices de caractères.
		\item Changez la taille de police à différents endroits.
		\item Mettez une partie de texte en italique et imbriquez plusieurs commandes 
			\lstinline|\emph{}| pour mettre l'emphase sur d'autres parties.
	\end{enumerate}
\end{frame}
\stepcounter{noExercice}

% ------ %
% Listes %
% ------ %

\begin{frame}[fragile]

	\frametitle{Listes}
	
	\begin{itemize}
		\item Deux principales sortes de listes :
		\begin{enumerate}
			\item \textbf{à puce} avec environnement \texttt{itemize}
			\item \textbf{numérotée} avec environnement \texttt{enumerate}
		\end{enumerate}
		\item Possible de les imbriquer les unes dans les autres
		\item Marqueurs adaptés automatiquement jusqu'à 4 niveaux
	\end{itemize}

	\pause
	
	\begin{codesource}
	\begin{itemize}
		\item Deux principales sortes de listes:
		\begin{enumerate}
			\item à puce avec environnement \verb=itemize=
			\item numérotée avec environnement \verb=enumerate=
		\end{enumerate}
		\item Possible de les imbriquer les unes
			dans les autres
		\item Marqueurs adaptés automatiquement jusqu'à 4 niveaux
	\end{itemize}
	\end{codesource}

\end{frame}

% -------------------- %
% Notes de bas de page %
% -------------------- %

\begin{frame}[fragile]

	\frametitle{Notes de bas de page}
	
	\begin{itemize}
		\item Note de bas de page insérée avec la commande \lstinline|\footnote{texte de la note}|
			
		\item Commande doit suivre immédiatement le texte à annoter
		
		\item Méthode recommandée : 
		\begin{codesource}
	... fera remarquer que Pierre Lasou\footnote{%
	Spécialiste en ressources documentaires} %
	fut d'une grande aide dans la préparation de ...
		\end{codesource}
				
		\item Numérotation et disposition automatiques
	\end{itemize}
\end{frame}

% ----------- %
% Code source %
% ----------- %

\begin{frame}[fragile,c]

	\frametitle{Code source}
	
	\begin{itemize}
		\item Environnement \texttt{verbatim}
		\begin{codesource}
	\begin{verbatim}
	Texte disposé exactement tel qu'il est tapé
	dans une police à largeur fixe
	\end{verbatim}
		\end{codesource}
	
		\item Commande \texttt{\textbackslash verb} dont la syntaxe est \lstinline|\verbcsourcec| 
			où \textit{c} est un caractère quelconque ne se trouvant pas dans \textit{source}
			
		\item Pour usage plus intensif, voir le paquetage \textbf{listings}
	\end{itemize}
\end{frame}

% -------- %
% Exercice %
% -------- %

\begin{frame}[c]

	\frametitle{Exercice \thenoExercice}
	
	\begin{enumerate}
		\item Créez une liste à puces et une liste numérotée avec quelques éléments à l'intérieur.
		\item Créez quelques notes de bas de page.
		\item Compilez votre document et observez le résultat.
	\end{enumerate}
\end{frame}
\stepcounter{noExercice}