% version-2
\section{Classe de document hecthese}

% ------------------------------------- %
% Un document conforme en un tournemain %
% ------------------------------------- %

\begin{frame}

	\frametitle{Un document conforme en un tournemain}
	
	\begin{itemize}
		\item \textbf{hecthese} livrée dans \TeX Live donc déjà installée sur votre ordinateur
		\item Mise en page conforme aux règles de présentation du
			\href{http://www.hec.ca/qualitecomm/caf/guide-redaction-travail-cycles.pdf}{%
				Guide pour la rédaction d'un travail de 1er, 2e ou 3e cycles}
		\item Basée sur la classe \textbf{memoir}
		\item Quelques nouvelles commandes pour la création de la page de titre
		\item De nouveaux environnements adaptés
		\item Partir d’un gabarit (disponibles après l'installation dans un répertoire de travail de 
			la classe)
		\item Utiliser des fichiers séparés pour chaque chapitre de la thèse ou	du mémoire
	\end{itemize}
\end{frame}

% ------------ %
% Et la suite? %
% ------------ %

\begin{frame}[c]

	\frametitle{Et la suite?}
	
	\begin{itemize}
		\item Goulet, V. (2016). \textit{Rédaction avec {\LaTeX}}, Édition 2016.11-3, Université Laval, 174~p. Accès : \href{http://mirrors.ctan.org/info/formation-latex-ul/doc/formation-latex-ul.pdf}{%
		http://mirrors.ctan.org/info/formation-latex-ul/doc/formation-latex-ul.pdf} (Consulté le 21 décembre 2017)
		\item Kopka, H. et Patrick W. Daly (2004). \textit{Guide to {\LaTeX}}, Fourth Edition, Addison-Wesley, 597~p.
		\item Mittelbach F. \textit{et al.} (2004). \textit{The {\LaTeX} Companion}, Second Edition, Addison-Wesley, 1119~p.
	\end{itemize}
\end{frame}