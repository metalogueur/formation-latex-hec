% version-2
\section{Classe de document hecthese}

% ------------------------------------- %
% Un document conforme en un tournemain %
% ------------------------------------- %

\begin{frame}

	\frametitle{Un document conforme en un tournemain}
	
	\begin{itemize}
		\item \textbf{hecthese} livrée dans \TeX Live donc déjà installée sur votre ordinateur
		\item Mise en page conforme aux règles de présentation du
			\href{http://www.hec.ca/qualitecomm/caf/guide-redaction-travail-cycles.pdf}{%
				Guide pour la rédaction d'un travail de 1er, 2e ou 3e cycles}
		\item Basée sur la classe \textbf{memoir}
		\item Quelques nouvelles commandes pour la création de la page de titre et plus\ldots
		\item De nouveaux environnements adaptés
		\item Partir d’un gabarit (disponibles après l'installation de la classe dans un répertoire de travail).
		\item Utiliser des fichiers séparés pour chaque chapitre de la thèse ou	du mémoire.
	\end{itemize}
\end{frame}

% ------------ %
% Et la suite? %
% ------------ %

\begin{frame}

	\frametitle{Et la suite?}
	
	\begin{onlyenv}<1>
		Pour les nostalgiques de l'odeur de l'encre\dots
		
		\begin{thebibliography}{99}
			\setbeamertemplate{bibliography item}[book]
			\bibitem{1} Goulet, V. (2016). \textit{Rédaction avec {\LaTeX}}, Édition 2016.11-3, Université Laval, 174~p. Accès : \href{http://mirrors.ctan.org/info/formation-latex-ul/doc/formation-latex-ul.pdf}{%
				http://mirrors.ctan.org/info/formation-latex-ul/doc/formation-latex-ul.pdf} (Consulté le 21 décembre 2017)
			\bibitem{2} Kopka, H. et Patrick W. Daly (2004). \textit{Guide to {\LaTeX}}, Fourth Edition, Addison-Wesley, 597~p.
			\bibitem{3} Mittelbach F. \textit{et al.} (2004). \textit{The {\LaTeX} Companion}, Second Edition, Addison-Wesley, 1119~p.
		\end{thebibliography}
	\end{onlyenv}

	\begin{onlyenv}<2>
		Pour les consciencieux de la forêt boréale...
		
		\begin{thebibliography}{99}
			\setbeamertemplate{bibliography item}[online]
			\bibitem{4} \href{https://en.wikibooks.org/wiki/LaTeX}{\LaTeX\ WikiBook}
			\bibitem{5} \href{https://tex.stackexchange.com/}{\TeX\ - \LaTeX\ Stack Exchange}
			\bibitem{6} \href{http://latex.org/forum/}{\LaTeX\ Community}
			\bibitem{7} \href{https://ctan.org/}{Comprehensive \TeX\ Archive Network}
			\bibitem{8} \href{http://www.tex.ac.uk/}{UK List of \TeX\ Frequently Asked Questions}
			\bibitem{9} Google\dots
		\end{thebibliography}
	\end{onlyenv}
\end{frame}

% -------- %
% Exercice %
% -------- %

\begin{frame}[c]

	\frametitle{Un dernier exercice\dots}
	
	\begin{enumerate}
		\item Rendez-vous sur le site de la formation : \href{http://bit.ly/latexhec}{http://bit.ly/latexhec}.
		\item Téléchargez l'ensemble des fichiers du projet.
		\item Compilez le fichier \texttt{formation-latex-hec.tex} et conservez la copie \texttt{.pdf} de la formation.
		\item \textbf{FACULTATIF MAIS SOUHAITABLE} : Analysez le code de la présentation.
	\end{enumerate}
\end{frame}