\documentclass[aspectratio=1610,compress,t,gabaritb,english,french]{hecppt}

%% Packages
\usepackage{fontawesome}
\usepackage{metalogo}
\usepackage{listings}
\usepackage{tabularx}
\usepackage{colortbl}
\usepackage{hyperref}

%% Commandes
\newcommand{\cmd}[1]{%
	\texttt{\textbackslash #1}
}

%% Environnements
\lstnewenvironment{codesource}{%
	\lstset{%
		basicstyle=\tiny,
		language=[LaTeX]TeX,
		backgroundcolor=\color{bleuPaleSecondaire!10},
		tabsize=2,
		% frame=leftline,
		% numbers=left,
		% numberstyle=\tiny,
		literate=%
		{à}{{\`a}}1
		{é}{{\'e}}1
		{ç}{{\c c}}1
		{«}{{\og}}1
		{»}{{\fg}}1
	}
}{}

%% Options des packages
\hypersetup{colorlinks=true,%
	urlcolor=bleuFoncePrimaire,%
	linkcolor=white,%
	pdfauthor=Benoit Hamel,%
	pdftitle=Rédaction avec LaTeX : Principes de base}
\frenchbsetup{og=«,fg=»}
\setlength{\parskip}{1ex}

%% Métadonnées du document

\title{Rédaction avec \\ \texttt{\textbackslash title}\{\textrm{\LaTeX}\} }
\subtitle{Notions avancées}
\HECauteur{Benoit Hamel}{Benoit Hamel}
\date[2018-02-28]{2018-02-28}
\subject{} % Sujet inséré dans les métadonnées du pdf
\keywords{} % Mots-clés insérés dans les métadonnées du pdf

\begin{document}

\pageTitre

% Pages liminaires

% Page titre
\begin{frame}
	Benoit Hamel \\
	Technicien en documentation, soutien technique \\
	Bibliothèque HEC Montréal
	\vfill
	{
		\Huge\bfseries
		Rédaction avec \\
		\texttt{\textbackslash title\{\textrm{\LaTeX}\}}
	}
	\vfill
	Édition HEC Montréal, revue et augmentée (version française)
\end{frame}

% Page de la licence
\begin{frame}
	\faCopyright\ 2016 Vincent Goulet pour la 
	\href{https://ctan.org/pkg/formation-latex-ul}{version originale}. La liste des sources qui ont 
	servi à l'élaboration de cette formation se trouve à la fin du présent document.
	
	\faCreativeCommons\ Cette création est mise à disposition selon le contrat 
	\href{http://creativecommons.org/licenses/by-sa/4.0/deed.fr}{%
	Attribution-Partage dans les mêmes conditions 4.0 International de Creative Commons}. 
	En vertu de ce contrat, vous êtes libre de :
	
	\begin{itemize}
		\item partager -- reproduire, distribuer et communiquer l’oeuvre;
		\item remixer -- adapter l’oeuvre;
		\item utiliser cette oeuvre à des fins commerciales.
	\end{itemize}

	Selon les conditions suivantes :
	
	\begin{itemize}
		\item Attribution -- Vous devez créditer l’oeuvre, intégrer un lien vers le contrat et indiquer si des modifications ont été effectuées à l’oeuvre. Vous devez indiquer ces informations par tous les moyens possibles, mais vous ne pouvez suggérer que l’Offrant vous soutient ou soutient la façon dont vous avez utilisé son oeuvre.
		\item Partage dans les mêmes conditions -- Dans le cas où vous modifiez, transformez ou créez à partir du matériel composant l’oeuvre
		originale, vous devez diffuser l’oeuvre modifiée dans les même conditions, c’est à dire avec le même contrat avec lequel l’oeuvre originale a été diffusée.
	\end{itemize}
\end{frame}

% Table des matières
\begin{frame}{Sommaire de la formation}
	\tableofcontents
\end{frame}
\section{Objets flottants}

\subsection{Tableaux}

% Tableaux (introduction)
\begin{frame}[fragile,c]{Les tableaux}
	\framesubtitle{Introduction}
	
	\begin{itemize}
		\item Construire des tableaux avec \LaTeX\ demande du doigté.
		\item Il n'existe pas une, pas deux, mais de multiples manières de construire des tableaux.
		\item \LaTeX\ fournit deux environnements de base : \texttt{tabular} et \texttt{tabular*}
	\end{itemize}

	\begin{columns}
		\begin{column}{.49\textwidth}
\begin{codesource}
	\begin{tabular}{colonnes}
		cellule1 & cellule2 & cellule3 \\
		cellule4 & cellule5 & cellule6 \\
		cellule7 & cellule8 & cellule9
	\end{tabular}
\end{codesource}
		\end{column}
		\begin{column}{.49\textwidth}
\begin{codesource}
	\begin{tabular*}{largeur}{colonnes}
		cellule1 & cellule2 & cellule3 \\
		cellule4 & cellule5 & cellule6 \\
		cellule7 & cellule8 & cellule9
	\end{tabular*}
\end{codesource}
		\end{column}
	\end{columns}

	\begin{itemize}
		\item Nous verrons également un troisième environnement, \texttt{tabularx}, fourni avec
		le \emph{package} du même nom.
		\item La syntaxe de \texttt{tabularx} est identique à celle de \texttt{tabular}.
	\end{itemize}
\end{frame}

% Tableaux (construction)
\begin{frame}[fragile]{Les tableaux}
	\framesubtitle{Construction}
	Reprenons les tableaux de la diapositive précédente:
	
	\begin{columns}
		\begin{column}{.49\textwidth}
			\begin{codesource}
				\begin{tabular}{colonnes}
					cellule1 & cellule2 & cellule3 \\
					cellule4 & cellule5 & cellule6 \\
					cellule7 & cellule8 & cellule9
				\end{tabular}
			\end{codesource}
		\end{column}
		\begin{column}{.49\textwidth}
			\begin{codesource}
				\begin{tabular*}{largeur}{colonnes}
					cellule1 & cellule2 & cellule3 \\
					cellule4 & cellule5 & cellule6 \\
					cellule7 & cellule8 & cellule9
				\end{tabular*}
			\end{codesource}
		\end{column}
	\end{columns}

	\begin{onlyenv}<2>
		\begin{itemize}
			\item On spécifie le \textbf{nombre de colonnes} et l'\textbf{alignement du texte} dans l'argument \texttt{colonnes}.
			\begin{itemize}
				\scriptsize
				\item Les arguments possibles sont \texttt{l} (\emph{left}), \texttt{c} (\emph{center}),
					et \texttt{r} (\emph{right}).
				\item On spécifie une colonne de largeur spécifique avec \texttt{p\{largeur\}}.
				\item \texttt{tabularx} accepte aussi l'argument \texttt{X}, qui ajuste la largeur de la
					colonne en fonction de la largeur du tableau.
				\item Le symbole \texttt{|} est utilisé pour insérer une ligne verticale entre des colonnes.
			\end{itemize}
		\end{itemize}
	\end{onlyenv}

	\begin{onlyenv}<3>
		\begin{itemize}
			\item La \textbf{largeur} d'un tableau dépend de l'environnement utilisé :
			\begin{itemize}
				\scriptsize
				\item \texttt{tabular} : largeur du tableau = largeur de son contenu;
				\item \texttt{tabular*} et \texttt{tabularx} : largeur déterminée par l'argument 
					\texttt{largeur}.
			\end{itemize}
		\end{itemize}
	\end{onlyenv}

	\begin{onlyenv}<4>
		\begin{itemize}
			\item On sépare chaque cellule d'une \textbf{ligne} avec le symbole \texttt{\&}.
			\item On termine une ligne avec \textbackslash\textbackslash, \textbf{à l'exception de la dernière ligne}.
			\item On insère une ligne horizontale pleine largeur entre deux lignes avec \cmd{hline}.
			\item La commande \cmd{multicolumn\{cols\}\{pos\}\{text\}} sert à «fusionner» les cellules d'une ligne.
			\begin{itemize}
				\scriptsize
				\item \texttt{cols} : étendue de la cellule en colonnes;
				\item \texttt{pos} : alignement du texte (\texttt{l},\texttt{c},\texttt{r});
				\item \texttt{text} : le contenu de la cellule.
			\end{itemize}
		\end{itemize}
	\end{onlyenv}
	
\end{frame}

% Tableaux (exemple concret)
\begin{frame}[fragile,c]{Les tableaux}
	\framesubtitle{Exemple concret}
\begin{codesource}
	\begin{tabularx}{\textwidth}{X|rrr|r|rrr}
		\textbf{\'{E}quipes}	&	\multicolumn{7}{c}{\textbf{Statistiques}} \\
		\hline\hline
		NFC North Team		&	W	&	L	&	T	&	PCT		&	PF	&	PA	&	Net Pts \\
		\hline
		Minnesota Vikings	&	13	&	3	&	0	&	.813	&	382	&	252	&	130 \\
		Detroit Lions		&	9	&	7	&	0	&	.563	&	410	&	376	&	34 \\
		Green Bay Packers	&	7	&	9	& 	0	&	.438	&	320	&	384	&	-64 \\
		Chicago Bears		&	5	&	11	&	0	&	.313	&	264	&	320	&	-56
	\end{tabularx}
\end{codesource}

	\begin{tabularx}{\textwidth}{X|rrr|r|rrr}
		\textbf{Équipes}	&	\multicolumn{7}{c}{\textbf{Statistiques}} \\
		\hline\hline
		NFC North Team		&	W	&	L	&	T	&	PCT		&	PF	&	PA	&	Net Pts \\
		\hline
		Minnesota Vikings	&	13	&	3	&	0	&	.813	&	382	&	252	&	130 \\
		Detroit Lions		&	9	&	7	&	0	&	.563	&	410	&	376	&	34 \\
		Green Bay Packers	&	7	&	9	& 	0	&	.438	&	320	&	384	&	-64 \\
		Chicago Bears		&	5	&	11	&	0	&	.313	&	264	&	320	&	-56
	\end{tabularx}
\end{frame}

\subsection{Figures}
% Bibliographie
\scriptsize

\section{Bibliographie}

\subsection*{Pour les nostalgiques de l'odeur de l'encre}

\begin{frame}[c]{Bibliographie}
	\framesubtitle{Pour les nostalgiques de l'odeur de l'encre}
	\setbeamertemplate{bibliography item}[book]
		
	\begin{thebibliography}{99}		
		\bibitem[Kopka and Daly, 2004]{kopkadaly:2004}
			Kopka, Helmut et Patrick W. Daly (2004).
			\newblock Guide to \LaTeX, Fourth Edition,
			\newblock Addison-Wesley,
			\newblock ISBN 978-0-321-17385-0, 597 p.
		\bibitem[Mittelbach et al., 2004]{mittelbach:2004}
			Mittelbach, Frank \emph{et al.} (2004).
			\newblock The \LaTeX\ Companion, Second Edition,
			\newblock Addison-Wesley,
			\newblock ISBN 978-0201362992, 1120p.
		\bibitem[Goossens and Mittelbach, 2007]{goossens:2007}
			Goossens, Michel et Franck Mittelbach (2007).
			\newblock The \LaTeX\ Graphics Companion, Second Edition,
			\newblock Addison-Wesley,
			\newblock ISBN 978-0321508928, 976p.
	\end{thebibliography}

\end{frame}

\subsection*{Pour les consciencieux de la forêt boréale}

\begin{frame}[c]{Bibliographie}
	\framesubtitle{Pour les consciencieux de la forêt boréale}
	\setbeamertemplate{bibliography item}[online]
	
	\begin{onlyenv}<1>
		\begin{thebibliography}{99}
			\bibitem[Goulet, 2016]{goulet:2016}
				Goulet, Vincent (2016).
				\newblock formation-latex-ul -- Introductory \LaTeX\ course in French,
				\newblock Comprehensive \TeX\ Archive Network,
				\newblock Consulté le 22 février 2018 à \href{https://ctan.org/pkg/formation-latex-ul}{%
					https://ctan.org/pkg/formation-latex-ul}
			\bibitem[Lees-Miller, 2018]{leesmiller:2018}
				Lees-Miller, John D. (2018).
				\newblock Free \& Interactive Online Introduction to \LaTeX,
				\newblock Overleaf,
				\newblock Consulté le 22 février 2018 à \href{https://www.overleaf.com/latex/learn/free-online-introduction-to-latex-part-1}{%
					https://www.overleaf.com/latex/learn/free-online-introduction-to-latex-part-1}
			\bibitem[ShareLaTeX, 2018]{sharelatex:2018}
				Share\LaTeX\ Documentation,
				\newblock Share\LaTeX,
				\newblock Consulté le 22 février à \href{https://fr.sharelatex.com/learn/Main_Page}{%
					https://fr.sharelatex.com/learn/Main\_Page}			
		\end{thebibliography}
	\end{onlyenv}

	\begin{onlyenv}<2>
		\begin{thebibliography}{99}
			\bibitem[LaTeX Wikibook]{wikibook}
				\href{https://en.wikibooks.org/wiki/LaTeX}{\LaTeX\ WikiBook}
			\bibitem[ShareLaTeX]{sharelatex}
				\href{https://fr.sharelatex.com/learn}{Share\LaTeX\ Documentation}
			\bibitem[Stack Exchange]{stackex}
				\href{https://tex.stackexchange.com/}{\TeX\ - \LaTeX\ Stack Exchange}
			\bibitem[LaTeX Community]{latexcomm}
				\href{http://latex.org/forum/}{\LaTeX\ Community}
			\bibitem[CTAN]{ctan}
				\href{https://ctan.org/}{Comprehensive \TeX\ Archive Network}
			\bibitem[TeX FAQ]{texfaq}
				\href{http://www.tex.ac.uk/}{UK List of TEX Frequently Asked Questions}
			\bibitem{google}
				Google\ldots
		\end{thebibliography}
	\end{onlyenv}
\end{frame}

% Période de questions
\begin{frame}[c]{Période de questions}
\LARGE
\begin{block}{Documentation de la formation}
	\ttfamily\href{http://bit.ly/ltxhec2}{http://bit.ly/ltxhec2}
\end{block}
\begin{block}{Évaluation de la formation}
	\ttfamily\href{http://bit.ly/ltxsurvey2}{http://bit.ly/ltxsurvey2}
\end{block}
\begin{block}{Support \TeX nique}
	\ttfamily Benoit Hamel : <benoit.2.hamel@hec.ca>
\end{block}
\end{frame}

\end{document}
