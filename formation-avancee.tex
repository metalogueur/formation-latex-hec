\documentclass[aspectratio=1610,compress,t,gabaritb,english,french]{hecppt}

%% Packages
\usepackage{fontawesome}
\usepackage{metalogo}
\usepackage{listings}
\usepackage{tabularx}
\usepackage{colortbl}
\usepackage{hyperref}

%% Commandes
\newcommand{\cmd}[1]{%
	\texttt{\textbackslash #1}
}

%% Environnements
\lstnewenvironment{codesource}{%
	\lstset{%
		basicstyle=\tiny,
		language=[LaTeX]TeX,
		backgroundcolor=\color{bleuPaleSecondaire!10},
		tabsize=2,
		% frame=leftline,
		% numbers=left,
		% numberstyle=\tiny,
		literate=%
		{à}{{\`a}}1
		{á}{{\'a}}1
		{é}{{\'e}}1
		{ç}{{\c c}}1
		{«}{{\og}}1
		{»}{{\fg}}1		
	}
}{}

%% Options des packages
\hypersetup{colorlinks=true,%
	urlcolor=bleuFoncePrimaire,%
	linkcolor=bleuFoncePrimaire,%
	pdfauthor=Benoit Hamel,%
	pdftitle=Rédaction avec LaTeX : Principes de base}
\frenchbsetup{og=«,fg=»}
\setlength{\parskip}{1ex}

\renewcommand{\frenchtablename}{Tableau}

%% Ajustement des espaces entre les éléments de la table des matières
\makeatletter
\patchcmd{\beamer@sectionintoc}
{\vfill}
{}
{}
{}
\makeatother 

%% Métadonnées du document

\title{Rédaction avec \\ \texttt{\textbackslash title}\{\textrm{\LaTeX}\} }
\subtitle{Notions avancées}
\HECauteur{Benoit Hamel}{Benoit Hamel}
\date[2018-02-28]{2018-02-28}
\subject{} % Sujet inséré dans les métadonnées du pdf
\keywords{} % Mots-clés insérés dans les métadonnées du pdf

\begin{document}

\pageTitre

% Pages liminaires
\scriptsize

% Page titre
\begin{frame}
	Benoit Hamel \\
	Technicien en documentation, soutien technique \\
	Bibliothèque HEC Montréal
	\vfill
	{
		\Huge\bfseries
		Rédaction avec \\
		\texttt{\textbackslash title\{\textrm{\LaTeX}\}}
	}
	\vfill
	Deuxième partie : notions avancées \\
	Édition HEC Montréal, revue et augmentée (version française)
\end{frame}

% Page de la licence
\begin{frame}
	\faCopyright\ 2016 Vincent Goulet pour la 
	\href{https://ctan.org/pkg/formation-latex-ul}{version originale}. La liste des sources qui ont 
	servi à l'élaboration de cette formation se trouve à la fin du présent document.
	
	\faCreativeCommons\ Cette création est mise à disposition selon le contrat 
	\href{http://creativecommons.org/licenses/by-sa/4.0/deed.fr}{%
	Attribution-Partage dans les mêmes conditions 4.0 International de Creative Commons}. 
	En vertu de ce contrat, vous êtes libre de :
	
	\begin{itemize}
		\item partager -- reproduire, distribuer et communiquer l’oeuvre;
		\item remixer -- adapter l’oeuvre;
		\item utiliser cette oeuvre à des fins commerciales.
	\end{itemize}

	Selon les conditions suivantes :
	
	\begin{itemize}
		\item Attribution -- Vous devez créditer l’oeuvre, intégrer un lien vers le contrat et indiquer si des modifications ont été effectuées à l’oeuvre. Vous devez indiquer ces informations par tous les moyens possibles, mais vous ne pouvez suggérer que l’Offrant vous soutient ou soutient la façon dont vous avez utilisé son oeuvre.
		\item Partage dans les mêmes conditions -- Dans le cas où vous modifiez, transformez ou créez à partir du matériel composant l’oeuvre
		originale, vous devez diffuser l’oeuvre modifiée dans les même conditions, c’est-à-dire avec le même contrat avec lequel l’oeuvre originale a été diffusée.
	\end{itemize}
\end{frame}

% Table des matières
\begin{frame}{Sommaire de la formation}
	\tableofcontents
\end{frame}
\section{Objets flottants}

\subsection{Tableaux}

% Tableaux (introduction)
\begin{frame}[fragile,c]{Les tableaux}
	\framesubtitle{Introduction}
	
	\begin{itemize}
		\item Construire des tableaux avec \LaTeX\ demande du doigté.
		\item Il n'existe pas une, pas deux, mais de multiples manières de construire des tableaux.
		\item \LaTeX\ fournit deux environnements de base : \texttt{tabular} et \texttt{tabular*}
	\end{itemize}

	\begin{columns}
		\begin{column}{.49\textwidth}
\begin{codesource}
	\begin{tabular}{colonnes}
		cellule1 & cellule2 & cellule3 \\
		cellule4 & cellule5 & cellule6 \\
		cellule7 & cellule8 & cellule9
	\end{tabular}
\end{codesource}
		\end{column}
		\begin{column}{.49\textwidth}
\begin{codesource}
	\begin{tabular*}{largeur}{colonnes}
		cellule1 & cellule2 & cellule3 \\
		cellule4 & cellule5 & cellule6 \\
		cellule7 & cellule8 & cellule9
	\end{tabular*}
\end{codesource}
		\end{column}
	\end{columns}

	\begin{itemize}
		\item Nous verrons également un troisième environnement, \texttt{tabularx}, fourni avec
		le \emph{package} du même nom.
		\item La syntaxe de \texttt{tabularx} est identique à celle de \texttt{tabular}.
	\end{itemize}
\end{frame}

% Tableaux (construction)
\begin{frame}[fragile]{Les tableaux}
	\framesubtitle{Construction}
	Reprenons les tableaux de la diapositive précédente:
	
	\begin{columns}
		\begin{column}{.49\textwidth}
			\begin{codesource}
				\begin{tabular}{colonnes}
					cellule1 & cellule2 & cellule3 \\
					cellule4 & cellule5 & cellule6 \\
					cellule7 & cellule8 & cellule9
				\end{tabular}
			\end{codesource}
		\end{column}
		\begin{column}{.49\textwidth}
			\begin{codesource}
				\begin{tabular*}{largeur}{colonnes}
					cellule1 & cellule2 & cellule3 \\
					cellule4 & cellule5 & cellule6 \\
					cellule7 & cellule8 & cellule9
				\end{tabular*}
			\end{codesource}
		\end{column}
	\end{columns}

	\begin{onlyenv}<2>
		\begin{itemize}
			\item On spécifie le \textbf{nombre de colonnes} et l'\textbf{alignement du texte} dans l'argument \texttt{colonnes}.
			\begin{itemize}
				\scriptsize
				\item Les arguments possibles sont \texttt{l} (\emph{left}), \texttt{c} (\emph{center}),
					et \texttt{r} (\emph{right}).
				\item On spécifie une colonne de largeur spécifique avec \texttt{p\{largeur\}}.
				\item \texttt{tabularx} accepte aussi l'argument \texttt{X}, qui ajuste la largeur de la
					colonne en fonction de la largeur du tableau.
				\item Le symbole \texttt{|} est utilisé pour insérer une ligne verticale entre des colonnes.
			\end{itemize}
		\end{itemize}
	\end{onlyenv}

	\begin{onlyenv}<3>
		\begin{itemize}
			\item La \textbf{largeur} d'un tableau dépend de l'environnement utilisé :
			\begin{itemize}
				\scriptsize
				\item \texttt{tabular} : largeur du tableau = largeur de son contenu;
				\item \texttt{tabular*} et \texttt{tabularx} : largeur déterminée par l'argument 
					\texttt{largeur}.
			\end{itemize}
		\end{itemize}
	\end{onlyenv}

	\begin{onlyenv}<4>
		\begin{itemize}
			\item On sépare chaque cellule d'une \textbf{ligne} avec le symbole \texttt{\&}.
			\item On termine une ligne avec \textbackslash\textbackslash, \textbf{à l'exception de la dernière ligne}.
			\item On insère une ligne horizontale pleine largeur entre deux lignes avec \cmd{hline}.
			\item La commande \cmd{multicolumn\{cols\}\{pos\}\{text\}} sert à «fusionner» les cellules d'une ligne.
			\begin{itemize}
				\scriptsize
				\item \texttt{cols} : étendue de la cellule en colonnes;
				\item \texttt{pos} : alignement du texte (\texttt{l},\texttt{c},\texttt{r});
				\item \texttt{text} : le contenu de la cellule.
			\end{itemize}
		\end{itemize}
	\end{onlyenv}
	
\end{frame}

% Tableaux (exemple concret)
\begin{frame}[fragile,c]{Les tableaux}
	\framesubtitle{Exemple concret}
\begin{codesource}
	\begin{tabularx}{\textwidth}{X|rrr|r|rrr}
		\textbf{\'{E}quipes}	&	\multicolumn{7}{c}{\textbf{Statistiques}} \\
		\hline\hline
		NFC North Team		&	W	&	L	&	T	&	PCT		&	PF	&	PA	&	Net Pts \\
		\hline
		Minnesota Vikings	&	13	&	3	&	0	&	.813	&	382	&	252	&	130 \\
		Detroit Lions		&	9	&	7	&	0	&	.563	&	410	&	376	&	34 \\
		Green Bay Packers	&	7	&	9	& 	0	&	.438	&	320	&	384	&	-64 \\
		Chicago Bears		&	5	&	11	&	0	&	.313	&	264	&	320	&	-56
	\end{tabularx}
\end{codesource}

	\begin{tabularx}{\textwidth}{X|rrr|r|rrr}
		\textbf{Équipes}	&	\multicolumn{7}{c}{\textbf{Statistiques}} \\
		\hline\hline
		NFC North Team		&	W	&	L	&	T	&	PCT		&	PF	&	PA	&	Net Pts \\
		\hline
		Minnesota Vikings	&	13	&	3	&	0	&	.813	&	382	&	252	&	130 \\
		Detroit Lions		&	9	&	7	&	0	&	.563	&	410	&	376	&	34 \\
		Green Bay Packers	&	7	&	9	& 	0	&	.438	&	320	&	384	&	-64 \\
		Chicago Bears		&	5	&	11	&	0	&	.313	&	264	&	320	&	-56
	\end{tabularx}
\end{frame}

\subsection{Figures}
% Mathématiques

\section{Maths}

% Introduction
\begin{frame}[c]{Maths in \LaTeX}
	\framesubtitle{Introduction}

	\begin{itemize}
		\item Maths are \textbf{THE} reason why \TeX\ exists. \TeX\ exists because it is otherwise very difficult to render complex equations in a document.
		\item The \emph{American Mathematical Society} supports \TeX\ and \LaTeX\ from the beginning. It has built numerous packages to facilitate the writing and rendering of maths.
		\item An \textbf{essential} package that you \textbf{have to use} is
		\href{https://ctan.org/pkg/amsmath}{\texttt{amsmath}}.
		\item \LaTeX\ takes care of all typographic conventions:
		\begin{itemize}
			\scriptsize
			\item constants vs variables, equation layout and numbering;
			\item spaces between symbols and operators.
		\end{itemize}
		\item To use maths in \LaTeX, you have to put it in ``Math Mode''.
	\end{itemize}
\end{frame}

\subsection{Math Modes}

% Modes mathématiques
\begin{frame}[fragile]{Math Modes}
	There is two ways of writing equations in \LaTeX:
	
	\begin{enumerate}
		\item ``Inline'', directly in the text like $(a + b)^2 = a^2 + 2ab + b^2$ by placing the equation between \$\ and \$.
\begin{codesource}
	``Inline'', directly in the text like $(a + b)^2 = a^2 + 2ab + b^2$ by placing 
	the equations between \$\ and \$.
\end{codesource}
		\item In their own ``paragraph'', separated from the text like
			\begin{equation*}
				\int_0^\infty f(x)\, dx =
				\sum_{i = 1}^n \alpha_i e^{x_i} f(x_i)
			\end{equation*}
			by using different types of environments.
\begin{codesource}
	In their own ``paragraph'', separated from the text like
	\begin{equation*}
		\int_0^\infty f(x)\, dx =
		\sum_{i = 1}^n \alpha_i e^{x_i} f(x_i)
	\end{equation*}
	by using different types of environments.
\end{codesource}
	\end{enumerate}
\end{frame}

% Environnements mathématiques standards

\begin{frame}[fragile,c]{Math Environments}
	\framesubtitle{\LaTeX\ Standard Environments}
	There are several \LaTeX\ environments you can use to write equations:
	\begin{itemize}
		\item One-line equations:
\begin{codesource}
	\begin{displaymath} equation...	\end{displaymath}
	\begin{equation} equation... \end{equation}
	\begin{equation*} equation... \end{equation*}
\end{codesource}
		\item Multiline equations:
\begin{codesource}
	\begin{eqnarray} equation...  \end{eqnarray}
	\begin{eqnarray*} equation... \end{eqnarray*}
\end{codesource}
	\end{itemize}

	\pause
	For multiline equations, you should use the \textbf{amsmath} package's environments. They are more versatile, easier to use and they give a better rendering of equations.
\end{frame}

% Environnements mathématiques de amsmath

\begin{frame}[fragile,c]{Math Environments}
	\framesubtitle{\textbf{amsmath} package's Environments}
	\begin{description}[aaaaaaaaaaaaaaaaaaa]
		\item[\texttt{multline, multline*}] For single equations too long to fit on one line.
		\item[\texttt{align, align*}] For multiple equations aligned on a single marker (usually the = sign).
		\item[\texttt{gather, gather*}] For multiple equations, horizontally centered.
		\item[\texttt{falign, falign*}] Like \texttt{align}, but separates both sides of the equation to fit the line width.
		\item[\texttt{alignat, alignat*}] The opposite of \texttt{falign}: no space separates both sides of the equation.
		\item[\texttt{split}] For single equations too long to fit on one line; allows the alignment of the equation on a single marker.
	\end{description}
\end{frame}

% Environnements mathématiques (exemples)

\begin{frame}[fragile]{Math Environments}
	\framesubtitle{Examples}

	\begin{onlyenv}<1>
\begin{codesource}
	\begin{equation}
		a = b
	\end{equation}
\end{codesource}	
	\begin{equation}
		a = b
	\end{equation}
	
\begin{codesource}
	\begin{equation*}
		a = b
	\end{equation*}
\end{codesource}	
	\begin{equation*}
		a = b
	\end{equation*}
	
\begin{codesource}
	\begin{multline}
		a + b + c + d + e + f \\
		+ i + j + k + l + m + n
	\end{multline}
\end{codesource}
	\begin{multline}
	a + b + c + d + e + f \\
	+ o + p + q + r + s + t
	\end{multline}	
	\end{onlyenv}

	\begin{onlyenv}<2>
\begin{codesource}
	\begin{align}
		a_1 &= b_1 + c_1 \\
		a_2 &= b_2 + c_2 - d_2 + e_2
	\end{align}
\end{codesource}
	\begin{align}
		a_1 &= b_1 + c_1 \\
		a_2 &= b_2 + c_2 - d_2 + e_2
	\end{align}
	
\begin{codesource}
	\begin{gather}
		a_1 = b_1 + c_1 \\
		a_2 = b_2 + c_2 - d_2 + e_2
	\end{gather}
\end{codesource}
	\begin{gather}
		a_1 = b_1 + c_1 \\
		a_2 = b_2 + c_2 - d_2 + e_2
	\end{gather}
	\end{onlyenv}

	\begin{onlyenv}<3>
\begin{codesource}
	\begin{equation}
		\begin{split}
			a &= b + c - d \\
			&\phantom{=} + e - f \\
			&= g + h \\
			&= i
		\end{split}
	\end{equation}
\end{codesource}
		\begin{equation}
			\begin{split}
				a &= b + c - d \\
				&\phantom{=} + e - f \\
				&= g + h \\
				&= i
			\end{split}
		\end{equation}
	\end{onlyenv}
\end{frame}

\subsection{Symbols}

% Principaux éléments du mode mathématique
\begin{frame}[fragile,c]{Main elements of Math Mode}
	\begin{itemize}
		\item Basic math symbols:
		 	\texttt{+ - = < > / : ! ' | [ ] ( ) \{ \}}
		\item Exponents are written with \textasciicircum. \lstinline|x^2| becomes $x^2$.
		\item Indices are written with the underscore \_. \lstinline|a_n| becomes $a_n$.
		\item Exponents and indices can be combined: \lstinline|x_i^k| becomes $x_i^k$.
		\item Exponents and indices can be grouped with \{ and \}. \lstinline|A_{i_s, k^n}^{y_i}|
			becomes $A_{i_s, k^n}^{y_i}$.
	\end{itemize}
\end{frame}

% Fractions
\begin{frame}[fragile,c]{Fractions}
	\begin{itemize}
		\item Fractions are written with \cmd{frac\{numerator\}\{denominator\}}.
		\begin{columns}
			\begin{column}{.4\textwidth}
			\vspace{-4.5mm}
\begin{codesource}
	% Fraction size inside text
	Let $z_1 = \frac{x}{y}$ and
	$z_2 = xy$...
\end{codesource}
			\end{column}
			\begin{column}{.4\textwidth}
				Let $z_1 = \frac{x}{y}$ and
				$z_2 = xy$...
			\end{column}
		\end{columns}
	
		\pause
		
		\begin{columns}
			\begin{column}{.4\textwidth}
				\vspace{-4.5mm}
\begin{codesource}
	% Fraction size outside text
	Let
	\begin{equation*}
		z_1 = \frac{x}{y}
	\end{equation*}
	and $z_2 = xy$...
\end{codesource}
			\end{column}
			\begin{column}{.4\textwidth}
				Let
				\begin{equation*}
					z_1 = \frac{x}{y}
				\end{equation*}
				and $z_2 = xy$...
			\end{column}
		\end{columns}
	
		\pause
		
		\begin{columns}
			\begin{column}{.4\textwidth}
				\vspace{-4.5mm}
\begin{codesource}
	% Combined sizes
	Let
	\begin{equation*}
		z = \frac{\frac{x}{2} + 1}{y}.
	\end{equation*}
\end{codesource}
			\end{column}
			\begin{column}{.4\textwidth}
				Let
				\begin{equation*}
					z = \frac{\frac{x}{2} + 1}{y}.
				\end{equation*}
			\end{column}
		\end{columns}
	\end{itemize}
\end{frame}

\begin{frame}[fragile]{Roots}
	\begin{itemize}
		\item Roots are written with \cmd{sqrt[n]\{arg\}}.
		\begin{itemize}
			\scriptsize
			\item The default root (if \texttt{n} as not been defined) is the square root.
			\item The root sign is automatically fitted to \texttt{arg}.
		\end{itemize}
		\begin{columns}
			\begin{column}{.4\textwidth}
			\vspace{-4.5mm}
\begin{codesource}
	\sqrt{2}
	
		
	\sqrt{625}
	
		
	\sqrt[3]{8}
	
	
	\sqrt[n]{x + y + z}
	
	
	\sqrt{\frac{x + y}{x^2 - y^2}}
\end{codesource}
			\end{column}
			\begin{column}{.4\textwidth}
				\begin{equation*}					
					\sqrt{2}		
				\end{equation*}	
				\begin{equation*}
					\sqrt{625}
				\end{equation*}		
				\begin{equation*}				
					\sqrt[3]{8}				
				\end{equation*}
				\begin{equation*}
					\sqrt[n]{x + y + z}
				\end{equation*}
				\begin{equation*}
					\sqrt{\frac{x + y}{x^2 - y^2}}
				\end{equation*}
			\end{column}
		\end{columns}
	\end{itemize}
\end{frame}

\begin{frame}[fragile,c]{Sums and Integrals}
	\begin{itemize}
		\item Sums are written with \cmd{sum}.
		\item Integrals are written with \cmd{int}
		\item Lower and upper limits are written with indices (\_) and exponents (\textasciicircum).
			\begin{columns}
				\column{.4\textwidth}					
\begin{codesource}
	\sum_{i = 0}^n x_1
\end{codesource}
				\column{.4\textwidth}
					\begin{equation*}
						\sum_{i = 0}^n x_1
					\end{equation*}
			\end{columns}
			\begin{columns}
				\column{.4\textwidth}
\begin{codesource}
	\int_0^{10} f(x)\, dx
\end{codesource}
				\column{.4\textwidth}
					\begin{equation*}
						\int_0^{10} f(x)\, dx
					\end{equation*}
			\end{columns}
		\item The \textbf{amsmath} package also provides the \cmd{iint}
			and \cmd{iiint} to generate multiple integrals like $\iint$ and $\iiint$.
	\end{itemize}
\end{frame}

\begin{frame}[c]{Functions, operators, etc.}
	Since in Math Mode letters are considered variables, we can't manually write functions. \LaTeX\ defines commands for these functions:
	
	\begin{center}
		\begin{tabular}{lllllll}
			\cmd{arccos} & \cmd{cosh} & \cmd{det} & \cmd{inf} & \cmd{limsup} & \cmd{Pr} & \cmd{tan} \\
			\cmd{arcsin} & \cmd{cot} & \cmd{dim} & \cmd{ker} & \cmd{ln} & \cmd{sec} & \cmd{tanh} \\
			\cmd{arctan} & \cmd{coth} & \cmd{exp} & \cmd{lg} & \cmd{log} & \cmd{sin} & 	\\
			\cmd{arg} &	\cmd{csc} & \cmd{gcd} & \cmd{lim} & \cmd{max} & \cmd{sinh} &	\\
			\cmd{cos} & \cmd{deg} & \cmd{hom} & \cmd{liminf} & \cmd{min} & \cmd{sup} &	
		\end{tabular}
	\end{center}

	\pause
	
	There are also commands for \textbf{greek letters}, \textbf{text} and \textbf{spaces}, \textbf{continuation dots}, \textbf{calligraphic letters}, \textbf{binary operators} and \textbf{relations}, \textbf{arrows}, \textbf{accents} and many more!
	
	Refer to the \textbf{amsmath} package documentation and the
	\href{http://tug.ctan.org/info/symbols/comprehensive/symbols-a4.pdf}{Comprehensive \LaTeX\ Symbol List} -- 338 pages of pleasant reading! -- to learn about all the functionalities.
	
\end{frame}
% Bibliographies et citations

\section{Bibliographies et citations}

\subsection{Types de bibliographies}

% Bibliographie manuelle
\begin{frame}[fragile,c]{Bibliographie manuelle}
	\begin{itemize}
		\item On peut se «tricoter» une bibliographie à la main avec l'environnement \texttt{thebibliography}.
\begin{codesource}
	\begin{thebibliography}{libellé le plus long}
		\bibitem[libellé]{id_citation} Entrée bibliographique #1
		\bibitem[libellé]{id_citation} Entrée bibliographique #2
		[...]
	\end{thebibliography}
\end{codesource}
		
		\pause
		
		\item Chaque entrée bibliographique est rédigée avec la commande \cmd{bibitem}.
		\begin{itemize}
			\scriptsize
			\item Le \texttt{libellé} est ce qu'on retrouvera dans la référence à l'intérieur du texte. S'il n'y a pas de libellé, \LaTeX\ produira un numéro séquentiel à la place.
			\item \texttt{id\_citation} est l'élément qu'on utilise pour citer une source.
			\item L'\texttt{entrée bibliographique} contient toutes les informations bibliographiques
				de notre source.
		\end{itemize}
	
		\pause
		
		\item Le \texttt{libellé le plus long} à l'ouverture correspond à celui des libellés de tous 
		les \cmd{bibitem} qui est le plus long.
		\item La bibliographie est insérée dans le document là où l'environnement 
			\texttt{thebibliography} est inséré dans le code.
	\end{itemize}
\end{frame}

\begin{frame}[fragile,c]{Bibliographie manuelle}
	\framesubtitle{Un exemple\ldots}
\begin{codesource}
	\begin{thebibliography}{99}		
		\bibitem[Kopka and Daly, 2004]{kopkadaly:2004}
			Kopka, Helmut et Patrick W. Daly (2004).
			\newblock Guide to \LaTeX, Fourth Edition,
			\newblock Addison-Wesley,
			\newblock ISBN 978-0-321-17385-0, 597 p.
		\bibitem[Mittelbach et al., 2004]{mittelbach:2004}
			Mittelbach, Frank \emph{et al.} (2004).
			\newblock The \LaTeX\ Companion, Second Edition,
			\newblock Addison-Wesley,
			\newblock ISBN 978-0201362992, 1120p.
		\bibitem[Goossens and Mittelbach, 2007]{goossens:2007}
			Goossens, Michel et Franck Mittelbach (2007).
			\newblock The \LaTeX\ Graphics Companion, Second Edition,
			\newblock Addison-Wesley,
			\newblock ISBN 978-0321508928, 976p.
	\end{thebibliography}
\end{codesource}
\end{frame}

\begin{frame}[c]{Bibliographie automatique}
	\framesubtitle{Une introduction à BiB\TeX}
	
	\begin{itemize}
		\item BiB\TeX\ est un programme (un compilateur) auxiliaire de \LaTeX\ qui construit 
			automatiquement une bibliographie à partir d'une base de données.
		\item Il est \emph{de facto} le système standard de traitement des bibliographies.
		\item Il est stable et simple à utiliser.
		\item C'est généralement le seul format accepté par les revues scientifiques.
		\item Vous pouvez exporter nos références bibliographiques stockées dans \textbf{EndNote}
			directement en format BiB\TeX.
		\item Vous pouvez télécharger des références en format BiB\TeX depuis HECo, Google Scholar,
			ProQuest, Ebsco et de nombreuses autres banques de données de la bibliothèque.
	\end{itemize}
\end{frame}

\begin{frame}[c]{Compilation d'un document avec BiB\TeX}
	\begin{itemize}
		\item À la formation précédente, nous avons schématisé la compilation d'un document comme suit:
	\end{itemize}
	{
		\begin{minipage}[t]{0.25\linewidth}
			\centering
			{\Large\faFileTextO} \\
			code source
		\end{minipage}
		\hfill{\Large\faArrowRight}\hfill
		\begin{minipage}[t]{0.25\linewidth}
			\centering
			{\Large\faCogs} \\
			pdf\LaTeX
		\end{minipage}
		\hfill{\Large\faArrowRight}\hfill
		\begin{minipage}[t]{0.25\linewidth}
			\centering
			{\Large\faFilePdfO} \\
			document .pdf
		\end{minipage}
	}
	
	\pause
	
	\begin{itemize}
		\item Avec BiB\TeX, la séquence de compilations change:		
	\end{itemize}

	{
		\begin{minipage}[t]{0.125\linewidth}
			\centering
			{\Large\faFileTextO} \\
			code source
		\end{minipage}
		\hfill{\Large\faArrowRight}\hfill
		\begin{minipage}[t]{0.125\linewidth}
			\centering
			{\Large\faCogs} \\
			pdf\LaTeX
		\end{minipage}
		\hfill{\Large\faArrowRight}\hfill
		\begin{minipage}[t]{0.125\linewidth}
			\centering
			{\Large\faCogs} \\
			BiB\TeX
		\end{minipage}
		\hfill{\Large\faArrowRight}\hfill
		\begin{minipage}[t]{0.125\linewidth}
			\centering
			{\Large\faCogs} \\
			pdf\LaTeX
		\end{minipage}
		\hfill{\Large\faArrowRight}\hfill
		\begin{minipage}[t]{0.125\linewidth}
			\centering
			{\Large\faCogs} \\
			pdf\LaTeX
		\end{minipage}
		\hfill{\Large\faArrowRight}\hfill
		\begin{minipage}[t]{0.125\linewidth}
			\centering
			{\Large\faFilePdfO} \\
			document .pdf
		\end{minipage}
	}
\end{frame}

\subsection{Création d'une bibliographie}

\subsection{Citations}
% Bibliographie
\scriptsize

\section{Bibliography}

\begin{frame}[c]{Bibliography}
	\framesubtitle{For those who still prefer the scent of ink}
	\setbeamertemplate{bibliography item}[book]
		
	\begin{thebibliography}{99}		
		\bibitem[Kopka and Daly, 2004]{kopkadaly:2004}
			Kopka, Helmut and Patrick W. Daly (2004).
			\newblock Guide to \LaTeX, Fourth Edition,
			\newblock Addison-Wesley,
			\newblock ISBN 978-0-321-17385-0, 597 p.
		\bibitem[Mittelbach et al., 2004]{mittelbach:2004}
			Mittelbach, Frank \emph{et al.} (2004).
			\newblock The \LaTeX\ Companion, Second Edition,
			\newblock Addison-Wesley,
			\newblock ISBN 978-0201362992, 1120p.
		\bibitem[Goossens and Mittelbach, 2007]{goossens:2007}
			Goossens, Michel and Franck Mittelbach (2007).
			\newblock The \LaTeX\ Graphics Companion, Second Edition,
			\newblock Addison-Wesley,
			\newblock ISBN 978-0321508928, 976p.
	\end{thebibliography}

\end{frame}

\begin{frame}[c]{Bibliography}
	\framesubtitle{For the environmentally conscious}
	\setbeamertemplate{bibliography item}[online]
	
	\begin{onlyenv}<1>
		\begin{thebibliography}{99}
			\bibitem[Goulet, 2016]{goulet:2016}
				Goulet, Vincent (2016).
				\newblock formation-latex-ul -- Introductory \LaTeX\ course in French,
				\newblock Comprehensive \TeX\ Archive Network,
				\newblock Viewed on February 22, 2018 at \href{https://ctan.org/pkg/formation-latex-ul}{%
					https://ctan.org/pkg/formation-latex-ul}
			\bibitem[Lees-Miller, 2018]{leesmiller:2018}
				Lees-Miller, John D. (2018).
				\newblock Free \& Interactive Online Introduction to \LaTeX,
				\newblock Overleaf,
				\newblock Viewed on February 22, 2018 at \href{https://www.overleaf.com/latex/learn/free-online-introduction-to-latex-part-1}{%
					https://www.overleaf.com/latex/learn/free-online-introduction-to-latex-part-1}
			\bibitem[ShareLaTeX, 2018]{sharelatex:2018}
				Share\LaTeX\ Documentation,
				\newblock Share\LaTeX,
				\newblock Viewed on February 22, 2018 at \href{https://fr.sharelatex.com/learn/Main_Page}{%
					https://fr.sharelatex.com/learn/Main\_Page}			
		\end{thebibliography}
	\end{onlyenv}

	\begin{onlyenv}<2>
		\begin{thebibliography}{99}
			\bibitem[LaTeX Wikibook]{wikibook}
				\href{https://en.wikibooks.org/wiki/LaTeX}{\LaTeX\ WikiBook}
			\bibitem[ShareLaTeX]{sharelatex}
				\href{https://fr.sharelatex.com/learn}{Share\LaTeX\ Documentation}
			\bibitem[Stack Exchange]{stackex}
				\href{https://tex.stackexchange.com/}{\TeX\ - \LaTeX\ Stack Exchange}
			\bibitem[LaTeX Community]{latexcomm}
				\href{http://latex.org/forum/}{\LaTeX\ Community}
			\bibitem[CTAN]{ctan}
				\href{https://ctan.org/}{Comprehensive \TeX\ Archive Network}
			\bibitem[TeX FAQ]{texfaq}
				\href{http://www.tex.ac.uk/}{UK List of TEX Frequently Asked Questions}
			\bibitem{google}
				Google\ldots
		\end{thebibliography}
	\end{onlyenv}
\end{frame}

% Période de questions
\begin{frame}[c]{Questions and comments}
\LARGE
\begin{block}{Training Session Documentation}
	\ttfamily\href{http://bit.ly/enltxhec2}{http://bit.ly/enltxhec2}
\end{block}
\begin{block}{Training Session Evaluation Survey}
	\ttfamily\href{http://bit.ly/enltxsurvey2}{http://bit.ly/enltxsurvey2}
\end{block}
\begin{block}{\TeX nical Support}
	\ttfamily Benoit Hamel : <benoit.2.hamel@hec.ca>
\end{block}
\end{frame}

\end{document}
