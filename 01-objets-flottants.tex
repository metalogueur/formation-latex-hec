\section{Objets flottants}

\begin{frame}[c]{Les objets «flottants»}

	On a déjà mentionné que la force de \TeX\ et \LaTeX\ était la typographie et qu'il valait mieux les laisser faire leur travail.
	
	Les tableaux et figures (images et graphiques) sont un excellent exemple du pouvoir du système.
\end{frame}

\subsection{Tableaux}

% Tableaux (introduction)
\begin{frame}[fragile,c]{Les tableaux}
	\framesubtitle{Introduction}
	
	\begin{itemize}
		\item Construire des tableaux avec \LaTeX\ demande du doigté.
		\item Il n'existe pas une, pas deux, mais de multiples manières de construire des tableaux.
		\item \LaTeX\ fournit deux environnements de base : \texttt{tabular} et \texttt{tabular*}
	\end{itemize}

	\begin{columns}
		\begin{column}{.49\textwidth}
\begin{codesource}
	\begin{tabular}{colonnes}
		cellule1 & cellule2 & cellule3 \\
		cellule4 & cellule5 & cellule6 \\
		cellule7 & cellule8 & cellule9
	\end{tabular}
\end{codesource}
		\end{column}
		\begin{column}{.49\textwidth}
\begin{codesource}
	\begin{tabular*}{largeur}{colonnes}
		cellule1 & cellule2 & cellule3 \\
		cellule4 & cellule5 & cellule6 \\
		cellule7 & cellule8 & cellule9
	\end{tabular*}
\end{codesource}
		\end{column}
	\end{columns}

	\begin{itemize}
		\item Nous verrons également un troisième environnement, \texttt{tabularx}, fourni avec
		le \emph{package} du même nom.
		\item La syntaxe de \texttt{tabularx} est identique à celle de \texttt{tabular}.
	\end{itemize}
\end{frame}

% Tableaux (construction)
\begin{frame}[fragile]{Les tableaux}
	\framesubtitle{Construction}
	Reprenons les tableaux de la diapositive précédente:
	
	\begin{columns}
		\begin{column}{.49\textwidth}
			\begin{codesource}
				\begin{tabular}{colonnes}
					cellule1 & cellule2 & cellule3 \\
					cellule4 & cellule5 & cellule6 \\
					cellule7 & cellule8 & cellule9
				\end{tabular}
			\end{codesource}
		\end{column}
		\begin{column}{.49\textwidth}
			\begin{codesource}
				\begin{tabular*}{largeur}{colonnes}
					cellule1 & cellule2 & cellule3 \\
					cellule4 & cellule5 & cellule6 \\
					cellule7 & cellule8 & cellule9
				\end{tabular*}
			\end{codesource}
		\end{column}
	\end{columns}

	\begin{onlyenv}<2>
		\begin{itemize}
			\item On spécifie le \textbf{nombre de colonnes} et l'\textbf{alignement du texte} dans l'argument \texttt{colonnes}.
			\begin{itemize}
				\scriptsize
				\item Les arguments possibles sont \texttt{l} (\emph{left}), \texttt{c} (\emph{center}),
					et \texttt{r} (\emph{right}).
				\item On spécifie une colonne de largeur spécifique avec \texttt{p\{largeur\}}.
				\item \texttt{tabularx} accepte aussi l'argument \texttt{X}, qui ajuste la largeur de la
					colonne en fonction de la largeur du tableau.
				\item Le symbole \texttt{|} est utilisé pour insérer une ligne verticale entre des colonnes.
			\end{itemize}
		\end{itemize}
	\end{onlyenv}

	\begin{onlyenv}<3>
		\begin{itemize}
			\item La \textbf{largeur} d'un tableau dépend de l'environnement utilisé :
			\begin{itemize}
				\scriptsize
				\item \texttt{tabular} : largeur du tableau = largeur de son contenu;
				\item \texttt{tabular*} et \texttt{tabularx} : largeur déterminée par l'argument 
					\texttt{largeur}.
			\end{itemize}
		\end{itemize}
	\end{onlyenv}

	\begin{onlyenv}<4>
		\begin{itemize}
			\item On sépare chaque cellule d'une \textbf{ligne} avec le symbole \texttt{\&}.
			\item On termine une ligne avec \textbackslash\textbackslash, \textbf{à l'exception de la dernière ligne}.
			\item On insère une ligne horizontale pleine largeur entre deux lignes avec \cmd{hline}.
			\item La commande \cmd{multicolumn\{cols\}\{pos\}\{text\}} sert à «fusionner» les cellules d'une ligne.
			\begin{itemize}
				\scriptsize
				\item \texttt{cols} : étendue de la cellule en colonnes;
				\item \texttt{pos} : alignement du texte (\texttt{l},\texttt{c},\texttt{r});
				\item \texttt{text} : le contenu de la cellule.
			\end{itemize}
		\end{itemize}
	\end{onlyenv}
	
\end{frame}

% Tableaux (exemple concret)
\begin{frame}[fragile,c]{Les tableaux}
	\framesubtitle{Exemple concret}
\begin{codesource}
	\begin{tabularx}{\textwidth}{X|rrr|r|rrr}
		\textbf{\'{E}quipes}	&	\multicolumn{7}{c}{\textbf{Statistiques}} \\
		\hline\hline
		NFC North			&	W	&	L	&	T	&	PCT		&	PF	&	PA	&	Net Pts \\
		\hline
		Minnesota Vikings	&	13	&	3	&	0	&	.813	&	382	&	252	&	130 \\
		Detroit Lions		&	9	&	7	&	0	&	.563	&	410	&	376	&	34 \\
		Green Bay Packers	&	7	&	9	& 	0	&	.438	&	320	&	384	&	-64 \\
		Chicago Bears		&	5	&	11	&	0	&	.313	&	264	&	320	&	-56
	\end{tabularx}
\end{codesource}

	\begin{tabularx}{\textwidth}{X|rrr|r|rrr}
		\textbf{Équipes}	&	\multicolumn{7}{c}{\textbf{Statistiques}} \\
		\hline\hline
		NFC North			&	W	&	L	&	T	&	PCT		&	PF	&	PA	&	Net Pts \\
		\hline
		Minnesota Vikings	&	13	&	3	&	0	&	.813	&	382	&	252	&	130 \\
		Detroit Lions		&	9	&	7	&	0	&	.563	&	410	&	376	&	34 \\
		Green Bay Packers	&	7	&	9	& 	0	&	.438	&	320	&	384	&	-64 \\
		Chicago Bears		&	5	&	11	&	0	&	.313	&	264	&	320	&	-56
	\end{tabularx}
\end{frame}

% Tableaux flottants
\begin{frame}[fragile]{Les tableaux flottants}
	
	\begin{onlyenv}<1-2>
		\begin{itemize}
			\item Les environnements \texttt{tabular}, \texttt{tabular*} et \texttt{tabularx} insèrent
				un tableau là où on le place dans le texte, ce qui n'est pas idéal.
			\item \LaTeX\ peut déterminer l'emplacement idéal pour insérer un tableau grâce à l'environnement \texttt{table}.
\begin{codesource}
	\begin{table}[emplacement]
		\begin{tabularx}{\textwidth}{lccc}
			...
		\end{tabularx}
		\caption{texte}
	\end{table}
\end{codesource}

			\pause
			\item L'argument optionnel \texttt{emplacement} prend une ou plusieurs des valeurs suivantes:
				\begin{description}[b]
					\item[t] Tableau placé en haut de la page (\emph{\textbf{t}op})
					\item[b] Tableau placé en bas de la page (\emph{\textbf{b}ottom})
					\item[p] Tableau placé sur une page à part (\emph{\textbf{p}age})
					\item[h] Tableau placé à l'endroit où il a été inséré dans le texte (\emph{\textbf{h}ere})
				\end{description}
			\item La commande \cmd{caption} insère une légende sous le tableau.
			\item La commande \cmd{listoftables} génère une liste de tous les environnements \texttt{table} insérés dans le texte.
		\end{itemize}
	\end{onlyenv}

	\begin{onlyenv}<3>
		\begin{codesource}
\begin{table}
	\begin{tabularx}{\textwidth}{X|rrr|r|rrr}
		\'{E}quipes			&	W	&	L	&	T	&	PCT		&	PF	&	PA	&	Net Pts \\
		\hline
		Minnesota Vikings	&	13	&	3	&	0	&	.813	&	382	&	252	&	130 \\
		Detroit Lions		&	9	&	7	&	0	&	.563	&	410	&	376	&	34 \\
		Green Bay Packers	&	7	&	9	& 	0	&	.438	&	320	&	384	&	-64 \\
		Chicago Bears		&	5	&	11	&	0	&	.313	&	264	&	320	&	-56
	\end{tabularx}
	\caption{Les statistiques 2017 des équipes de la NFC North de la NFL}
\end{table}
		\end{codesource}
		
		\begin{table}
			\begin{tabularx}{\textwidth}{X|rrr|r|rrr}
				Équipes				&	W	&	L	&	T	&	PCT		&	PF	&	PA	&	Net Pts \\
				\hline
				Minnesota Vikings	&	13	&	3	&	0	&	.813	&	382	&	252	&	130 \\
				Detroit Lions		&	9	&	7	&	0	&	.563	&	410	&	376	&	34 \\
				Green Bay Packers	&	7	&	9	& 	0	&	.438	&	320	&	384	&	-64 \\
				Chicago Bears		&	5	&	11	&	0	&	.313	&	264	&	320	&	-56
			\end{tabularx}
			\caption{Les statistiques 2017 des équipes de la NFC North de la NFL}
		\end{table}
	\end{onlyenv}
\end{frame}

\subsection{Figures}

% Insertion d'images
\begin{frame}[fragile,c]{Insertion d'images}
	\begin{itemize}
		\item Pour insérer des images dans un document \LaTeX, nous avons besoin de trois commandes:
\begin{codesource}
	%% Préambule	
	\usepackage{graphicx}
	\graphicspath{{repertoire1}{repertoire2}...}
	
	%% Intérieur du document
	\includegraphics[options]{fichier}
\end{codesource}

		\pause
		\item Le \emph{package} \textbf{graphicx} doit être chargé dans le préambule.
		\item La commande \cmd{graphicspath} sert à spécifier dans quel(s) répertoire(s) se trouvent
			les images.
		\item La commande \cmd{includegraphics} insère l'image dans le document.
		\item Les options de la commande \cmd{includegraphics} règlent, entre autres, la taille, la rotation et l'origine de l'image. Consultez la \href{http://mirrors.ctan.org/macros/latex/required/graphics/grfguide.pdf}{documentation de graphicx} pour connaître la liste des options.
	\end{itemize}
\end{frame}

% Environnement picture
\begin{frame}[fragile]{Insertion de graphiques}
	
	On peut construire des graphiques dans \LaTeX\ avec l'environnement \texttt{picture}
	\footnote{\url{https://en.wikibooks.org/wiki/LaTeX/Picture\#Plotting_graphs}}.
	
	\begin{columns}
		\begin{column}{.49\textwidth}
\begin{codesource}
	\setlength{\unitlength}{1cm}
	\begin{picture}(0,0)(-3,2)
	\put(-1.5,0){\vector(1,0){3}}
	\put(2.7,-0.1){$\chi$}
	\put(0,-1.5){\vector(0,1){3}}
	\multiput(-2.5,1)(0.4,0){13}
	{\line(1,0){0.2}}
	\multiput(-2.5,-1)(0.4,0){13}
	{\line(1,0){0.2}}
	\put(0.2,1.4)
	{$\beta=v/c=\tanh\chi$}
	\qbezier(0,0)(0.8853,0.8853)
	(2,0.9640)
	\qbezier(0,0)(-0.8853,-0.8853)
	(-2,-0.9640)	
	\end{picture}
\end{codesource}	
		\end{column}
		
		\begin{column}{.49\textwidth}
			\setlength{\unitlength}{1cm}
			\begin{picture}(0,0)(-3,2)
			\put(-1.5,0){\vector(1,0){3}}
			\put(2.7,-0.1){$\chi$}
			\put(0,-1.5){\vector(0,1){3}}
			\multiput(-2.5,1)(0.4,0){13}
			{\line(1,0){0.2}}
			\multiput(-2.5,-1)(0.4,0){13}
			{\line(1,0){0.2}}
			\put(0.2,1.4)
			{$\beta=v/c=\tanh\chi$}
			\qbezier(0,0)(0.8853,0.8853)
			(2,0.9640)
			\qbezier(0,0)(-0.8853,-0.8853)
			(-2,-0.9640)	
			\end{picture}
		\end{column}
	\end{columns}	
	
	Pour un usage vraiment intensif des graphiques, vous pouvez utiliser le 
	\href{https://ctan.org/pkg/pgf}{\emph{package} \textbf{TikZ PGF}}.
\end{frame}

% Images et graphiques flottants
\begin{frame}[fragile,c]{Les images et graphiques flottants}
	\begin{itemize}
		\item Tout comme les tableaux, il est préférable de laisser \TeX\ et \LaTeX\ déterminer
			l'emplacement idéal pour les images et graphiques.
		\item Cela est rendu possible avec l'environnement \texttt{figure}.
		\vspace{-2.2mm}
		\begin{columns}
			\begin{column}{.4\textwidth}
\begin{codesource}
	\begin{figure}[emplacement]
		\includegraphics[options]{fichier}
		\caption{texte}
	\end{figure}
\end{codesource}
			\end{column}
			\begin{column}{.4\textwidth}
\begin{codesource}
	\begin{figure}[emplacement]
		\begin{picture}(width,height)(x,y)
			...
		\end{picture}
		\caption{texte}
	\end{figure}
\end{codesource}
			\end{column}
		\end{columns}
	
		\pause
		\item L'argument optionnel \texttt{emplacement} prend les mêmes valeurs qu'avec l'environnement \texttt{table}: \texttt{t},\texttt{b},\texttt{p},\texttt{h}.
		\item La commande \cmd{caption} insère une légende sous la figure.
		\item La commande \cmd{listoffigures} génère une liste de tous les environnements \texttt{figure} insérés dans le texte.
	\end{itemize}
\end{frame}