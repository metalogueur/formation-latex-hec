\section{Objets flottants}

\subsection{Tableaux}

% Tableaux (introduction)
\begin{frame}[fragile,c]{Les tableaux}
	\framesubtitle{Introduction}
	
	\begin{itemize}
		\item Construire des tableaux avec \LaTeX\ demande du doigté.
		\item Il n'existe pas une, pas deux, mais de multiples manières de construire des tableaux.
		\item \LaTeX\ fournit deux environnements de base : \texttt{tabular} et \texttt{tabular*}
	\end{itemize}

	\begin{columns}
		\begin{column}{.49\textwidth}
\begin{codesource}
	\begin{tabular}{colonnes}
		cellule1 & cellule2 & cellule3 \\
		cellule4 & cellule5 & cellule6 \\
		cellule7 & cellule8 & cellule9
	\end{tabular}
\end{codesource}
		\end{column}
		\begin{column}{.49\textwidth}
\begin{codesource}
	\begin{tabular*}{largeur}{colonnes}
		cellule1 & cellule2 & cellule3 \\
		cellule4 & cellule5 & cellule6 \\
		cellule7 & cellule8 & cellule9
	\end{tabular*}
\end{codesource}
		\end{column}
	\end{columns}

	\begin{itemize}
		\item Nous verrons également un troisième environnement, \texttt{tabularx}, fourni avec
		le \emph{package} du même nom.
		\item La syntaxe de \texttt{tabularx} est identique à celle de \texttt{tabular}.
	\end{itemize}
\end{frame}

% Tableaux (construction)
\begin{frame}[fragile]{Les tableaux}
	\framesubtitle{Construction}
	Reprenons les tableaux de la diapositive précédente:
	
	\begin{columns}
		\begin{column}{.49\textwidth}
			\begin{codesource}
				\begin{tabular}{colonnes}
					cellule1 & cellule2 & cellule3 \\
					cellule4 & cellule5 & cellule6 \\
					cellule7 & cellule8 & cellule9
				\end{tabular}
			\end{codesource}
		\end{column}
		\begin{column}{.49\textwidth}
			\begin{codesource}
				\begin{tabular*}{largeur}{colonnes}
					cellule1 & cellule2 & cellule3 \\
					cellule4 & cellule5 & cellule6 \\
					cellule7 & cellule8 & cellule9
				\end{tabular*}
			\end{codesource}
		\end{column}
	\end{columns}

	\begin{onlyenv}<2>
		\begin{itemize}
			\item On spécifie le \textbf{nombre de colonnes} et l'\textbf{alignement du texte} dans l'argument \texttt{colonnes}.
			\begin{itemize}
				\scriptsize
				\item Les arguments possibles sont \texttt{l} (\emph{left}), \texttt{c} (\emph{center}),
					et \texttt{r} (\emph{right}).
				\item On spécifie une colonne de largeur spécifique avec \texttt{p\{largeur\}}.
				\item \texttt{tabularx} accepte aussi l'argument \texttt{X}, qui ajuste la largeur de la
					colonne en fonction de la largeur du tableau.
				\item Le symbole \texttt{|} est utilisé pour insérer une ligne verticale entre des colonnes.
			\end{itemize}
		\end{itemize}
	\end{onlyenv}

	\begin{onlyenv}<3>
		\begin{itemize}
			\item La \textbf{largeur} d'un tableau dépend de l'environnement utilisé :
			\begin{itemize}
				\scriptsize
				\item \texttt{tabular} : largeur du tableau = largeur de son contenu;
				\item \texttt{tabular*} et \texttt{tabularx} : largeur déterminée par l'argument 
					\texttt{largeur}.
			\end{itemize}
		\end{itemize}
	\end{onlyenv}

	\begin{onlyenv}<4>
		\begin{itemize}
			\item On sépare chaque cellule d'une \textbf{ligne} avec le symbole \texttt{\&}.
			\item On termine une ligne avec \textbackslash\textbackslash, \textbf{à l'exception de la dernière ligne}.
			\item On insère une ligne horizontale pleine largeur entre deux lignes avec \cmd{hline}.
			\item La commande \cmd{multicolumn\{cols\}\{pos\}\{text\}} sert à «fusionner» les cellules d'une ligne.
			\begin{itemize}
				\scriptsize
				\item \texttt{cols} : étendue de la cellule en colonnes;
				\item \texttt{pos} : alignement du texte (\texttt{l},\texttt{c},\texttt{r});
				\item \texttt{text} : le contenu de la cellule.
			\end{itemize}
		\end{itemize}
	\end{onlyenv}
	
\end{frame}

% Tableaux (exemple concret)
\begin{frame}[fragile,c]{Les tableaux}
	\framesubtitle{Exemple concret}
\begin{codesource}
	\begin{tabularx}{\textwidth}{X|rrr|r|rrr}
		\textbf{\'{E}quipes}	&	\multicolumn{7}{c}{\textbf{Statistiques}} \\
		\hline\hline
		NFC North Team		&	W	&	L	&	T	&	PCT		&	PF	&	PA	&	Net Pts \\
		\hline
		Minnesota Vikings	&	13	&	3	&	0	&	.813	&	382	&	252	&	130 \\
		Detroit Lions		&	9	&	7	&	0	&	.563	&	410	&	376	&	34 \\
		Green Bay Packers	&	7	&	9	& 	0	&	.438	&	320	&	384	&	-64 \\
		Chicago Bears		&	5	&	11	&	0	&	.313	&	264	&	320	&	-56
	\end{tabularx}
\end{codesource}

	\begin{tabularx}{\textwidth}{X|rrr|r|rrr}
		\textbf{Équipes}	&	\multicolumn{7}{c}{\textbf{Statistiques}} \\
		\hline\hline
		NFC North Team		&	W	&	L	&	T	&	PCT		&	PF	&	PA	&	Net Pts \\
		\hline
		Minnesota Vikings	&	13	&	3	&	0	&	.813	&	382	&	252	&	130 \\
		Detroit Lions		&	9	&	7	&	0	&	.563	&	410	&	376	&	34 \\
		Green Bay Packers	&	7	&	9	& 	0	&	.438	&	320	&	384	&	-64 \\
		Chicago Bears		&	5	&	11	&	0	&	.313	&	264	&	320	&	-56
	\end{tabularx}
\end{frame}

\subsection{Figures}