% Principes de base

\section{Principes de base}

\subsection{Structure d'un document}

% Structure d'un document
\begin{frame}[fragile]{Structure d'un document}

	Un document \LaTeX\ est toujours composé de deux parties :
	
\begin{codesource}
	
	\documentclass[11pt,french]{article}
	\usepackage[utf8]{inputenc}
	\usepackage[T1]{fontenc}
	\usepackage{babel}
	\usepackage[autolanguage]{numprint}
	
	\begin{document}
		
		\section{Primo}
		
		Ac class dis donec erat facilisis magna mattis 
		placerat potenti praesent primis sed tellus turpis 
		ut vehicula. Ad amet eleifend eros fames habitant 
		imperdiet integer laoreet leo magna magnis neque 
		netus senectus taciti torquent. 
		
		\section{Deuxio}
		
		Cursus dui egestas eget eros et hac magna massa mollis 
		natoque penatibus sagittis sed tellus urna velit 
		vestibulum vitae vulputate. 
	\end{document}
\end{codesource}

	\begin{picture}(0,0)
		\thicklines\color{bleuFonceSecondaire}
		\onslide<2>\put(1,47){\dashbox{1}(87,15){}}
		\onslide<2>\put(89,53){\Large\textbf{\faArrowLeft\ Préambule}}
		\onslide<3>\put(1,6){\dashbox{1}(87,41){}}
		\onslide<3>\put(89,24){\Large\textbf{\faArrowLeft\ Corps du document}}
	\end{picture}
\end{frame}

% Préambule : la classe de document
\begin{frame}[fragile]{Préambule}
	\framesubtitle{La classe de document}
	La \textbf{première commande} du préambule est normalement la déclaration de la classe.
	
\begin{codesource}
	\documentclass[options]{classe}	
\end{codesource}

	\begin{columns}
		
		\pause
		
		\begin{HECcomparaison}{Principales classes}
			\begin{itemize}
				\item article, book, letter, report
				\item memoir, \textbf{hecthese}
				\item slides, beamer, \textbf{hecppt}
			\end{itemize}
		\end{HECcomparaison}
	
		\pause
		
		\begin{HECcomparaison}{Principales options}
			\begin{itemize}
				\item 10pt, 11pt, 12pt
				\item oneside, twoside
				\item openright, openany
				\item english, french
			\end{itemize}
		\end{HECcomparaison}
	\end{columns}
\end{frame}

% Préambule : les packages
\begin{frame}[fragile,c]{Préambule}
	\framesubtitle{Les \emph{packages}}
	Les \emph{packages} permettent de \textbf{modifier des commandes} ou d’\textbf{ajouter des fonctionnalités} au système.
	
	Ils sont chargés dans le préambule avec la commande \lstinline|\usepackage[options]{package}|.
	
\begin{codesource}
	\documentclass[options]{classe}
	
	\usepackage{package}
	\usepackage[options]{package}
	\usepackage{package1,package2,package3,...}
\end{codesource}
\end{frame}

% Commandes
\begin{frame}[fragile]{Commandes}
	\begin{itemize}
		\item Débutent toujours par un \textbackslash
		\item Formes générales:
\begin{codesource}
	\nomcommande[args_optionnels]{args_obligatoires}
	\nomcommande*[args_optionnels]{args_obligatoires}
	\nomcommande
\end{codesource}
		\item Arguments obligatoires entre \{\ et \}
		\item Arguments optionnels entre [ et ]
		\item Commande sans argument : le nom se termine par tout caractère qui n’est pas une lettre (y
		compris l’espace)
		\item Portée d’une commande limitée à la zone entre \{\ et \}.
	\end{itemize}
\end{frame}

% Environnements
\begin{frame}[fragile,c]{Environnements}
	\begin{itemize}
		\item Délimités par
\begin{codesource}
	\begin{environnement}
		...
	\end{environnement}
\end{codesource}
		\item Contenu de l’environnement traité différemment du reste du texte
		\item Changements s’appliquent uniquement à l’intérieur de l’environnement
	\end{itemize}
\end{frame}

\subsection{Rédaction}