% version-2
\section{Présentation de {\TeX} et \LaTeX}

% ------------ %
% Ce que c'est %
% ------------ %

\begin{frame}[c]

	\frametitle{Ce que c'est / ce que ce n'est pas}
		
			\begin{itemize}
				
				\item \TeX\ est un système de mise en	page (\emph{typesetting}) ou de
				préparation de documents.
				
				\item {\LaTeX} est un ensemble de macro-commandes pour faciliter l’utilisation de \TeX.
				
				\item Langage de balisage (\emph{Markup Language}) pour indiquer la	mise en forme
					du texte (pensez HTML).
					
				\item Accent mis sur la production de documents de grande qualité à la typographie
					soignée (surtout pour les mathématiques).
			\end{itemize}

\end{frame}

% ------------------- %
% Ce que ce n'est pas %
% ------------------- %

\begin{frame}

	\frametitle{Ce que ce n'est pas}
	
	\begin{itemize}
		
		\item Un traitement de texte
		\begin{itemize}
			\item priorité accordée à la qualité de la mise en page
		\end{itemize}
	
		\item WYSIWYG
		\begin{itemize}
			\item plutôt \emph{What You See Is What You Mean}
		\end{itemize}
	
		\item Incompatible
		\begin{itemize}
			\item format identique sur tous les systèmes d’exploitation
		\end{itemize}
	
		\item Instable
		\begin{itemize}
			\item noyau arrivé à maturité
			\item \TeX\ est aujourd’hui considéré essentiellement exempt de bogues
		\end{itemize}
	
		\item Imprévisible
		\begin{itemize}
			\item {\LaTeX} fait ce qu’on lui demande, ni plus, ni moins
		\end{itemize}
	\end{itemize}
\end{frame}

% ------------------ %
% Moteurs et formats %
% ------------------ %

\begin{frame}[c]

\frametitle{Moteurs et formats}

\begin{table}
	\begin{tabular}{rlcc}
		\hline\hline
						&	Moteur				&	Format			&	Fichier de sortie \\
		\hline
						&	tex					&	plain \TeX		&	DVI	\\
						&	tex (latex)			&	\LaTeX			&	DVI \\
		\faArrowRight	&	pdftex (pdflatex)	&	pdf\LaTeX		&	PDF \\
						&	xetex (xelatex)		&	\XeLaTeX		&	PDF \\
		\hline\hline
	\end{tabular}
\end{table}

\end{frame}

% ----------------------------------------- %
% Processus de création d'un document LaTeX %
% ----------------------------------------- %

\begin{frame}[c]
	\frametitle{Processus de création d'un document {\LaTeX}}
	\Huge
	\begin{minipage}[t]{0.25\linewidth}
		\centering
		\faFileTextO \\ \bigskip
		\footnotesize
		rédaction du texte et balisage avec un \emph{éditeur de texte}
	\end{minipage}
	\hfill\faArrowRight\hfill
	\begin{minipage}[t]{0.25\linewidth}
		\centering
		\faCogs \\  \bigskip
		\footnotesize
		compilation avec un \emph{moteur} {\TeX} depuis la ligne de commande
	\end{minipage}
	\hfill\faArrowRight\hfill
	\begin{minipage}[t]{0.25\linewidth}
		\centering
		\faFilePdfO \\  \bigskip
		\footnotesize
		visualisation avec visionneuse externe
	\end{minipage}
	
\end{frame}

% ------------------------------------------ %
% Qques choses simples à réaliser avec LaTeX %
% ------------------------------------------ %

\begin{frame}

	\frametitle{Quelques choses simples à réaliser avec {\LaTeX}}
	\framesubtitle{(et pas nécessairement avec un logiciel de traitement de texte)}
	
	\begin{itemize}
		\item Page de titre
		\item Table des matières
		\item Numérotation des pages
		\item Figures et tableaux : disposition sur la page, numérotation, renvois
		\item Équations mathématiques : disposition, numérotation et renvois
		\item Citations et composition de la bibliographie
		\item Coupure de mots
		\item Document recto verso
	\end{itemize}

\end{frame}

% ------------- %
% Distributions %
% ------------- %

\begin{frame}[c]

	\frametitle{Distributions}
	
	Le système {\LaTeX} est livré sous forme de \emph{distributions}.
	
	\begin{itemize}
		\item La bibliothèque recommande \href{https://www.tug.org/texlive}{\TeX Live}
		\item Mac OS : \href{https://www.tug.org/mactex}{Mac\TeX} (dérivée de \TeX Live)
		\item Une autre distribution a été testée avec \texttt{hecthese} :
			\href{https://miktex.org/download}{MiK\TeX}
	\end{itemize}
\end{frame}

% ---------- %
% Exercice 1 %
% ---------- %

\begin{frame}[c]

\frametitle{Exercice 1}

\begin{enumerate}
	\item TODO : refaire cet exercice avec baconipsum et à partir du fichier du premier exercice
	\item Démarrez votre éditeur de code intégré.
	\item Ouvrez et compilez le fichier \texttt{exercice\_minimal.tex}.
\end{enumerate}

\end{frame}