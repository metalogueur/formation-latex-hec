% Bibliographies et citations

\section{Bibliographies et citations}

\subsection{Types de bibliographies}

% Bibliographie manuelle
\begin{frame}[fragile,c]{Bibliographie manuelle}
	\begin{itemize}
		\item On peut se «tricoter» une bibliographie à la main avec l'environnement \texttt{thebibliography}.
\begin{codesource}
	\begin{thebibliography}{libellé le plus long}
		\bibitem[libellé]{id_citation} Entrée bibliographique #1
		\bibitem[libellé]{id_citation} Entrée bibliographique #2
		[...]
	\end{thebibliography}
\end{codesource}
		
		\pause
		
		\item Chaque entrée bibliographique est rédigée avec la commande \cmd{bibitem}.
		\begin{itemize}
			\scriptsize
			\item Le \texttt{libellé} est ce qu'on retrouvera dans la référence à l'intérieur du texte. S'il n'y a pas de libellé, \LaTeX\ produira un numéro séquentiel à la place.
			\item \texttt{id\_citation} est l'élément qu'on utilise pour citer une source.
			\item L'\texttt{entrée bibliographique} contient toutes les informations bibliographiques
				de notre source.
		\end{itemize}
	
		\pause
		
		\item Le \texttt{libellé le plus long} à l'ouverture correspond à celui des libellés de tous 
		les \cmd{bibitem} qui est le plus long.
		\item La bibliographie est insérée dans le document là où l'environnement 
			\texttt{thebibliography} est inséré dans le code.
	\end{itemize}
\end{frame}

\begin{frame}[fragile,c]{Bibliographie manuelle}
	\framesubtitle{Un exemple\ldots}
\begin{codesource}
	\begin{thebibliography}{99}		
		\bibitem[Kopka and Daly, 2004]{kopkadaly:2004}
			Kopka, Helmut et Patrick W. Daly (2004).
			\newblock Guide to \LaTeX, Fourth Edition,
			\newblock Addison-Wesley,
			\newblock ISBN 978-0-321-17385-0, 597 p.
		\bibitem[Mittelbach et al., 2004]{mittelbach:2004}
			Mittelbach, Frank \emph{et al.} (2004).
			\newblock The \LaTeX\ Companion, Second Edition,
			\newblock Addison-Wesley,
			\newblock ISBN 978-0201362992, 1120p.
		\bibitem[Goossens and Mittelbach, 2007]{goossens:2007}
			Goossens, Michel et Franck Mittelbach (2007).
			\newblock The \LaTeX\ Graphics Companion, Second Edition,
			\newblock Addison-Wesley,
			\newblock ISBN 978-0321508928, 976p.
	\end{thebibliography}
\end{codesource}
\end{frame}

\begin{frame}[c]{Bibliographie automatique}
	\framesubtitle{Une introduction à BiB\TeX}
	
	\begin{itemize}
		\item BiB\TeX\ est un programme (un compilateur) auxiliaire de \LaTeX\ qui construit 
			automatiquement une bibliographie à partir d'une base de données.
		\item Il est \emph{de facto} le système standard de traitement des bibliographies.
		\item Il est stable et simple à utiliser.
		\item C'est généralement le seul format accepté par les revues scientifiques.
		\item Vous pouvez exporter nos références bibliographiques stockées dans \textbf{EndNote}
			directement en format BiB\TeX.
		\item Vous pouvez télécharger des références en format BiB\TeX depuis HECo, Google Scholar,
			ProQuest, Ebsco et de nombreuses autres banques de données de la bibliothèque.
	\end{itemize}
\end{frame}

\begin{frame}[c]{Compilation d'un document avec BiB\TeX}
	\begin{itemize}
		\item À la formation précédente, nous avons schématisé la compilation d'un document comme suit:
	\end{itemize}
	{
		\begin{minipage}[t]{0.25\linewidth}
			\centering
			{\Large\faFileTextO} \\
			code source
		\end{minipage}
		\hfill{\Large\faArrowRight}\hfill
		\begin{minipage}[t]{0.25\linewidth}
			\centering
			{\Large\faCogs} \\
			pdf\LaTeX
		\end{minipage}
		\hfill{\Large\faArrowRight}\hfill
		\begin{minipage}[t]{0.25\linewidth}
			\centering
			{\Large\faFilePdfO} \\
			document .pdf
		\end{minipage}
	}
	
	\pause
	
	\begin{itemize}
		\item Avec BiB\TeX, la séquence de compilations change:		
	\end{itemize}

	{
		\begin{minipage}[t]{0.125\linewidth}
			\centering
			{\Large\faFileTextO} \\
			code source
		\end{minipage}
		\hfill{\Large\faArrowRight}\hfill
		\begin{minipage}[t]{0.125\linewidth}
			\centering
			{\Large\faCogs} \\
			pdf\LaTeX
		\end{minipage}
		\hfill{\Large\faArrowRight}\hfill
		\begin{minipage}[t]{0.125\linewidth}
			\centering
			{\Large\faCogs} \\
			BiB\TeX
		\end{minipage}
		\hfill{\Large\faArrowRight}\hfill
		\begin{minipage}[t]{0.125\linewidth}
			\centering
			{\Large\faCogs} \\
			pdf\LaTeX
		\end{minipage}
		\hfill{\Large\faArrowRight}\hfill
		\begin{minipage}[t]{0.125\linewidth}
			\centering
			{\Large\faCogs} \\
			pdf\LaTeX
		\end{minipage}
		\hfill{\Large\faArrowRight}\hfill
		\begin{minipage}[t]{0.125\linewidth}
			\centering
			{\Large\faFilePdfO} \\
			document .pdf
		\end{minipage}
	}
\end{frame}

\subsection{Création d'une bibliographie}

\begin{frame}[fragile]{Création d'une base de données}
	La première chose à faire est de se créer une base de données de références qu'on stockera dans un fichier \texttt{.bib}.
	
\begin{codesource}
	% Exemple de contenu d'un fichier bibliographie.bib
	
	@article{amaralcardiac2014,
		author = {Amaral, Joice Anaize Tonon do and Nogueira, Marcela Leme and Roque, Adriano L 
			and Guida, Heraldo Lorena and Abreu, Luiz Carlos de and Raimundo, Rodrigo Daminello 
			and Vanderlei, Luiz Carlos Marques and Ribeiro, Vivian F and Ferreira, Celso and 
			Valenti, Vitor Engrácia},
		title = {Cardiac autonomic regulation during exposure to auditory stimulation with classical
			baroque or heavy metal music of different intensities},
		journal = {Archives of the Turkish Society of Cardiology},
		pages = {139-146},
		ISSN = {1016-5169},
		year = {2014},
		type = {Journal Article}
	}
	
	@article{mobergfaster2009,
		author = {Moberg, Marcus},
		title = {Faster for the master!: exploring issues of religious expression and alternative
			Christian identity within the Finnish Christian metal music scene},
		year = {2009},
		type = {Journal Article}
	}
\end{codesource}
\end{frame}

% Package natbib
\begin{frame}[fragile]{\emph{Package} natbib}
	\begin{itemize}
		\item Par défaut, \LaTeX\ ne supporte que les bibliographies avec un format de citation numérique.
		\item Le format de citations adopté dans les sciences sociales en général, et à HEC Montréal en particulier, est le format \emph{auteur, année}.
		\item Le \emph{package} \textbf{natbib} permet l'utilisation des citations \emph{auteur, année}.

		\pause
		
\begin{codesource}
	\documentclass[english,french]{hecthese}
	
	\usepackage[utf8]{inputenc}
	\usepackage[T1]{fontenc}
	\usepackage{babel}
	\usepackage[autolanguage]{numprint}
	\usepackage{icomma}
	\usepackage{natbib}
	\usepackage{hyperref}
	
	\begin{document}
		contenu...
	\end{document}
\end{codesource}
		\item natbib doit \textbf{absolument} être appelé \textbf{après} babel.
	\end{itemize}
\end{frame}

% Insertion de la bibliographie
\begin{frame}[fragile]{Insertion de la bibliographie}
	\begin{itemize}
		\item Avant d'insérer la bibliographie dans le texte, il faut signifier à BiB\TeX\ dans quel
		style bibliographique nous voulons que nos références s'affichent.
\begin{codesource}
	\bibliographystyle{style}
\end{codesource}
		
		\pause
		
		\item Ce ne sont pas tous les styles bibliographiques qui sont compatibles avec le format de 
			citation \emph{auteur, année}.
			\begin{itemize}
				\scriptsize
				\item Utilisez le style \texttt{francais} si vous rédigez en français;
				\item Utilisez le style \texttt{apalike} si vous rédigez en anglais.
\begin{codesource}
	% Rédaction en français
	\bibliographystyle{francais}
	
	% Rédaction en anglais
	\bibliographystyle{apalike}
\end{codesource}
				\item Ces deux styles sont ceux qui se rapprochent le plus de celui de HEC Montréal.
			\end{itemize}
		
		\pause
		
		\item Une fois qu'on a choisi notre style, on insère la bibliographie.
\begin{codesource}
	\bibliographystyle{francais}
	\bibliography{fichier_bib} % Nom du fichier .bib entre accolades, sans l'extension .bib
\end{codesource}
	\end{itemize}
\end{frame}

\subsection{Citations}

% Comment citer ses sources
\begin{frame}[c]{Comment citer ses sources}
	\begin{itemize}
	\item Il existe trois commandes pour citer des sources bibliographiques dans le texte, dont deux 	
		proviennent de \textbf{natbib} :
	
	\begin{description}[aaaaaaaaaaaaaaaaaaaaaaaaaaa]
		\item[\cmd{cite[extra]\{id\_citation\}}] Citation numérique
		\item[\cmd{citet[extra]\{id\_citation\}}] Citation textuelle
		\item[\cmd{citep[extra]\{id\_citation\}}] Citation entre parenthèses
	\end{description}
	
	\item L'argument \texttt{id\_citation} est celui qu'on a utilisé pour identifier un item
		bibliographique.
	\item L'argument optionnel \texttt{extra} permet d'ajouter des informations supplémentaires
		à la suite d'une situation comme, par exemple, un numéro de page.
	\item Nous vous recommandons d'utiliser les commandes \cmd{citet} et \cmd{citep}, qui sont plus descriptives.
	\end{itemize}
\end{frame}

% Exemples de citations
\begin{frame}[fragile]{Exemples de citation}
	
	Soit l'item bibliographique suivant:
\begin{codesource}
	\bibitem{jones99}
		F. J. Jones, H. P. Baker, and W. V. Toms, [...] 1999.
\end{codesource}

	Voici un aperçu de ce que donnerait chaque commande de citation:
	
	\pause
	
	\begin{columns}
		\column{.49\textwidth}
		\vspace{-4.5mm}
\begin{codesource}
	Je ne suis pas peu fier de voir que quelqu'un
	pense comme moi\cite{jones99}\ldots
\end{codesource}
		\column{.49\textwidth}
			Je ne suis pas peu fier de voir que quelqu'un
			pense comme moi[1]\ldots
	\end{columns}

	\begin{columns}
		\column{.49\textwidth}
		\vspace{-4.5mm}
\begin{codesource}
	Je ne suis pas peu fier de voir que quelqu'un
	pense comme moi\cite[p.22]{jones99}\ldots
\end{codesource}
		\column{.49\textwidth}
			Je ne suis pas peu fier de voir que quelqu'un
			pense comme moi[1, p.22]\ldots
		\end{columns}
	
	\pause

	\begin{columns}
		\column{.49\textwidth}
		\vspace{-4.5mm}
\begin{codesource}
	Je ne suis pas peu fier de voir que 
	\citet{jones99}pense comme moi\ldots
\end{codesource}
		\column{.49\textwidth}
			Je ne suis pas peu fier de voir que Jones et al., (1999) pense comme moi\ldots
		\end{columns}

		\begin{columns}
		\column{.49\textwidth}
		\vspace{-4.5mm}
\begin{codesource}
	Je ne suis pas peu fier de voir que 
	\citet[p.22]{jones99}pense comme moi\ldots
\end{codesource}
		\column{.49\textwidth}
			Je ne suis pas peu fier de voir que Jones et al., (1999, p.22) pense comme moi\ldots
		\end{columns}
	
		\pause
		
		\begin{columns}
			\column{.49\textwidth}
			\vspace{-4.5mm}
\begin{codesource}
	Je ne suis pas peu fier de voir que quelqu'un
	pense comme moi\citep{jones99}\ldots
\end{codesource}
			\column{.49\textwidth}
			Je ne suis pas peu fier de voir que quelqu'un pense comme moi (Jones et al., 1999)\ldots
		\end{columns}
	
		\begin{columns}
			\column{.49\textwidth}
			\vspace{-4.5mm}
\begin{codesource}
	Je ne suis pas peu fier de voir que quelqu'un
	pense comme moi\citep[p.22]{jones99}\ldots
\end{codesource}
			\column{.49\textwidth}
			Je ne suis pas peu fier de voir que quelqu'un pense comme moi (Jones et al., 1999, p.22)\ldots
		\end{columns}
\end{frame}