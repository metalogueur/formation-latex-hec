% Bibliographies et citations

\section{Bibliographies and citations}

\subsection{Types of bibliographies}

% Bibliographie manuelle
\begin{frame}[fragile,c]{Manual bibliographies}
	\begin{itemize}
		\item We can manually write our bibliography with the \texttt{thebibliography} environment.
\begin{codesource}
	\begin{thebibliography}{longest label}
		\bibitem[label]{id_citation} Bibliographic entry #1
		\bibitem[label]{id_citation} Bibliographic entry #2
		[...]
	\end{thebibliography}
\end{codesource}
		
		\pause
		
		\item Each bibliographic entry is written with the \cmd{bibitem} command.
		\begin{itemize}
			\scriptsize
			\item The \texttt{label} is what we'll find as reference in the text. If there is no label, \LaTeX\ will insert a sequential number.
			\item \texttt{id\_citation} is what is used to cite a bibliographic entry.
			\item The \texttt{bibliographic entry} contains all information concerning the source.
		\end{itemize}
	
		\pause
		
		\item The \texttt{longest label} at the beginning of the environment is the longest of all labels found in the \texttt{bibitem}s.
		\item The bibliography is inserted in the document where the \texttt{thebibliography} environment has been inserted in the code.
	\end{itemize}
\end{frame}

\begin{frame}[fragile,c]{Manual bibliographies}
	\framesubtitle{Example}
\begin{codesource}
	\begin{thebibliography}{99}		
		\bibitem[Kopka and Daly, 2004]{kopkadaly:2004}
			Kopka, Helmut and Patrick W. Daly (2004).
			\newblock Guide to \LaTeX, Fourth Edition,
			\newblock Addison-Wesley,
			\newblock ISBN 978-0-321-17385-0, 597 p.
		\bibitem[Mittelbach et al., 2004]{mittelbach:2004}
			Mittelbach, Frank \emph{et al.} (2004).
			\newblock The \LaTeX\ Companion, Second Edition,
			\newblock Addison-Wesley,
			\newblock ISBN 978-0201362992, 1120p.
		\bibitem[Goossens and Mittelbach, 2007]{goossens:2007}
			Goossens, Michel and Franck Mittelbach (2007).
			\newblock The \LaTeX\ Graphics Companion, Second Edition,
			\newblock Addison-Wesley,
			\newblock ISBN 978-0321508928, 976p.
	\end{thebibliography}
\end{codesource}
\end{frame}

\begin{frame}[c]{Automatic Bibliographies}
	\framesubtitle{An introduction to BiB\TeX}
	
	\begin{itemize}
		\item BiB\TeX\ is a \LaTeX\ auxiliary program (compiler) that automatically builds a bibliography using a database.
		\item It is the \emph{de facto} standard system for building bibliographies.
		\item It is stable and simple to use.
		\item It usually is the only format accepted by scientific journals.
		\item You can export your bibliographic entries from \textbf{EndNote} to BiB\TeX.
		\item You can download references in BiB\TeX\ format from HECo, Google Scholar,
			ProQuest, Ebsco and many more databases found at the Library.
	\end{itemize}
\end{frame}

\begin{frame}[c]{Compiling a document with BiB\TeX}
	\begin{itemize}
		\item In the previous training session, we have schematized a document's compilation as such:
	\end{itemize}
	{
		\begin{minipage}[t]{0.25\linewidth}
			\centering
			{\Large\faFileTextO} \\
			source code
		\end{minipage}
		\hfill{\Large\faArrowRight}\hfill
		\begin{minipage}[t]{0.25\linewidth}
			\centering
			{\Large\faCogs} \\
			pdf\LaTeX
		\end{minipage}
		\hfill{\Large\faArrowRight}\hfill
		\begin{minipage}[t]{0.25\linewidth}
			\centering
			{\Large\faFilePdfO} \\
			.pdf document
		\end{minipage}
	}
	
	\pause
	
	\begin{itemize}
		\item With BiB\TeX, the compilation sequence changes:		
	\end{itemize}

	{
		\begin{minipage}[t]{0.125\linewidth}
			\centering
			{\Large\faFileTextO} \\
			source code
		\end{minipage}
		\hfill{\Large\faArrowRight}\hfill
		\begin{minipage}[t]{0.125\linewidth}
			\centering
			{\Large\faCogs} \\
			pdf\LaTeX
		\end{minipage}
		\hfill{\Large\faArrowRight}\hfill
		\begin{minipage}[t]{0.125\linewidth}
			\centering
			{\Large\faCogs} \\
			BiB\TeX
		\end{minipage}
		\hfill{\Large\faArrowRight}\hfill
		\begin{minipage}[t]{0.125\linewidth}
			\centering
			{\Large\faCogs} \\
			pdf\LaTeX
		\end{minipage}
		\hfill{\Large\faArrowRight}\hfill
		\begin{minipage}[t]{0.125\linewidth}
			\centering
			{\Large\faCogs} \\
			pdf\LaTeX
		\end{minipage}
		\hfill{\Large\faArrowRight}\hfill
		\begin{minipage}[t]{0.125\linewidth}
			\centering
			{\Large\faFilePdfO} \\
			.pdf document
		\end{minipage}
	}
\end{frame}

\subsection{Creating a Bibliography}

\begin{frame}[fragile]{Creating a Database}
	The first thing to do is to create a database of references that is going to be stored in a \texttt{.bib} file.
	
\begin{codesource}
	% Example taken from bibliography.bib
	
	@article{amaralcardiac2014,
		author = {Amaral, Joice Anaize Tonon do and Nogueira, Marcela Leme and Roque, Adriano L 
			and Guida, Heraldo Lorena and Abreu, Luiz Carlos de and Raimundo, Rodrigo Daminello 
			and Vanderlei, Luiz Carlos Marques and Ribeiro, Vivian F and Ferreira, Celso and 
			Valenti, Vitor Engrácia},
		title = {Cardiac autonomic regulation during exposure to auditory stimulation with classical
			baroque or heavy metal music of different intensities},
		journal = {Archives of the Turkish Society of Cardiology},
		pages = {139-146},
		ISSN = {1016-5169},
		year = {2014},
		type = {Journal Article}
	}
	
	@article{mobergfaster2009,
		author = {Moberg, Marcus},
		title = {Faster for the master!: exploring issues of religious expression and alternative
			Christian identity within the Finnish Christian metal music scene},
		year = {2009},
		type = {Journal Article}
	}
\end{codesource}
\end{frame}

% Package natbib
\begin{frame}[fragile]{natbib Package}
	\begin{itemize}
		\item By default, \LaTeX\ only supports numerical citations.
		\item The citation format used in science in general, and at HEC Montréal particularly, is the \emph{author, year} format.
		\item The \textbf{natbib} package allows the use of the \emph{author, year} format.

		\pause
		
\begin{codesource}
	\documentclass[english,french]{hecthese}
	
	\usepackage[utf8]{inputenc}
	\usepackage[T1]{fontenc}
	\usepackage{babel}
	\usepackage[autolanguage]{numprint}
	\usepackage{icomma}
	\usepackage{natbib}
	\usepackage{hyperref}
	
	\begin{document}
		content...
	\end{document}
\end{codesource}
		\item natbib must \textbf{absolutely} be loaded \textbf{after} babel.
	\end{itemize}
\end{frame}

% Insertion de la bibliographie
\begin{frame}[fragile]{Inserting a Bibliography}
	\begin{itemize}
		\item Before inserting our bibliography in our document, we have to tell BiB\TeX\ in which bibliographic style we want our references to be displayed.
\begin{codesource}
	\bibliographystyle{style}
\end{codesource}
		
		\pause
		
		\item Not all bibliographic styles are compatible with the \emph{author, year} citation format. 
			\begin{itemize}
				\scriptsize
				\item Use the \texttt{francais} style if you write in French;
				\item Use the \texttt{apalike} style if you write in English.
\begin{codesource}
	% Writing in French
	\bibliographystyle{francais}
	
	% Writing in English
	\bibliographystyle{apalike}
\end{codesource}
				\item These two styles resemble most HEC Montréal's style.
			\end{itemize}
		
		\pause
		
		\item Once we have chosen our bibliographic style, we can insert our bibliography.
\begin{codesource}
	\bibliographystyle{apalike}
	\bibliography{bibfile} % Name of .bib file between curly braces, with the file extension
\end{codesource}
	\end{itemize}
\end{frame}

\subsection{Citations}

% Comment citer ses sources
\begin{frame}[c]{Referring to sources}
	\begin{itemize}
	\item There are three ways to cite bibliographic entries, two of them coming from the \textbf{natbib} package:
	
	\begin{description}[aaaaaaaaaaaaaaaaaaaaaaaaaaa]
		\item[\cmd{cite[extra]\{id\_citation\}}] Numerical citation
		\item[\cmd{citet[extra]\{id\_citation\}}] Inline citation
		\item[\cmd{citep[extra]\{id\_citation\}}] Citation between parentheses
	\end{description}
	
	\item The \texttt{id\_citation} argument is what is used to identify a bibliographic entry.
	\item The optional \texttt{extra} argument allows us to insert extra information after the citation, e.g. a page number.
	\item We advise you to use the \cmd{citet} and \cmd{citep} commands, which are more descriptive.
	\end{itemize}
\end{frame}

% Exemples de citations
\begin{frame}[fragile]{Citation examples}
	
	Look at the following bibliographic entry:
\begin{codesource}
	\bibitem{jones99}
		F. J. Jones, H. P. Baker, and W. V. Toms, [...] 1999.
\end{codesource}

	This is what each of the citation commands' output looks like :
	
	\pause
	
	\begin{columns}
		\column{.49\textwidth}
		\vspace{-4.5mm}
\begin{codesource}
	I am so proud that someone thinks exactly
	like me\cite{jones99}\ldots
\end{codesource}
		\column{.49\textwidth}
				I am so proud that someone thinks exactly
			like me[1]\ldots
	\end{columns}

	\begin{columns}
		\column{.49\textwidth}
		\vspace{-4.5mm}
\begin{codesource}
		I am so proud that someone thinks exactly
	like me\cite[p.22]{jones99}\ldots
\end{codesource}
		\column{.49\textwidth}
				I am so proud that someone thinks exactly
			like me[1, p.22]\ldots
		\end{columns}
	
	\pause

	\begin{columns}
		\column{.49\textwidth}
		\vspace{-4.5mm}
\begin{codesource}
		I am so proud that \citet{jones99}
		thinks exactly like me\ldots
\end{codesource}
		\column{.49\textwidth}
			I am so proud that Jones et al., (1999) thinks exactly like me\ldots
		\end{columns}

		\begin{columns}
		\column{.49\textwidth}
		\vspace{-4.5mm}
\begin{codesource}
	I am so proud that \citet[p.22]{jones99}
	thinks exactly like me\ldots
\end{codesource}
		\column{.49\textwidth}
			I am so proud that Jones et al., (1999, p.22) thinks exactly like me\ldots
		\end{columns}
	
		\pause
		
		\begin{columns}
			\column{.49\textwidth}
			\vspace{-4.5mm}
\begin{codesource}
	I am so proud that someone thinks exactly
	like me\citep{jones99}\ldots
\end{codesource}
			\column{.49\textwidth}
			I am so proud that someone thinks exactly
			like me (Jones et al., 1999)\ldots
		\end{columns}
	
		\begin{columns}
			\column{.49\textwidth}
			\vspace{-4.5mm}
\begin{codesource}
	I am so proud that someone thinks exactly
	like me\citep[p.22]{jones99}\ldots
\end{codesource}
			\column{.49\textwidth}
			I am so proud that someone thinks exactly
			like me (Jones et al., 1999, p.22)\ldots
		\end{columns}
\end{frame}