\section{Writing}

\subsection{Basic formatting}

% Titre, auteur et date du document
\begin{frame}[fragile]{Title, author and date}
	\begin{itemize}
		\item Automatic formatting
\begin{codesource}
	\documentclass{article}
	
	\title{Document title}
	\author{Author name}
	\date{A date}
	
	\begin{document}
		\maketitle
		
		% Document content...
	\end{document}
\end{codesource}
		\item Manual formatting
\begin{codesource}
	\documentclass{article}
	
	\begin{document}
		\begin{titlepage}
			% Title page built manually...
		\end{titlepage}
	\end{document}
\end{codesource}
	\end{itemize}
\end{frame}

% Paragraphes, sauts de lignes et espaces blancs
\begin{frame}[c]{Paragraphs, line breaks and white space}
	\begin{itemize}
		\item \LaTeX\ automatically deletes all extra white spaces.
		\item Line breaks are created with \textbackslash\textbackslash.
		\item There needs to be at least one blank line between paragraphs in
			the code in order to distinguish them in the text.
	\end{itemize}
\end{frame}

% Caractères spéciaux
\begin{frame}{Reserved characters}
	\framesubtitle{\TeX\ reserved characters}
	\begin{description}[\#]
		\item[\#] Argument identifier in commands
		\item[\$] Math mode delimiter
		\item[\&] Column delimiter in tables
		\item[\%] Start of a comment
		\item[\_] Indice (math)
		\item[\textasciicircum] Exponent (math)
		\item[\textasciitilde] No-break space
		\item[\{] Opens a command or environment definition
		\item[\}] Closes a command or environment definition
	\end{description}
	\begin{picture}(0,0)
	\thicklines\color{bleuFonceSecondaire}
	\onslide<2>\put(90,5){\dashbox{1}(53,58){}}
	\onslide<2>\put(97,59){\textbf{\MakeUppercase{To use them:}}}
	\onslide<2>\put(94,55){\parbox[t]{.3\textwidth}{\centering\bfseries\textbackslash \# \\[5pt] %
			\textbackslash \$ \\[5pt] \textbackslash \& \\[5pt] \textbackslash \% \\[5pt] %
			\textbackslash \_ \\[5pt] \textbackslash textasciicircum \\[4pt] %
			\textbackslash textasciitilde \\[4pt] \textbackslash \{ \\[4pt] %
			\textbackslash \} }}
	\end{picture}
\end{frame}

% Caractères spéciaux - la suite
\begin{frame}[fragile]{Reserved characters}
	\framesubtitle{Part deux\ldots}
	\begin{itemize}
		\item Quote marks
			\begin{itemize}
				\item We open english single quotes with (\lstinline|`|)
					and double quotes with (\lstinline|``|). We close them with one
					(\lstinline|'|) or two (\lstinline|''|) apostrophes, depending on the case.
				\item On utilise les chevrons (« et ») pour ouvrir et fermer les guillemets français.
					Il faut cependant inscrire la commande suivante à la fin de notre préambule:
\begin{codesource}
	\frenchbsetup{og=«,fg=»}
\end{codesource}				
			\end{itemize}
		\item On inscrit les traits d'union avec un tiret (\lstinline|-|), les traits demi-cadratins avec deux tirets (\lstinline|--|) et les traits cadratins avec trois tirets (\lstinline|---|).
	\end{itemize}
\end{frame}

% Commentaires
\begin{frame}[c]{Commentaires}
\begin{itemize}
\item Pour se retrouver dans le code (ou des documents longs), il est conseillé 
d'y insérer des commentaires.
\item Ceux-ci commencent toujours avec le symbole \texttt{\%}.
\item Ils s'affichent dans le code, mais pas dans le document final.
\end{itemize}
\end{frame}

\subsection{Apparence du texte}

% Polices de caractères
\begin{frame}[c]{Polices de caractères}
	\begin{itemize}
		\item Par défaut, tous les documents \LaTeX\ utilisent la même police, \textrm{Computer Modern}.
		\item Privilégier les polices de grande qualité et très complètes (lettres accentuées, grand choix de symboles)
		\item Peu de polices sont adaptées pour les mathématiques : Palatino, Times, Lucida (\$) sont des choix sûrs
		\item Dans la classe \textbf{hecthese}, les paquetages mathptmx et mathpazo sont chargés par défaut afin	d’offrir les polices de caractères Times et Palatino.
	\end{itemize}
\end{frame}

% Changement d'attribut de la police
\begin{frame}{Changement d'attribut de la police}
	\begin{tabularx}{\textwidth}{XXX}
		\arrayrulecolor{grisPrimaire!40}\hline\hline
		\multicolumn{3}{l}{\textbf{familles}}	\\
		\hline
		\textrm{romain}						&	\cmd{rmfamily}		&	\cmd{textrm\{<texte>\}}\\
		\texttt{largeur fixe}				&	\cmd{ttfamily}		&	\cmd{texttt\{<texte>\}}\\
		sans empattements					&	\cmd{sffamily}		&	\cmd{textsf\{<texte>\}}\\
		\hline
		\multicolumn{3}{l}{\textbf{formes}}	\\
		\hline
		droit								&	\cmd{upshape}		&	\cmd{textup\{<texte>\}}\\
		\emph{italique}						&	\cmd{itshape}		&	\cmd{textit\{<texte>\}}\\
		\textsl{penché}						&	\cmd{slshape}		&	\cmd{textsl\{<texte>\}}\\
		\textrm{\textsc{petites capitales}}	&	\cmd{scshape}		&	\cmd{textsc\{<texte>\}}\\
		\hline
		\multicolumn{3}{l}{\textbf{séries}}	\\
		\hline
		\textmd{moyen}						&	\cmd{mdseries}		&	\cmd{textmd\{<texte>\}}\\
		\textbf{gras}						&	\cmd{bfseries}		&	\cmd{textbf\{<texte>\}}\\
		\hline\hline
	\end{tabularx}

	\begin{picture}(0,0)
		\thicklines\color{bleuFonceSecondaire}
		\onslide<2>\put(38,-7){\dashbox{1}(40,64)[b]{\parbox{.25\textwidth}{\centering\textbf{s'applique à tout le texte qui suit}}}}
		\onslide<3>\put(92,-7){\dashbox{1}(40,64)[b]{\parbox{.25\textwidth}{\centering\textbf{s'applique au texte en argument}}}}
	\end{picture}
\end{frame}

% Taille de la police
\begin{frame}{Taille de la police}
	\begin{tabularx}{\textwidth}{l|l}
		\arrayrulecolor{grisPrimaire!40}
		\textbf{Commandes standards} 	& 	\textbf{Rendu}	\\
		\hline
		\cmd{tiny}						&	{\tiny vraiment petit}	\\
		\cmd{scriptsize}				&	{\scriptsize encore plus petit}	\\
		\cmd{footnotesize}				&	{\footnotesize plus petit}	\\
		\cmd{small}						&	{\small petit}	\\
		\cmd{normalsize}				&	{\normalsize normal}	\\
		\cmd{large}						&	{\large grand}	\\
		\cmd{Large}						&	{\Large plus grand}	\\
		\cmd{LARGE}						&	{\LARGE encore plus grand}	\\
		\cmd{huge}						&	{\huge énorme}	\\
		\cmd{Huge}						&	{\Huge encore plus énorme}		
	\end{tabularx}
\end{frame}

% Caractères gras, italiques et soulignés
\begin{frame}[c]{Caractères gras, italiques et soulignés}
	\begin{itemize}
		\item Caractères \textbf{gras} : \cmd{textbf\{\}}
		\item Caractères \emph{italiques} :
		\begin{itemize}
			\item \cmd{textit\{\}}
			\item \cmd{emph\{\}} -- commande à privilégier
		\end{itemize}
		\item Caractères \underline{soulignés} : \cmd{underline\{\}}
	\end{itemize}
\end{frame}

