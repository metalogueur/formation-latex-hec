\section{Disposition du texte}

% Alignement du texte
\begin{frame}[fragile]{Alignement du texte}
	\begin{itemize}
		\item Par défaut, le texte est pleinement justifié.
		\item Pour aligner le texte à gauche, on utilise l'environnement \texttt{flushleft}.
\begin{codesource}
\begin{flushleft}
	Le texte sera aligné à gauche.
\end{flushleft}
\end{codesource}
		\item On utilise l'environnement \texttt{center} pour centrer le texte.
\begin{codesource}
	\begin{center}
		Le texte sera centré.
	\end{center}
\end{codesource}
		\item Pour aligner le texte à droite, on utilise l'environnement \texttt{flushright}.
\begin{codesource}
	\begin{flushright}
		Le texte sera aligné à droite.
	\end{flushright}
\end{codesource}
	\end{itemize}
\end{frame}

% Listes
\begin{frame}[fragile]{Listes}
	\framesubtitle{Listes à puces et listes numérotées}
	\begin{itemize}
		\item Les listes à puces sont construites avec l'environnement \texttt{itemize}.
\begin{codesource}
	\begin{itemize}
		\item Premier item
		\item Deuxième item
		\item etc.
	\end{itemize}
\end{codesource}
		\item Les listes numérotées sont construites avec l'environnement \texttt{enumerate}.
\begin{codesource}
	\begin{enumerate}
		\item Premier item
		\item Deuxième item
		\item etc.
	\end{enumerate}
\end{codesource}
		\item La commande \cmd{item} est utilisée pour lister les items.
		\item On peut imbriquer jusqu'à quatre niveaux de listes.
	\end{itemize}
\end{frame}

\begin{frame}[c, fragile]{Listes}
	\framesubtitle{Listes de définitions}
	
	On crée une liste de définitions avec l'environnement \texttt{description}.
	
\begin{codesource}
	\begin{description}
		\item[Premier terme] Définition du premier terme.
		\item[Deuxième terme] Définition du deuxième terme.
	\end{description}
\end{codesource}

	\begin{description}
		\item[Premier terme] Définition du premier terme. Auctor est gravida habitasse leo lobortis mollis nec platea posuere
		 sollicitudin tempus.
		\item[Deuxième terme] Définition du deuxième terme. Aenean consequat dictumst dignissim duis facilisis himenaeos id
		 pharetra placerat porta posuere primis senectus tortor.
	\end{description}
\end{frame}

% Citations
\begin{frame}[fragile,c]{Citations}
\framesubtitle<1>{Citations courtes}
\framesubtitle<2>{Citations longues}
\begin{onlyenv}<1>
	On utilise l'environnement \texttt{quote} pour insérer une citation courte (un paragraphe)
	dans le texte.
	
	\begin{columns}
		\begin{column}{.49\textwidth}
			\vspace{-17mm}
\begin{codesource}
	\begin{quote}
		Life is what happens to you while 
		you're busy making other plans. 
		-- John Lennon
	\end{quote}
\end{codesource}
		\end{column}
		
		\begin{column}{.49\textwidth}
			\begin{quote}
				Life is what happens to you while you're busy making other plans. -- John Lennon
			\end{quote}
		\end{column}
	\end{columns}
\end{onlyenv}

\begin{onlyenv}<2>
	On utilise l'environnement \texttt{quotation} pour insérer une citation longue (plus
	d'un paragraphe).
	
	\begin{columns}
		\begin{column}{.49\textwidth}
			\vspace{-38mm}
\begin{codesource}
	\begin{quotation}
		I've missed more than 9000 shots in my 
		career. I've lost almost 300 games. 26 
		times I've been trusted to take the game 
		winning shot and missed.
		
		I've failed over and over and over again 
		in my life. And that is why I succeed. 
		-- Michael Jordan
	\end{quotation}
\end{codesource}	
		\end{column}
		
		\begin{column}{.49\textwidth}
			\begin{quotation}
				I've missed more than 9000 shots in my career. 
				I've lost almost 300 games. 26 times I've been 
				trusted to take the game winning shot and missed.
				
				I've failed over and over and over again in my life. 
				And that is why I succeed. -- Michael Jordan
			\end{quotation}
		\end{column}
	\end{columns}
\end{onlyenv}
\end{frame}

% Notes de bas de page
\begin{frame}[fragile,c]{Notes de bas de page}
\begin{itemize}
\item Une note de bas de page est insérée avec la commande suivante:
\begin{codesource}
	\footnote{texte de la note}
\end{codesource}
\item La commande doit suivre immédiatement le texte à annoter.
\item Méthode recommandée :
\begin{codesource}
	... fera remarquer que Pierre Lasou\footnote{%
		Spécialiste en ressources documentaires} %
	fut une grande aide dans la préparation de ...
\end{codesource}
\item La numérotation et la disposition sont automatiques.
\end{itemize}
\end{frame}

% Code source
\begin{frame}[fragile,c]{Code source}
\begin{onlyenv}<1>
\begin{itemize}
\item Pour rédiger du code source en bloc, on utilise l'environnement \texttt{verbatim}
\begin{codesource}
	\begin{verbatim}
        Texte disposé tel qu'il est saisi
        dans une police à largeur fixe.
	\end{verbatim}
\end{codesource}
\item Pour rédiger du code source à même le texte, on utilise la commande \cmd{verb}, dont la
syntaxe est \cmd{verbcsourcec} où \emph{c} est un caractère quelconque ne se trouvant pas dans \emph{source}.
\begin{codesource}
	Du texte avec \verb|du code|.
\end{codesource}
\item Pour un usage plus intensif, consultez la documentation du \emph{package} \textbf{listings}.
\end{itemize}
\end{onlyenv}

\begin{onlyenv}<2>
Un exemple\footnote{tiré du site \href{http://r4stats.com/examples/programming/}{r4stats.com}.} :
\begin{codesource}
# ---Writing Your Own Functions (Macros)---

# A good function that just prints.
mystats <- function(x) {
	print( mean(x, na.rm = TRUE) )
	print(   sd(x, na.rm = TRUE) )
}
mystats(myvar)

# A function with vector output.
mystats  <- function(x) {
	mymean <- mean(x, na.rm = TRUE)
	mysd   <-   sd(x, na.rm = TRUE)
	c(mean = mymean, sd = mysd )
}
mystats(myvar)
myVector <- mystats(myvar)
myVector
\end{codesource}
\end{onlyenv}
\end{frame}