% Organisation d'un document

\section{Document organization}

\subsection{Parts of a document}

% Choix d'une classe
\begin{frame}[c]{Document class choice}
	The first thing to do before we start writing a \LaTeX\ document is to choose
	a document class.
	
	\begin{table}[c]
		\begin{tabularx}{\textwidth}{lllll}
			\arrayrulecolor{grisPrimaire!40}\hline\hline
			\textbf{Class} & \textbf{Divisions} & \textbf{Layout} & \textbf{Headers} &	\textbf{Footers} \\
			\hline
			\texttt{article}			&	parts, sections, \ldots					&	one-sided	&	empty			&	centered page numbers\\
			\texttt{report}				&	parts, chapters, sections, \ldots		&	one-sided	&	empty			&	centered page numbers\\
			\texttt{book}				&	parts, chapters, sections, \ldots		&	two-sided	& 	page numbers, titles	&	empty \\
			\texttt{hecthese}			&	chapters, sections, subsections			&	two-sided	&	empty			&	centered page numbers \\
			\hline\hline
		\end{tabularx}
	\end{table}
\end{frame}

% Résumé
\begin{frame}[fragile,c]{Abstract}
	\begin{itemize}
		\item \textbf{article}, \textbf{report} or \textbf{memoir} classes: the abstract is created with
		the \lstinline|abstract| environment.
\begin{codesource}
	\begin{abstract}
		...
	\end{abstract}
\end{codesource}

		\item \textbf{hecthese} class: french and english abstracts considered as normal unnumbered
			chapters
	\end{itemize}
\end{frame}

% Sections
\begin{frame}[fragile]{Sections}
	\begin{itemize}
		\item A document is divided in sections with the following commands:
\begin{codesource}
	\part[short title]{long title}
	\chapter[short title]{long title}
	\section[short title]{long title}
	\subsection[short title]{long title}
	
	\subsubsection[short title]{long title} 	% à éviter dans un livre
	
	\paragraph[short title]{long title} 		% ne jamais utiliser
	\subparagraph[short title]{long title} 	% ne jamais JAMAIS utiliser
\end{codesource}

		\item Automatic numbering
		\item Commands followed by an * = unnumbered section
		\item Short title is optional
	\end{itemize}
\end{frame}

% Annexes
\begin{frame}[fragile,c]{Appendices}
	\begin{itemize}
		\item Appendices are sections using an alphanumeric numbering (A,
			A.1, \ldots).
		\item Sections following the \cmd{appendix} command are considered appendices.
		\item In the section title, ```Chapter'' is changed to ``Appendix''.
	\end{itemize}
\end{frame}

% Structure logique d'un livre
\begin{frame}[fragile]{A book's logical structure}
	\framesubtitle{book, memoir and hecthese classes}

\begin{onlyenv}<1>
\begin{codesource}
	\frontmatter
\end{codesource}	
	\begin{itemize}
		\item preface, table of contents, etc.
		\item roman page numbering (i, ii, \ldots)
		\item unnumbered chapters
	\end{itemize}
\begin{codesource}
	\mainmatter
\end{codesource}	
	\begin{itemize}
		\item the document's main content
		\item arabic page numbering starting at 1
		\item numbered chapter
	\end{itemize}
\end{onlyenv}

\begin{onlyenv}<2>
\begin{codesource}
	\backmatter
\end{codesource}
	\begin{itemize}
		\item everything else (bibliography, index, etc.)
		\item page numbering continues 
		\item unnumbered chapters
	\end{itemize}
\end{onlyenv}
\end{frame}

\subsection{Table of contents and in-text references}

% Table des matières
\begin{frame}[fragile,c]{Table of contents}
	
	\begin{itemize}
		\item The table of contents is automatically generated with \cmd{tableofcontents}.
		\item It requires \textbf{many} compilations.
		\item Unnumbered sections are not included.
		\item With the \textbf{hyperref} package, \cmd{tableofcontents} also generates the .pdf file's table
		of contents.
		\pause
		\item The memoir document class also provides the \cmd{tableofcontents*} command
			which excludes the table of contents from the table of contents.
		\pause
		\item \cmd{listoffigures} generates the list of figures.
		\item \cmd{listoftables} generates of the list of tables.
	\end{itemize}

\end{frame}

% Étiquettes et renvois automatiques
\begin{frame}[fragile]{Labels and cross-references}
	\framesubtitle{Because your computer will do it better than you\ldots}
	\begin{onlyenv}<1>
		\begin{itemize}
			\item \textbf{NEVER} refer to a section, an equation, a table, etc., manually.
			\item ``Name'' an element with \cmd{label}
			\item Refer to that element using its name with \cmd{ref}
			\item Requires 2 or 3 compilations
		\end{itemize}
	
\begin{codesource}
	\section{Definitions}
		\label{sec:definitions}
	
		Lorem ipsum dolor sit amet, consectetur adipiscing elit, 
		sed do eiusmod tempor incididunt ut labore et dolore magna aliqua. 
		Ut enim ad minim veniam, quis nostrud exercitation ullamco laboris 
		nisi ut aliquip ex ea commodo consequat.
	
	\section{Historique}
		As seen in section \ref{sec:definitions}...
\end{codesource}
	\end{onlyenv}
	\begin{onlyenv}<2>
		\begin{itemize}
			\item The \textbf{hyperref} package inserts hyperlinks with the in-text references
				in the .pdf files.
			\item The \cmd{autoref\{\}} command allows us to:
				\begin{enumerate}
					\item automatically name the reference type (section, equation, table, etc.);
					\item convert to a hyperlink the reference's text \textbf{and} number.
\begin{codesource}
	As seen in \autoref{sec:definitions}...
\end{codesource}
				\end{enumerate}
			\item The \cmd{pageref\{\}} command refers to a specific page.
			\item The \textbf{amsmath} provides the \cmd{eqref\{\}} command for
				equation referring.
		\end{itemize}
	\end{onlyenv}
\end{frame}