% version-2
\section{Organisation d'un document}

% ---------------------- %
% Titre et page de titre %
% ---------------------- %

\begin{frame}[fragile]

	\frametitle{Titre et page de titre}
	
	\begin{itemize}
		
		\item Mise en forme automatique
		
		\begin{codesource}
	%% préambule
	\title{Titre du document}
	\author{Prénom Nom}
	\date{31 octobre 2014} % automatique si omis
	%% corps du document
	\maketitle
		\end{codesource}
	
		\item Mise en forme libre
		
		\begin{columns}
			\column{.4\textwidth}			
			\textbf{Classes standards}
			
			\begin{codesource}
	\begin{titlepage}
		...
	\end{titlepage}
			\end{codesource}
		
			\column{.4\textwidth}		
			\textbf{Classes memoir et hecthese}
			
			\begin{codesource}
	\begin{titlingpage}
		...
	\end{titlingpage}
			\end{codesource}
			\end{columns}
		
	\end{itemize}
\end{frame}

% ------ %
% Résumé %
% ------ %

\begin{frame}[c,fragile]

	\frametitle{Résumé}
	
	\begin{itemize}
		\item Classes \textbf{article}, \textbf{report} ou \textbf{memoir} : résumé créé avec l’environnement
		
		\begin{codesource}
	\begin{abstract}
		...
	\end{abstract}
		\end{codesource}
	
		\item Classe \textbf{hecthese} : résumés français et anglais traités comme des chapitres normaux (non numérotés)
	\end{itemize}
\end{frame}

% -------- %
% Sections %
% -------- %

\begin{frame}[fragile]

	\frametitle{Sections}
	
	\begin{itemize}
		\item Découpage du document en sections avec les commandes
		
		\begin{codesource}
	\part{titre}
	\chapter{titre}
	\section{titre}
	\subsection{titre}
		\end{codesource}
	
		\begin{codesource}
	\subsubsection{titre}		% à éviter dans un livre
		\end{codesource}
	
		\begin{codesource}
	\paragraph{titre}				% jamais (?) utilisé
	\subparagraph{titre}		% idem
		\end{codesource}
	
		\item Numérotation automatique
		\item Commande suivie d’une * = section non numérotée
		\item Titre «court» en argument optionnel
	\end{itemize}
\end{frame}

% ------- %
% Annexes %
% ------- %

\begin{frame}[c,fragile]
	
	\frametitle{Annexes}
	
	\begin{itemize}
		\item Les annexes sont des sections ou des chapitres avec une numérotation alphanumérique (A, A.1, ...)
		\item Sections suivantes identifiées comme des annexes par la
		commande \\
		\lstinline|\appendix|
		\item Dans le titre, «Chapitre» changé pour «Annexe» le cas échéant
	\end{itemize}
\end{frame}

% -------- %
% Exercice %
% -------- %

\begin{frame}

	\frametitle{Exercice \thenoExercice}
	
	\begin{enumerate}
		\item Ajoutez un titre, un(e) auteur(e) et la date au préambule de votre document \texttt{.tex}.
		\item Insérez la commande permettant de mettre en forme automatiquement une page titre
			au début du corps du document.
		\item Encadrez le premier paragraphe de votre document dans l'environnement \texttt{abstract}.
		\item Divisez votre document en deux chapitres et donnez un titre à chacun d'entre eux.
		\item Divisez chaque document en au moins une section et sous-section et donnez un titre à chacune d'entre elles.
		\item Compilez votre document et observez le résultat.
		\item Changes la classe de document actuelle de votre document pour une autre (\texttt{article} ou \texttt{book}), recompilez votre document et observez la différence.
	\end{enumerate}
\end{frame}
\stepcounter{noExercice}

% ---------------------------- %
% Structure logique d'un livre %
% ---------------------------- %

\begin{frame}[fragile]

	\frametitle{Structure logique d'un livre}
	\framesubtitle{(classes book, memoir, hecthese)}
	
	\lstinline|\frontmatter|	
	
	\begin{itemize}
		\item préface, table des matières, etc.
		\item numérotation des pages en chiffres romains (i, ii, ...)
		\item chapitres non numérotés
	\end{itemize}
	
	\lstinline|\mainmatter|	
	
	\begin{itemize}
		\item le contenu à proprement parler
		\item numérotation des pages à partir de 1 en chiffres arabes
		\item chapitres numérotés
	\end{itemize}

\end{frame}

% ------------------------------------ %
% Structure logique d'un livre (suite) %
% ------------------------------------ %

\begin{frame}[c,fragile]

	\frametitle{Structure logique d'un livre (suite)}
	\framesubtitle{(classes book, memoir, hecthese)}
	
	\lstinline|\backmatter|
	
	\begin{itemize}
		\item tout le reste (bibliographie, index, etc.)
		\item numérotation des pages se poursuit
		\item chapitres non numérotés
	\end{itemize}
\end{frame}

% ------------------ %
% Table des matières %
% ------------------ %

\begin{frame}[fragile]

	\frametitle{Table des matières}
	
	\begin{itemize}
		\item Table des matières produite automatiquement avec \\
		\lstinline|\tableofcontents|
		\item Requiert plusieurs compilations
		\item Sections non numérotées pas incluses
		\item Avec \textbf{hyperref}, produit également la table des matières du fichier PDF
		
		\pause
		
		\item Classe memoir fournit également \\
		\lstinline|\tableofcontents*| \\
		qui n’insère pas la table des matières dans la table des matières
		
		\pause
		
		\item Aussi disponibles : \\
		\lstinline|\listoffigures| \\
		\lstinline|\listoftables| \\
		(et leurs versions * dans \textbf{memoir})
	\end{itemize}
\end{frame}

% ---------------------------------- %
% Étiquettes et renvois automatiques %
% ---------------------------------- %

\begin{frame}[c,fragile]
	
	\frametitle{Étiquettes et renvois automatiques}
	\framesubtitle{Parce que l'ordinateur le fera mieux que vous}
	
	\begin{itemize}
		\item Ne \textbf{jamais} renvoyer manuellement à un numéro de section,
		d’équation, de tableau, etc.
		\item « Nommer » un élément avec \lstinline|\label|
		\item Faire référence par son nom avec \lstinline|\ref|
		\item Requiert 2 à 3 compilations
	\end{itemize}
\end{frame}

% ------------------------------------------ %
% Étiquettes et renvois automatiques (suite) %
% ------------------------------------------ %

\begin{frame}[c,fragile]

	\frametitle{Étiquettes et renvois automatiques (suite)}
	\framesubtitle{Parce que l'ordinateur le fera mieux que vous}
	
	\begin{codesource}
	\section{Définitions}
	\label{sec:definitions}
	Lorem ipsum dolor sit amet, consectetur
	adipiscing elit. Duis in auctor dui. Vestibulum
	ut, placerat ac, adipiscing vitae, felis.
	\section{Historique}
	Tel que vu à la section \ref{sec:definitions},
	on a...
	\end{codesource}
\end{frame}

% ----------------------- %
% Renvois automatiques ++ %
% ----------------------- %

\begin{frame}[c,fragile]

	\frametitle{Renvois automatiques ++}
	
	\begin{itemize}
		\item Paquetage \textbf{hyperref} insère des hyperliens vers des renvois dans les fichiers
		.pdf
		\item Commande \lstinline|\autoref{}| permet de
		
			\begin{enumerate}
				\item nommer automatiquement le type de renvoi (section, équation, tableau, etc.)
				\item transformer en hyperlien le texte \textbf{et} le numéro
			\end{enumerate}
	\end{itemize}
\end{frame}