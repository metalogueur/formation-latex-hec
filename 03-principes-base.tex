% version-2
\section{Principes de base}

% --------- %
% Rédaction %
% --------- %

\begin{frame}

	\frametitle{Rédaction}
	
	\begin{itemize}
		\item On se concentre sur le contenu et la \textbf{structure} du document, pas
		sur son \textbf{apparence}.
		
		\texttt{\textbackslash textbf\{titre\} {\faArrowRight} \textbackslash section\{titre\}}
		
		\texttt{\textbackslash textit\{texte\} {\faArrowRight} \textbackslash emph\{texte\}}
		
		\item Apparence prise en charge par {\LaTeX} et généralement préférable	de ne pas la modifier.
		\item Mots séparés par un ou plusieurs \textbf{espaces}.
		\item Paragraphes séparés par une ou plusieurs \textbf{lignes blanches}.
		\item Utilisation de \textbf{commandes} pour indiquer la structure du texte
	\end{itemize}

\end{frame}

% ----------------------- %
% Structure d'un document %
% ----------------------- %

\begin{frame}[fragile,c]

	\frametitle{Structure d'un document}
	
	Un fichier source {\LaTeX} est toujours composé de deux parties :
	
	\begin{codesource}
	\documentclass[11pt,french]{article}		
	\usepackage{babel}
	\usepackage[autolanguage]{numprint}
	\usepackage[utf8]{inputenc}
	\usepackage[T1]{fontenc}
	
	\begin{document}	
		Lorem ipsum dolor sit amet, consectetur
		adipiscing elit. Donec quam nulla, bibendum
		vitae ipsum vel, fermentum pellentesque orci.		
	\end{document}
	\end{codesource}

	\begin{textblock*}{\linewidth}(85mm,42mm)
		\Large\bfseries
		\faArrowLeft\ Préambule
	\end{textblock*}

	\begin{textblock*}{\linewidth}(85mm,57mm)
		\Large\bfseries
		\faArrowLeft\ Corps du document
	\end{textblock*}
	
\end{frame}

% ------------------ %
% Classe de document %
% ------------------ %

\begin{frame}[fragile]

	\frametitle{Classe de document}
	
	\begin{itemize}
		\item La première commande du préambule est normalement la déclaration de la classe
	\begin{codesource}
	\documentclass[options]{classe}
	\end{codesource}
				
		\pause
		
		\item Principales classe : \\
			article, book, letter, report \\
			memoir, hecthese \\
			slides, beamer
		
		\pause
		
		\item Principales options : \\
			10pt, 11pt, 12pt \\
			oneside, twoside \\
			openright, openany \\
			article (classe memoir)
	\end{itemize}
\end{frame}

% -------- %
% Exercice %
% -------- %

\begin{frame}[c]

\frametitle{Exercice \thenoExercice}

Travaillez à partir de votre document \texttt{.tex}. À chaque étape de l'exercice, compilez votre
document pour observer les résultats.

\begin{enumerate}
	\item Compilez le document avec la classe \textbf{article}, puis avec la classe
		\textbf{book}.
	\item Changez les options de votre classe de document, par exemple en changeant la taille de la
		police de caractères (10pt, 11pt, 12pt).
	\item Ajoutez du texte en français (avec des diacritiques).
\end{enumerate}
\end{frame}
\stepcounter{noExercice}

% ---------- %
% Paquetages %
% ---------- %

\begin{frame}[c,fragile]

	\frametitle{\emph{Packages}}
	
	\begin{itemize}
		\item Permettent de modifier des commandes ou d’ajouter des	fonctionnalités au système
		\item Chargés dans le préambule avec
		
	\begin{codesource}
	\usepackage{package}
	\usepackage[options]{package}
	\usepackage{package1,package2,...}
	\end{codesource}
	\end{itemize}
\end{frame}

% --------- %
% Commandes %
% --------- %

\begin{frame}[fragile]

	\frametitle{Commandes}
	
	\begin{itemize}
		\item Débutent toujours par \textbackslash
		\item Formes générales : 
			
	\begin{codesource}
	\nomcommande[arg_optionnel]{arg_obligatoire}
	\nomcommande*[arg_optionnel]{arg_obligatoire}
	\nomcommande
	\end{codesource}
					
		\item Arguments obligatoires entre \{  et \}.
		\item Arguments optionnels entre [ et ].
		\item Commande sans argument : le nom se termine par tout caractère	qui n’est pas une lettre (y compris l’espace).
		\item Portée d’une commande limitée à la zone entre \{  et \}.
	\end{itemize}

\end{frame}

% -------------- %
% Environnements %
% -------------- %

\begin{frame}[fragile]

	\frametitle{Environnements}
	
	\begin{itemize}
		\item Délimités par
		
	\begin{codesource}
	\begin{environnement}
		...
	\end{environnement}
	\end{codesource}
	
		\item Contenu de l’environnement traité différemment du reste du texte
		\item Changements s’appliquent uniquement à l’intérieur de l’environnement
	\end{itemize}

\end{frame}

% ------------ %
% Commentaires %
% ------------ %

\begin{frame}[fragile]
	
	\frametitle{Commentaires}
	
	\begin{itemize}
		\item Le symbole \% sert à identifier les commentaires dans le code	source
		\item Tout ce qui suit \% sur la ligne est ignoré
		
		\begin{codesource}
	texte % ignoré par LaTeX	
		\end{codesource}
	\end{itemize}
\end{frame}

% ------------------- %
% Caractères spéciaux %
% ------------------- %

\begin{frame}[fragile]

	\frametitle{Caractères spéciaux}
	
	\begin{itemize}
		\item Caractères réservés par \TeX  : \\
		\lstinline|# $ & ~ _ ^ % { }|
		
		\item Pour les utiliser, précéder par \textbackslash  :\\
		\begin{tabular}{lcr}
			\lstinline|\#|  \#	&	\lstinline|\$|  \$	&	\lstinline|\%|  \% \\
			\lstinline|\_|  --	&	\lstinline|\{|  \{	&	\lstinline|\}|  \}
		\end{tabular}
	
		\item On écrira donc
		
		\begin{columns}
			\column{.4\textwidth}
			
			\lstinline|L'augmentation de 2~\$ repr{\'e}sente|
			\lstinline|une hausse de 5~\%.|
			
			\column{.4\textwidth}
			
			L’augmentation de 2~\$ représente une hausse de 5~\%.
		\end{columns}
	\end{itemize}

\end{frame}

% --------------------------- %
% Caractères spéciaux (suite) %
% --------------------------- %

\begin{frame}[fragile]

	\frametitle{Caractères spéciaux (suite)}
	
	\begin{itemize}
		\item Espace insécable : \textasciitilde
		
		\begin{codesource}
	M.~Tremblay me doit 200~\$.
		\end{codesource}
	
		\item Guillemets :
		
		\begin{columns}
			\column{.4\textwidth}
			
			\lstinline|``guillemets anglais''|
			
			\lstinline[literate={«}{{\og}}1{»}{{\fg}}1{ç}{{\c c}}1]|«guillemets français»|
			
			\column{.4\textwidth}
			
			``guillemets anglais''
			
			«guillemets français»
		\end{columns}
	
		\item Trait d'union, tiret demi-cadratin, tiret cadratin
		
		\begin{columns}
			\column{.263 \textwidth}
				\lstinline|-|  -
			\column{.263 \textwidth}
				\lstinline|--|  --
			\column{.263 \textwidth}
				\lstinline|---|  ---
		\end{columns}
	\end{itemize}
\end{frame}

% -------------------------------------------- %
% LaTeX en français -- préambule pour pdfLaTeX %
% -------------------------------------------- %

\begin{frame}[fragile]

	\frametitle{{\LaTeX} en français -- préambule pour pdf\LaTeX}
	
	Il faut charger un certain nombre de paquetages pour franciser {\LaTeX}.
	
	\begin{codesource}
	\documentclass[french]{memoir}
	\usepackage{babel}
	\usepackage[autolanguage]{numprint}
	\usepackage[utf8]{inputenc}
	\usepackage[T1]{fontenc}
	\usepackage{icomma}
	\end{codesource}

	\pause
	
	\begin{itemize}
		\item \textbf{babel} : traduction des mots-clés prédéfinis, typographie française, coupure de mots, document multilingue
		
		\pause
		
		\item \textbf{inputenc} et \textbf{fontenc} : lettres accentuées dans le code source
		
		\pause
		
		\item \textbf{icomma} : virgule comme séparateur décimal
		
		\pause
		
		\item \textbf{numprint} : espace comme séparateur des milliers
	\end{itemize}
\end{frame}

% ---------- %
% Exercice 3 %
% ---------- %

\begin{frame}[c]

\frametitle{Exercice 3}

Question de voir ce que {\LaTeX} peut faire, compiler le document élaboré
\texttt{exercice\_demo.tex} de la manière suivante :

\begin{enumerate}
	\item une fois avec LaTeX ;
	\item une fois avec BibTeX ;
	\item deux à trois fois avec LaTeX.
\end{enumerate}

\end{frame}

% ---------- %
% Exercice 4 %
% ---------- %

\begin{frame}[fragile]

\frametitle{Exercice 4}

Utiliser le fichier \texttt{exercice\_classe+paquetages.tex}.

\begin{enumerate}
	\item Compiler le fichier tel que fourni.
	\item Changer la police du document pour 11 points, puis 12 points.
	Observer l’effet sur les marges et sur la coupure automatique
	des mots.
	\item Activer le paquetage \textbf{icomma} en supprimant le symbole \% au
	début de la ligne dans le préambule. Observer l’effet sur la
	formule mathématique.
	\item Charger le paquetage \textbf{numprint} avec l’option \texttt{autolanguage}
	(\emph{après} le paquetage \textbf{babel}). Dans le code source de la formule
	mathématique, changer \texttt{10 000} pour \texttt{\textbackslash nombre\{10000\}}
	et observer le résultat.
\end{enumerate}

\end{frame}