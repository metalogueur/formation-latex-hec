% version-2
\small

% ------------- %
% Page de titre %
% ------------- %

\begin{frame}
	
	\tiny
	Benoit Hamel \\
	Technicien en documentation, Bibliothèque
	\vfill
	{\Huge\bfseries Rédaction avec \\
	\textbackslash \texttt{title}\{\textrm{\LaTeX}\}}
	\vfill
	Édition HEC Montréal
	
\end{frame}

% --------------- %
% Page de licence %
% --------------- %

\begin{frame}
	
	\tiny
	{\faCopyright} 2016 Vincent Goulet pour la
	\href{https://ctan.org/pkg/formation-latex-ul}{version originale}.
	
	{\faCreativeCommons} Cette création est mise à disposition selon le contrat
	\href{http://creativecommons.org/licenses/by-sa/4.0/deed.fr}%
	{Attribution-Partage dans les mêmes conditions 4.0 International de Creative Commons}.
	En vertu de ce contrat, vous êtes libre de :
	
	\begin{itemize}
		\item partager -- reproduire, distribuer et communiquer l’oeuvre ;
		\item remixer -- adapter l’oeuvre ;
		\item utiliser cette oeuvre à des fins commerciales.
	\end{itemize}

	Selon les conditions suivantes :
	
	\begin{itemize}
		
		\item \textbf{Attribution} -- Vous devez créditer l’oeuvre, intégrer un lien vers le contrat et indiquer si des modifications ont été effectuées	à l’oeuvre. Vous devez indiquer ces informations par tous les moyens possibles, mais vous ne pouvez suggérer que l’Offrant vous soutient ou soutient la façon dont vous avez utilisé son oeuvre.
		
		\item \textbf{Partage dans les mêmes conditions} — Dans le cas où vous modifiez, transformez ou créez à partir du matériel composant l’oeuvre originale, vous devez diffuser l’oeuvre modifiée dans les même conditions, c’est à dire avec le même contrat avec	lequel l’oeuvre originale a été diffusée.
		
	\end{itemize}

\end{frame}

% ------------------------- %
% Pré-requis à la formation %
% ------------------------- %

\begin{frame}
	
	\frametitle{Pré-requis à la formation}
	
	\begin{enumerate}
		
		\item Installer une distribution {\LaTeX} sur votre poste de travail ; nous
		recommandons la distribution \href{https://www.tug.org/texlive/}{\TeX\ Live}.
		
			\begin{itemize}
				\item \href{https://youtu.be/7MfodhaghUk}{Installation sur Windows}
				\item \href{https://youtu.be/kA53EQ3Q47w}{Installation sur MacOS}
			\end{itemize}
		
		\item Installer un éditeur de code intégré sur votre poste de travail ; nous recommandons
		\href{https://www.texstudio.org/}{\TeX studio}.
		
		\item \textbf{ALTERNATIVE} : Vous ouvrir un compte dans \href{https://www.overleaf.com/}{Overleaf}.
		
		\item Composer un document très simple de type \emph{Hello World!}
		
			\begin{itemize}
				\item \textbf{EXERCICE!}
			\end{itemize}
	\end{enumerate}
	
\end{frame}

% -------- %
% Sommaire %
% -------- %

\begin{frame}{Sommaire}
	\tableofcontents
\end{frame}