% Création d'un document LaTeX

\section{Création d'un document \LaTeX}

\subsection{Structure d'un document}

% Un document LaTeX dans sa plus simple expression
\begin{frame}[c,fragile]{Un document \LaTeX\ dans sa plus simple expression}

	Dans un éditeur de texte, ouvrez un nouveau fichier et saisissez le code suivant :
	
\begin{codesource}	
	\documentclass{article}
	
	\begin{document}		
		Ceci est mon premier document LaTeX et j'en suis fier.
	\end{document}

\end{codesource}

	Sauvegardez votre fichier avec l'extension \texttt{.tex} et compilez-le. Observez le résultat.
	
\end{frame}

% Les parties d'un document : la déclaration de la classe de document
\begin{frame}[c,fragile]{Les parties d'un document}
	\framesubtitle{La déclaration de la classe de document}
	\begin{itemize}
		\item Un document commence toujours par la \textbf{commande} \cmd{documentclass}.
		
\begin{codesource}
	\documentclass[options]{classe}
\end{codesource}

		\item La \lien{https://en.wikibooks.org/wiki/LaTeX/Document_Structure\#Document_classes}{classe de document}
			détermine le type de document.		
		\item Plusieurs options de classe peuvent être utilisées pour changer la mise en page.
	\end{itemize}
\end{frame}

% Les parties d'un document : le corps du document
\begin{frame}[c,fragile]{Les parties d'un document}
	\framesubtitle{Le corps du document}
	Le contenu du document est rédigé dans l'\textbf{environnement} \texttt{document},
	entre les commandes \cmd{begin\{document\}} et \cmd{end\{document\}}.
\begin{codesource}
	\documentclass[options]{classe}
	
	\begin{document}
		Le contenu du document est rédigé ici...
	\end{document}
\end{codesource}
\end{frame}

% Les parties d'un document : le préambule
\begin{frame}[c,fragile]{Les parties d'un document}
	\framesubtitle{Le préambule}
	Tout ce qui se trouve avant la commande \cmd{begin\{document\}} constitue le \textbf{préambule} du document.
	
\begin{codesource}
	\documentclass[options]{classe}
	
	%% Ici se trouve le préambule du document...
	
	\begin{document}
		Ici se trouve le contenu du document...
	\end{document}
\end{codesource}

	Dans le préambule se trouvent:
	\begin{itemize}
		\item des \emph{packages};
		\item des commandes de configuration;
		\item des commandes et des environnements personnalisés;
		\item des métadonnées.
	\end{itemize}
\end{frame}

% Création d'un document plus complexe
\begin{frame}[c]{Création d'un document plus complexe}
	\begin{itemize}
		\item Ouvrez le premier document \texttt{.tex} que vous avez créé.
		\item Rendez-vous sur le \lien{http://www.hec.ca/nouvelles/index.html}{site des nouvelles de HEC Montréal}.
		\item Copiez-collez l'intégralité du contenu d'un article dans votre document.
		\item Sauvegardez et compilez votre document, puis observez le résultat.
	\end{itemize}
\end{frame}

\subsection{Personnalisation de \LaTeX}

% Préambule : les packages
\begin{frame}[fragile,c]{Préambule}
	\framesubtitle{Les \emph{packages}}
	Les \emph{packages} permettent de \textbf{modifier des commandes} ou d’\textbf{ajouter des fonctionnalités} au système.

	Ils sont chargés dans le préambule avec la commande \cmd{usepackage[options]\{package\}}.

\begin{codesource}
	\documentclass[options]{classe}
	
	\usepackage{package}
	\usepackage[options]{package}
	\usepackage{package1,package2,package3,...}
\end{codesource}

	La documentation de chaque package peut être consultée sur le site du
	\lien{https://ctan.org/}{Comprehensive \TeX\ Archive Network}.
\end{frame}

% Commandes
\begin{frame}[fragile]{Commandes}
	\begin{itemize}
		\item Débutent toujours par un \textbackslash
		\item Formes générales:
\begin{codesource}
	\nomcommande[args_optionnels]{args_obligatoires}
	\nomcommande*[args_optionnels]{args_obligatoires}
	\nomcommande
\end{codesource}
		\item Arguments obligatoires entre \{\ et \}
		\item Arguments optionnels entre [ et ]
		\item Commande sans argument : le nom se termine par tout caractère qui n’est pas une lettre (y
		compris l’espace)
		\item Portée d’une commande limitée à la zone entre \{\ et \}.
	\end{itemize}
\end{frame}

% Environnements
\begin{frame}[fragile,c]{Environnements}
	\begin{itemize}
		\item Délimités par
\begin{codesource}
	\begin{environnement}
		...
	\end{environnement}
\end{codesource}
		\item Contenu de l’environnement traité différemment du reste du texte
		\item Changements s’appliquent uniquement à l’intérieur de l’environnement
	\end{itemize}
\end{frame}

% Commandes et environnements personnalisés
\begin{frame}[c]{Commandes et environnements personnalisés}
	\begin{itemize}
		\item Vous pouvez \textbf{créer} de nouvelles commandes avec \cmd{newcommand}.
		\item Vous pouvez \textbf{modifier} des commandes existantes avec \cmd{renewcommand}.
		\item Vous pouvez \textbf{créer} de nouveaux environnements avec \cmd{newenvironment}.
		\item Vous pouvez \textbf{modifier} des environnements existants avec \cmd{renewenvironment}.
	\end{itemize}
\end{frame}

% LaTeX en français - préambule pour pdfLaTeX
\begin{frame}[fragile]{\LaTeX\ en français -- préambule pour pdf\LaTeX}
	Il faut charger un certain nombre de \emph{packages} pour franciser \LaTeX.

\begin{codesource}
	\documentclass[french]{hecthese}
	\usepackage[utf8]{inputenc}
	\usepackage[T1]{fontenc}
	\usepackage{babel}
	\usepackage[autolanguage]{numprint}
	\usepackage{icomma}
\end{codesource}

	\pause
	\begin{description}[inputenc et fontenc]
		\item[babel] traduction des mots-clés prédéfinis, typographie française, coupure de mots,
		document multilingue
		
		\pause
		\item[inputenc et fontenc] lettres accentuées dans le code source
		
		\pause
		\item[icomma] virgule comme séparateur décimal
		
		\pause
		\item[numprint] espace comme séparateur de milliers
	\end{description}
\end{frame}