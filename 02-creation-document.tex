% Création d'un document LaTeX

\section{\LaTeX\ document creation}

\subsection{Document structure}

% Un document LaTeX dans sa plus simple expression
\begin{frame}[c,fragile]{The most basic \LaTeX\ document}

	In a text editor, open a new file and write the following code:
	
\begin{codesource}	
	\documentclass{article}
	
	\begin{document}		
		This is my first LaTeX document and I am proud of it.
	\end{document}

\end{codesource}

	Save your file with the \texttt{.tex} extension and compile it. Look at the results.
	
\end{frame}

% Les parties d'un document : la déclaration de la classe de document
\begin{frame}[c,fragile]{The parts of a document}
	\framesubtitle{Document class declaration}
	\begin{itemize}
		\item A document always starts with the \cmd{documentclass} \textbf{command}.
		
\begin{codesource}
	\documentclass[options]{class}
\end{codesource}

		\item The \lien{https://en.wikibooks.org/wiki/LaTeX/Document_Structure\#Document_classes}{document class}
			determines a document's type.		
		\item Many options can be used to change a document's layout.
	\end{itemize}
\end{frame}

% Les parties d'un document : le corps du document
\begin{frame}[c,fragile]{The parts of a document}
	\framesubtitle{Document body}
	A document's content is written inside the \texttt{document} \textbf{environment},
	between the \cmd{begin\{document\}} and \cmd{end\{document\}} commands.
\begin{codesource}
	\documentclass[options]{class}
	
	\begin{document}
		The document's content is written here...
	\end{document}
\end{codesource}
\end{frame}

% Les parties d'un document : le préambule
\begin{frame}[c,fragile]{The parts of a document}
	\framesubtitle{The preamble}
	Everything that is written before the \cmd{begin\{document\}} command is called
	the document \textbf{preamble}.
	
\begin{codesource}
	\documentclass[options]{class}
	
	%% Here lies the document preamble...
	
	\begin{document}
		The document's content is written here...
	\end{document}
\end{codesource}

	In the preamble, you will find:
	\begin{itemize}
		\item packages;
		\item configuration commands;
		\item custom commands and environments;
		\item metadata.
	\end{itemize}
\end{frame}

% Création d'un document plus complexe
\begin{frame}[c]{Creating a more complex document}
	\begin{itemize}
		\item Open the first \texttt{.tex} file you created.
		\item Go to the \lien{http://www.hec.ca/en/news/index.html}{HEC Montréal news web page}.
		\item Copy and paste a whole article in your document.
		\item Save and compile your document, then look at the results.
	\end{itemize}
\end{frame}

\subsection{\LaTeX\ customization}

% Préambule : les packages
\begin{frame}[fragile,c]{Preamble}
	\framesubtitle{Packages}
	Packages are used to \textbf{modify commands} or \textbf{add functionalities} to the system.
	
	They are loaded in the preamble with the \cmd{usepackage[options]\{package\}} command.

\begin{codesource}
	\documentclass[options]{class}
	
	\usepackage{package}
	\usepackage[options]{package}
	\usepackage{package1,package2,package3,...}
\end{codesource}
	
	Each package's documentation can be found on the 
	\lien{https://ctan.org/}{Comprehensive \TeX\ Archive Network} website.
\end{frame}

% Commandes
\begin{frame}[fragile]{Commands}
	\begin{itemize}
		\item Always start with a \textbackslash
		\item General syntax:
\begin{codesource}
	\nomcommande[optional_args]{mandatory_args}
	\nomcommande*[optional_args]{optional_args}
	\nomcommande
\end{codesource}
		\item Mandatory arguments are placed between \{\ and \}
		\item Optional arguments are placed between [ et ]
		\item Commands without arguments : their name ends with any character that isn't a letter, including
			a white space
		\item The scope of a command is limited in the zone between \{\ and \}.
	\end{itemize}
\end{frame}

% Environnements
\begin{frame}[fragile,c]{Environments}
	\begin{itemize}
		\item Delimited by
\begin{codesource}
	\begin{environment}
		...
	\end{environment}
\end{codesource}
		\item An environment's content is treated differently from the remainder
			of the text
		\item Changes only apply inside the environment
	\end{itemize}
\end{frame}

% Commandes et environnements personnalisés
\begin{frame}[c]{Custom commands and environments}
	\begin{itemize}
		\item You can \textbf{create} new commands with \cmd{newcommand}.
		\item You can \textbf{modify} existing commands with \cmd{renewcommand}.
		\item You can \textbf{create} new environments with \cmd{newenvironment}.
		\item You can \textbf{modify} existing environments with \cmd{renewenvironment}.
	\end{itemize}
\end{frame}