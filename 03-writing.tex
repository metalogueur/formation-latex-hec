\section{Writing}

\subsection{Basic formatting}

% Titre, auteur et date du document
\begin{frame}[fragile]{Title, author and date}
	\begin{itemize}
		\item Automatic formatting
\begin{codesource}
	\documentclass{article}
	
	\title{Document title}
	\author{Author name}
	\date{A date}
	
	\begin{document}
		\maketitle
		
		% Document content...
	\end{document}
\end{codesource}
		\item Manual formatting
\begin{codesource}
	\documentclass{article}
	
	\begin{document}
		\begin{titlepage}
			% Title page built manually...
		\end{titlepage}
	\end{document}
\end{codesource}
	\end{itemize}
\end{frame}

% Paragraphes, sauts de lignes et espaces blancs
\begin{frame}[c]{Paragraphs, line breaks and white space}
	\begin{itemize}
		\item \LaTeX\ automatically deletes all extra white spaces.
		\item Line breaks are created with \textbackslash\textbackslash.
		\item There needs to be at least one blank line between paragraphs in
			the code in order to distinguish them in the text.
	\end{itemize}
\end{frame}

% Caractères spéciaux
\begin{frame}{Reserved characters}
	\framesubtitle{\TeX\ reserved characters}
	\begin{description}[\#]
		\item[\#] Argument identifier in commands
		\item[\$] Math mode delimiter
		\item[\&] Column delimiter in tables
		\item[\%] Start of a comment
		\item[\_] Indice (math)
		\item[\textasciicircum] Exponent (math)
		\item[\textasciitilde] No-break space
		\item[\{] Opens a command or environment definition
		\item[\}] Closes a command or environment definition
	\end{description}
	\begin{picture}(0,0)
	\thicklines\color{bleuFonceSecondaire}
	\onslide<2>\put(90,5){\dashbox{1}(53,58){}}
	\onslide<2>\put(97,59){\textbf{\MakeUppercase{To use them:}}}
	\onslide<2>\put(94,55){\parbox[t]{.3\textwidth}{\centering\bfseries\textbackslash \# \\[5pt] %
			\textbackslash \$ \\[5pt] \textbackslash \& \\[5pt] \textbackslash \% \\[5pt] %
			\textbackslash \_ \\[5pt] \textbackslash textasciicircum \\[4pt] %
			\textbackslash textasciitilde \\[4pt] \textbackslash \{ \\[4pt] %
			\textbackslash \} }}
	\end{picture}
\end{frame}

% Caractères spéciaux - la suite
\begin{frame}[fragile]{Reserved characters}
	\framesubtitle{Part deux\ldots}
	\begin{itemize}
		\item Quote marks
			\begin{itemize}
				\item We open english single quotes with (\lstinline|`|)
					and double quotes with (\lstinline|``|). We close them with one
					(\lstinline|'|) or two (\lstinline|''|) apostrophes, depending on the case.
				\item We use chevrons (« and ») to open and close french quotation marks.
					To do this, you must enter the following command in the preamble:
\begin{codesource}
	\frenchbsetup{og=«,fg=»}
\end{codesource}				
			\end{itemize}
		\item You write hyphens with a single (\lstinline|-|) sign, n-dashes with two (\lstinline|--|) signs
		and m-dashes with three (\lstinline|---|) signs.
	\end{itemize}
\end{frame}

% Commentaires
\begin{frame}[c]{Comment}
	\begin{itemize}
		\item To clarify your code (or long documents), it is advised that
			you insert comments in your document.
		\item They always begin with the \texttt{\%} symbol.
		\item Comments are visible in your code but not in the final document.
	\end{itemize}
\end{frame}

\subsection{Text appearance}

% Polices de caractères
\begin{frame}[c]{Fonts}
	\begin{itemize}
		\item By default, all \LaTeX\ documents use the \textrm{Computer Modern} font.
		\item Preferably use high-quality and complete fonts (diacritics, great choice of symbols).
		\item Very few fonts are adapted to maths : Palatino, Times, Lucida (\$) are safe choices.
		\item In the \textbf{hecthese} document class, the mathptmx and mathpazo packages are preloaded,
			so you can choose between the Times and Palatino fonts.
	\end{itemize}
\end{frame}

% Changement d'attribut de la police
\begin{frame}{Font attributes}
	\begin{tabularx}{\textwidth}{XXX}
		\arrayrulecolor{grisPrimaire!40}\hline\hline
		\multicolumn{3}{l}{\textbf{families}}	\\
		\hline
		\textrm{roman}						&	\cmd{rmfamily}		&	\cmd{textrm\{<text>\}}\\
		\texttt{fixed width}				&	\cmd{ttfamily}		&	\cmd{texttt\{<text>\}}\\
		sans serif							&	\cmd{sffamily}		&	\cmd{textsf\{<text>\}}\\
		\hline
		\multicolumn{3}{l}{\textbf{shapes}}	\\
		\hline
		upright								&	\cmd{upshape}		&	\cmd{textup\{<text>\}}\\
		\emph{italic}						&	\cmd{itshape}		&	\cmd{textit\{<text>\}}\\
		\textsl{slanted}					&	\cmd{slshape}		&	\cmd{textsl\{<text>\}}\\
		\textrm{\textsc{small caps}}		&	\cmd{scshape}		&	\cmd{textsc\{<text>\}}\\
		\hline
		\multicolumn{3}{l}{\textbf{series}}	\\
		\hline
		\textmd{medium}						&	\cmd{mdseries}		&	\cmd{textmd\{<text>\}}\\
		\textbf{bold}						&	\cmd{bfseries}		&	\cmd{textbf\{<text>\}}\\
		\hline\hline
	\end{tabularx}

	\begin{picture}(0,0)
		\thicklines\color{bleuFonceSecondaire}
		\onslide<2>\put(38,-7){\dashbox{1}(40,64)[b]{\parbox{.25\textwidth}{\centering\textbf{applies to all following text}}}}
		\onslide<3>\put(92,-7){\dashbox{1}(40,64)[b]{\parbox{.25\textwidth}{\centering\textbf{applies to the text in braces}}}}
	\end{picture}
\end{frame}

% Taille de la police
\begin{frame}{Font size}
	\begin{tabularx}{\textwidth}{l|l}
		\arrayrulecolor{grisPrimaire!40}
		\textbf{Commands} 				& 	\textbf{Rendering}	\\
		\hline
		\cmd{tiny}						&	{\tiny smallest}	\\
		\cmd{scriptsize}				&	{\scriptsize even more smaller}	\\
		\cmd{footnotesize}				&	{\footnotesize smaller}	\\
		\cmd{small}						&	{\small small}	\\
		\cmd{normalsize}				&	{\normalsize normal}	\\
		\cmd{large}						&	{\large large}	\\
		\cmd{Large}						&	{\Large larger}	\\
		\cmd{LARGE}						&	{\LARGE largest}	\\
		\cmd{huge}						&	{\huge huge}	\\
		\cmd{Huge}						&	{\Huge humongus}		
	\end{tabularx}
\end{frame}

% Caractères gras, italiques et soulignés
\begin{frame}[c]{Bold, italics and underline}
	\begin{itemize}
		\item \textbf{Bold} characters: \cmd{textbf\{\}}
		\item Characters in \emph{italics} :
		\begin{itemize}
			\item \cmd{textit\{\}}
			\item \cmd{emph\{\}} -- command of choice
		\end{itemize}
		\item \underline{Underlined} characters : \cmd{underline\{\}}
	\end{itemize}
\end{frame}

